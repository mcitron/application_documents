\documentclass[11pt]{article}
\usepackage{amsmath}
\usepackage{cancel}
\usepackage{xspace}
\usepackage{amssymb}
\usepackage{amsthm}
\usepackage{amscd}
\usepackage{amsfonts}
\usepackage{graphicx}%
\usepackage{fancyhdr}


\theoremstyle{plain} \numberwithin{equation}{section}
\newtheorem{theorem}{Theorem}[section]
\newtheorem{corollary}[theorem]{Corollary}
\newtheorem{conjecture}{Conjecture}
\newtheorem{lemma}[theorem]{Lemma}
\newtheorem{proposition}[theorem]{Proposition}
\theoremstyle{definition}
\newtheorem{definition}[theorem]{Definition}
\newtheorem{finalremark}[theorem]{Final Remark}
\newtheorem{remark}[theorem]{Remark}
\newtheorem{example}[theorem]{Example}
\newtheorem{question}{Question} \topmargin-2cm

\textwidth6in

\setlength{\topmargin}{0in} \addtolength{\topmargin}{-\headheight}
\addtolength{\topmargin}{-\headsep}

\setlength{\oddsidemargin}{0in}
%\newcommand{\alphat}{\ensuremath{\alpha_{\text{T}}}\xspace}
\DeclareRobustCommand{\alphat}{$\alpha_{\text{T}}$}
\DeclareRobustCommand{\met}{$\mbox{$E_\text{T}^{\rm miss}$}\xspace$}


\oddsidemargin  0.0in \evensidemargin 0.0in

\pagestyle{fancy}\lhead{Research Statement} \rhead{August 2005}
\chead{{\large{\bf My Name}}} \lfoot{} \rfoot{\bf \thepage} \cfoot{}

\newcounter{list}

\begin{document}

My research has centred on searching for signatures from new physics models
which may allow fundamental problems in particle physics, such as the origin of 
dark matter, to be resolved. In the three years of my PhD I have taken a 
leading role in the \alphat analysis group searching for generic models of Supersymmetry (SUSY) and dark matter. 
As part of the SUSY group in the CMS collaboration, we used the first data from Run 2 of the LHC 
at the highest ever energies reached at a collider to place strong constraints 
on new physics models. The \alphat analysis group is small and so I had direct involvement 
in most areas of the analysis. This meant I gained experience in rapid data analysis as well
as understanding in how sensitivity to new physics signatures can be maximised
while maintaining a robust analysis. 

In the coming years the jumps in energies and luminosity that marked the previous running of the 
LHC will come to an end. While inclusive style analyses will still play a role in continuing 
to explore ever greater masses, I believe the best chance for discovery will come from 
exploiting the large datasets with more targeted searches. Such searches present particular
challenges and in the following paragraphs I will discuss these challenges and how the experience 
gained during my PhD has prepared me well to meet them.

\section*{Research activities}
\subsection*{Hardware}
For any analysis the trigger strategy is critical as data lost
at this stage cannot be recovered. On joining CMS I have worked on designing effective
trigger algorithms at both the hardware and analysis level. At the hardware level
I worked to design an algorithm to identify jets as well as subtract contributions from simultaneous collisions, 
pile-up (PU), for the level 1 hardware trigger. By taking advantage of the increased granularity 
and flexibility of the upgraded system the new algorithm saw a significant increase in performance.
Such service work is not only an important duty for the collaboration but is also
very important to gain understanding of the detector performance and, as discussed below,
can provide important lessons for analysis.
\subsection*{Analysis}

SUSY is a leading candidate as a BSM theory to resolve problems in the Standard Model (SM).
To naturally provide a higgs mass at $m_H \approx 125~\text{GeV}$, coloured SUSY particles at the TeV
scale may be expected. For R-parity conserving SUSY, these will decay to the 
lightest supersymmetric particle (LSP), a dark matter candidate. The final state 
typically contains hadronic activity (in the form of jets) as well as momentum 
imbalance (\met) from the LSP. When the mass splitting in the SUSY spectra are small, 
discovery may be particularly challenging as the energy in the final state is reduced.

The record energy reached for Run 2 of the LHC provided excellent opportunity for discovery
in the first months of operation. I worked on the \alphat analysis searching
for a new physics final state containing jets and \met. The results from the 
\alphat analysis were among the first to be shown publicly with $2.6{fb}^{-1}$ at the 
November Jamboree and $12.9{fb}^{-1}$ at ICHEP16, for which I gave the successful
approval talk. I held key roles within the analysis and gained important experience 
in rapidly understanding and then analysing data to search for new physics. 

The first task I undertook with the \alphat analysis built on my experience with the 
trigger to design a level one trigger strategy to increase acceptance for compressed models
with lower energies in the final state. The trigger is a crucial part of any search
as data lost at this stage cannot be recovered. Taking advantage of the new algorithm
the strategy I designed was able to significantly increase acceptance for compressed models.

My main responsibility for the \alphat analysis has been the statistical interpretation
of the results of the search. This key role requires a holistic and deep understanding
of all sections of the analysis that are included in the final likelihood model. I have been 
instrumental in efforts to measure systematic uncertainties in data and through simulation and to
ensure their effect with the correct correlation scheme is robustly included in the likelihood 
model with the correct correlation scheme. This experience is extremely valuable for searches 
for new physics, as correctly modelling the backgrounds and their uncertainties in a systematics 
dominated environment may be crucial for discovery.

Another of my responsibilities has been the addition of a \met like dimension to the \alphat
analysis. This was a significant change in strategy as the analysis moved from a simple cut and
count to a template based analysis. This provides signifcant increase in sensitivity to a wide 
range of models, however, also provides challenges to ensure a robust analysis. I acheived this
by ensuring the modeling of the \met shape was validated, and systematic uncertianties derived, 
using signal depleted control regions in data. In addition, systematic effects from known sources were included
with relevant correlation.

Alongside my work in the \alphat analysis I have actively contributed within the SUSY group.
As well as my work on the trigger strategy I was part of a working group that investigated
the reliability of limits for stop production and decay to top and LSP where the mass splitting is
close to the top mass. This required an indepth study of the features of the decay and how it 
may be separated from the background.  

\subsection*{Simplified Likelihood}
Independantly from my group, I worked with a collaborator in CMS to make a proposal 
for additional material to be released by CMS analyses
to allow their searches to be easily reinterpreted by those outside the collaboration. 
This work built on my experience with the statistical framework for $\alpha_T$ as well as being 
informed by my work with the mastercode collaboration. By releasing the predictions and covariances between analysis bins 
a simplified likelihood may be defined for any search. A recommondation to release this information for all analyses
will be made in the SUSY and exotica groups. In addition, I have co-authored a document,
that will be mane public, describing generically how this may be used to reinterpret a search.

\subsection*{Mastercode}
For the MasterCode collaboration I worked on a framework for 
deriving constraints from direct searches for new physics on GUT scale models of supersymmetry 
(SUSY). This required the rapid comprehension, implementation and validation of several 
searches from both the CMS and ATLAS experiments. Finally, I worked on 
showing that through combining several analysis targeting different final states the sensitivity of the 
limit to the non-flavoured sector of the SUSY spectra can be approximately 
removed. These universal limits can be used to greatly reduce the 
time taken to sample a GUT model parameter space. This experience has been useful in understanding
the kinds of models that are well motivated and evade current experimental limits, such as gauge or anomoly mediated
supersymmetry breaking models, and may be targeted in the future. I have also gained an appreciation
of the information needed to reliably reinterpret an analysis.

\section*{Research Plans}

My extensive experience in

\begin{itemize}
\item 
\end{itemize}


%\clearpage\end{CJK*}                              % if you are typesetting your resume in Chinese using CJK; the \clearpage is required for fancyhdr to work correctly with CJK, though it kills the page numbering by making \lastpage undefined
\end{document}


%% end of file `template.tex'.

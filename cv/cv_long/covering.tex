\section{}

Dear James Annott, Joel Goldstein, Dave Newbold, and Annela Seddon,

I am a Research Associate of Imperial College London, working in the
field of experimental high energy physics (HEP). \\

% imperial background. based at international laboratory. 

Mainly teaching.


From a research persective, I have been a collaboration member of the
CMS experiment since 2000, which operates at the Large Hadron Collider
(LHC), CERN, Geneva. The laboratory and its experiments provide
world-class research in the area of fundamental particles and
forces. The CMS experiment is 

a flagship activity at CERN, which is
supported by a 4000-strong international collaboration. 

I hope this curriculum vitae identifies my contributions to the field
of experimental high energy physics, primarily as a CMS collaboration
member, and the ``big science'' environment within which I work gives
context to my achievements and roles within this
structure. \\

My research activities and interests can be summarised as follows. I
have nearly a decade of experience in R\&D, commissioning,
calibration, and operation of large-scale silicon-based tracking
detectors and their control and readout systems. I have also led
activities related to the analysis of data collected by the CMS
detector for several years, with expertise in the extraction of
physics observables from the collision data collected by CMS, and in
searches for evidence of new fundamental physics processes at the
LHC. My publications, as a leading author, that resulted from these
studies can be considered high impact by several measures, including
the accumulation of well in excess of 1000 citations. 

I have been posted on two separate Long Term Attachments at CERN,
prior to and during Run I and II of the LHC, to coordinate activities
in these areas. I have held various positions of responsibility within
the CMS collaboration. I currently convene one of the larger physics
working groups, the ``inclusive SUSY'' group, within CMS.

I have considerable experience in teaching and assessing undergraduate
students during laboratory and computing courses. I lecture to
postgraduate students each academic year. I supervise closely the work
of both postdoctoral researchers and postgraduate students of Imperial
and other institutes in the UK (e.g. Bristol) and abroad. I regularly
participate in outreach activities, such as masterclasses for school
students, summer schools for undergraduates, or as a mentor for school
students wishing to gain work experience in a research environment. \\

My experience and reputation in both the detector and physics
communities means that I am well placed to steer and contribute to the
future research plans of the Bristol HEP group, which has a strong
presence in the detector upgrade projects of CMS, and ensure the group
maximally exploits the physics potential of its commitments to CMS. \\


The HEP group at Bristol plays a continuous and crucial role within
the team, which has produced several high-impact publications that
have yielded an aggregated citation count in excess of
800~\cite{pub-alphat6, pub-alphat5, pub-alphat4, pub-alphat3,
  pub-alphat2, pub-alphat1}. \\






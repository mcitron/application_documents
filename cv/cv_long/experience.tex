\section{Academic Experience}

%\cventry{2013--2017}{PhD in High Energy Physics}{}{Imperial College London}{}{Search for supersymmetry in 13TeV proton-proton collisions with the CMS experiment at the LHC.}%
\cventry{2017--present}{Postdoctoral Scholar, Experimental High Energy Physics}{University of California, Santa Barbara}{}{}{}
\vspace{0.1cm}
   \cvitem{}{As the co-convenor of the CMS long-lived particle subgroup I am responsible for organising and reviewing the CMS long-lived particle analyses. In addition, with my co-convenor I organised the first CMS long-lived particle workshop to identify and find solutions to common problems in long-lived particles searches as well as develop new ideas for the future.}
\vspace{0.1cm}
   \cvitem{}{Pioneered the use of calorimetry timing information to search for hadronic decays of neutral long-lived particles. This lead to the first published search for BSM physics with the full $137~\text{fb}^{-1}$ 13 TeV data set from the CMS experiment.}
\vspace{0.1cm}
   \cvitem{}{Developing new hardware triggers for long-lived particles using new information from the CMS hadronic calorimeter.}
\vspace{0.1cm}
   \cvitem{}{Developing a search for hadronic decays in the CMS muon system. This represents the first such use of the CMS muon system and has the potential to provide new sensitivity for highly displaced decays.}
\vspace{0.1cm}
   \cvitem{}{Extensive leadership roles in the milliQan experiment including the design, construction and operation of a prototype detector that has been collecting data since 2017.}
\vspace{0.1cm}
   \cvitem{}{As physics analysis coordinator for the milliQan experiment (since 2018), I coordinated the calibration of the milliQan prototype, characterised the backgrounds faced by the detector, and designed and executed a search for millicharged particles, which has been accepted for publication in PRD.}
\vspace{0.1cm}
   \cvitem{}{Using the lessons from the prototype detector, I proposed improvements to the design of the full milliQan detector and designed a substantially upgraded prototype detector, the phase-1 milliQan detector, that can be built and operated during the LHC Run 3. This detector provides significant improvements in sensitivity compared to the prototype with a much smaller cost than a full-scale detector.}
   \vspace{0.1cm}
   \cvitem{}{I received the Harvey L. Karp Discovery Award, which will provide \$$47$k to construct the phase-1 milliQan detector.}
\vspace{0.1cm}
   \cvitem{}{Worked with collaborators to develop new tools for reinterpreting the likelihood model of searches. This formed the basis of a paper published in JHEP.}
\vspace{0.1cm}
\cvitem{}{Member of the LHC LLP community organisational committee}
\vspace{0.2cm}

\cventry{2013--2017}{PhD, Experimental High Energy Physics}{Imperial College London}{}{}{}
\vspace{0.1cm}
   \cvitem{}{Search for supersymmetry in 13 TeV proton-proton collisions with the CMS experiment at the LHC.}
\vspace{0.1cm}
   \cvitem{}{Held a key role in ensuring the results from the $\alpha_T$~search for supersymmetry were among the first to be shown publicly with  $2.3~\text{fb}^{-1}$ of data at CERN in November 2015
   and with $12.9~\text{fb}^{-1}$ of data at ICHEP16.}
\vspace{0.1cm}
   \cvitem{}{Responsible for developing the statistical analysis of the search as well as looking at new variables and strategies to optimise sensitivity to a wide range of models.}
% \vspace{0.1cm}
%    \cvitem{}{Worked on development of python framework for analysis.}
\vspace{0.1cm}
   \cvitem{}{For the CMS collaboration, worked to develop the jet algorithm for the Stage 2 upgrade of the CMS Level One trigger, particularly focusing on novel methods of pile-up subtraction.}
% \vspace{0.1cm}
%    \cvitem{}{Undertook trigger shifts for CMS.}
% \vspace{0.1cm}
%    \cvitem{}{ Moved to France to work at CERN (2014-2016)}
\vspace{0.1cm}
   \cvitem{}{Honorary Research Associate at Bristol University (2015--2017).}
\vspace{0.1cm}
   \cvitem{}{Member of a phenomenology collaboration (MasterCode) with which I have worked to 
   determine the impact of direct searches at CMS and ATLAS on the allowed parameter space of GUT scale models.}
\vspace{0.2cm}

\cventry{Summer 2012}{Undergraduate research opportunities programme (UROP)}{Imperial College London}{}{}{}
\vspace{0.1cm}
\cvitem{}{As part of the MasterCode collaboration, recast searches for supersymmetry at the LHC for use in a scan 
of GUT scale models to determine the impact of these searches on the SUSY parameter space.}
\vspace{0.1cm}
\cvitem{}{Studied the collider signatures of long-lived supersymmetric taus as part of a proposal for a new search.}
\vspace{0.1cm}
\cvitem{}{Gained experience with event generation (PYTHIA) as well as fast detector simulation (DELPHES).}
\vspace{0.1cm}
\cvitem{}{Both the work within the mastercode collaboration and the study of supersymmetric taus contribute to publications for which I am a co-author.}
\vspace{0.2cm}

\cventry{Summer 2011}{DAAD Scholarship}{Max-Planck-Institut f\"{u}r Kernphysik (MPIK), Heidelberg}{}{}{}
\vspace{0.1cm}
\cvitem{}{Research placement working with the electron beam ion trap (EBIT) group at MPIK.}
\vspace{0.1cm}
\cvitem{}{Assisted the EBIT group at MPIK with research into highly charged ions and gained experience in hardware and data analysis from work developing an EBIT into a Penning trap.}
\vspace{0.2cm}

% \cventry{Summer 2010}{Work placement}{Scottish Universities Environmental Research Centre (SUERC)}{}{}{}
% \vspace{0.1cm}
% \cvitem{}{Research placement working with the luminescence department at SUERC.}
% \vspace{0.1cm}
% \cvitem{}{Gained experience in experimental techniques and data analysis 
% through investigating the luminescent properties of different materials with the aim of using luminescence as an environmental dosimeter.}

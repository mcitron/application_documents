\section{Overview}

\cvitem{\textbf{Context}}{I am a Research Associate of Imperial
  College London working in the field of experimental high energy
  physics (HEP). I am a long-time collaboration member of the CMS
  experiment that operates at the Large Hadron Collider (LHC), CERN,
  Geneva. The laboratory and its experiments provide world-class
  research in the area of fundamental particles and
  forces. Participation in the CMS experiment is one of the flagship
  activities of the Imperial College HEP group and a long-term core
  component of its research programme.}

\cvitem{}{This curriculum vitae identifies my contributions to the
  field of experimental high energy physics. The field is
  characterised by large-scale experiments supported by large
  international collaborations, with membership numbers sometimes in
  the thousands. Within the UK, financial support for participation in
  these experiments is typically provided at the group level via
  grants from a central funding agency that support the work of tens
  of staff within the group. The publications that represent the
  research output of an experiment are typically signed by the entire
  collaboration, thus articles with hundreds of authors are not
  uncommon. In the case of CMS, the collaboration is approximately
  4000-strong and international in nature, which gives context to my
  achievements and roles within this structure. All statements made
  below are supported by further details provided within this
  curriculum vitae and publications list.}

\cvitem{\textbf{Past \quad Research}}{The Imperial HEP group has
  played a leading role in the design and delivery of the CMS Tracker
  detector, an 80M CHF (\textsterling 60M) project of unprecedented
  scale and complexity. I was posted to CERN on a Long Term Attachment
  for the period 2005-2011 to support the HEP group's activities. I
  have led several key areas of research related to detector
  performance over a period of a decade within the CMS Tracker
  Collaboration, which comprises 500 members. These activities have
  led to a number of publications~\cite{pub-tracking, pub-cosmics,
    pub-hlt, pub-tracker-cosmics, pub-tracker-perf, pub-magnet-test,
    pub-phys-tdr, pub-hips, pub-apvs, pub-feds} and conference
  proceedings~\cite{conf-chep2007-1, conf-chep2007-2, conf-chep2007-3,
    conf-chep2007-4, conf-vertex2007, conf-twepp2006-1,
    conf-twepp2006-2, conf-twepp2005, conf-twepp2004-1,
    conf-twepp2004-2, conf-twepp2004-3, conf-twepp2004-4,
    conf-twepp2004-5, conf-sirad2004, conf-twepp2002-1,
    conf-twepp2002-2}. I have held several important positions of
  responsibility within this community, such as the convenorship of a
  working group ($\sim$15 persons, $\sim$5 FTE), as a member of the
  editorial board, and as an on-call systems expert.}

\cvitem{\textbf{Current Research}}{I am the lead researcher and
  coordinator of a team currently comprising 16 researchers that
  searches for evidence of supersymmetry (SUSY) and Dark Matter with
  the CMS detector. I have more than seven years experience in this
  research area, having participated since the LHC start-up in
  2009. Under my coordination, the research team has produced several
  results and four publications~\cite{pub-alphat4, pub-alphat3,
    pub-alphat2, pub-alphat1} (with an additional two pending) of
  which I was the primary author and editor. These papers are of the
  highest impact by several measures: the team published the first
  SUSY result from the LHC in 2010, the publications contain several
  world-leading results, two of the papers are "famous" (i.e. $>$250
  citations), one was the subject of an APS {\em Physics highlights}
  article, and the collective citation count is in excess of 750. The
  impact of these results is given further context when considering
  the considerable challenges faced in order to deliver novel results
  at the energy frontier with new, large-scale detectors. I have also
  contributed as} \cvitem{}{a leading author to several other
  high-impact physics publications in recent years~\cite{pub-hinv,
    pub-sms, pub-ttbar, pub-jec}, in addition to the detector-related
  publications~\cite{pub-tracking, pub-cosmics, pub-hlt,
    pub-tracker-cosmics, pub-tracker-perf, pub-magnet-test,
    pub-phys-tdr, pub-hips, pub-apvs, pub-feds} mentioned above, with
  a collective citation count in excess of 2000. I am also proud to
  note, as a long-time collaboration member, my co-authorship of the
  Higgs discovery paper~\cite{pub-higgs} ($>$5000 citations).  }
%  I have also contributed as a leading author to several other
%  high-impact physics publications in recent years~\cite{pub-hinv,
%    pub-sms, pub-ttbar, pub-jec} (with a collective citation count in
%  excess of 600) including the Higgs discovery paper~\cite{pub-higgs}
%  ($>$5000 citations), in addition to the detector-related
%  publications~\cite{pub-tracking, pub-cosmics, pub-hlt,
%    pub-tracker-cosmics, pub-tracker-perf, pub-magnet-test,
%    pub-phys-tdr, pub-hips, pub-apvs, pub-feds} mentioned above
%  ($>$1400 citations).

\cvitem{\textbf{Teaching, Outreach}}{I teach and assess undergraduate
  students during laboratory and computing courses. I lecture to
  postgraduate students and I also supervise and assess the research
  of several postgraduate students, six of whom have already graduated
  from Imperial College. I also supervise the work of students from
  other UK and international institutes. I provide pastoral care for
  postdoctoral students while on Long Term Attachment at CERN. I
  regularly participate in outreach activities, e.g. I interact with
  undergraduate and school students participating in masterclasses or
  summer schools, or act as a mentor for short periods of work
  experience.}
  
\cvitem{\textbf{Responsibility, Management}}{I have been a research
  team contact person to the wider CMS Collabration since 2011. I have
  managed the team's activities with increasing autonomy since 2011
  and I supervise closely the work of both postdoctoral researchers
  and postgraduate students of Imperial College and other institutes
  in the UK and abroad. The team has constantly evolved over the
  years, with many staff and students participating from CERN and
  universities within the UK and US. I am responsible for defining the
  research strategy and managing the team such that it can execute its
  primary goal of delivering high-impact research in an extremely
  competitive environment. I am currently based at CERN on Long Term
  Attachment to coordinate UK activities in this area. I contribute
  material to the research proposals of the group to secure funding,
  and liaise closely with line managers and PIs to ensure that
  necessary financial support is foreseen to support the activities of
  the team.}

\cvitem{}{In October 2015, I accepted a new role within the management
  structure of the CMS Collaboration to steer and review the
  scientific output of a high-profile supersymmetry working group. The
  group comprises several analysis teams and a collective authorship
  in access of 100 researchers. Positions at this level of management
  are typically assigned to Fellows or faculty members. My position
  strengthens the group's standing and visibility within the
  Collaboration during an important phase of the experiment, when data
  collected during Run~2 will have important consequences for
  supersymmetric and Dark Matter models. The working group recently
  released five preliminary results~\cite{prelim-17, prelim-16,
    prelim-15, prelim-14, prelim-13} based on the data collected
  during Run~2, which are currently being prepared for publication.}

\cvitem{\textbf{Leadership, Reputation}}{I aim to lead by example and
  thrive on motivating teams to deliver high-impact science within a
  competitive environment. The work of the research team is highly
  reputed within the CMS Collaboration and in the wider field, and its
  results are frequently discussed at major international conferences
  and topical workshops. The work of the team is identified by CMS as
  a High Priority Analysis, which has led to targeted results for
  conferences~\cite{prelim-0}.}

\cvitem{}{My research contributions have been recognised by my peers
  within the CMS Collaboration, via important scientific management
  roles and through invitations to communicate scientific results on
  behalf of the Collaboration, including several recent high-profile
  presentations at major international
  conferences~\cite{conf-lathuile2016, conf-moriond2015,
    conf-lathuile2012, conf-ichep2010}. I organise working meetings
  within CMS on a weekly basis. I regularly participate in and present
  at topical workshops, and I have organised and chaired sessions. I
  actively participate in peer review, as a member of analysis review
  committees within CMS and as a journal referee.}

\cvitem{}{My experience and reputation in both the detector and
  physics communities means that I am well placed to steer and
  contribute to the future research plans of the HEP group, which has
  a strong presence in the detector upgrade projects of CMS, and
  ensure the group maximally exploits the physics potential of its
  commitments to CMS.}

%\cvitem{}{ In order to lead the activities of the Imperial HEP group
%  within these areas, I have been posted on two separate Long Term
%  Attachments at CERN, prior to and during Run I and II of the LHC.
%  My research in the areas of
%  detector R\&D and physics performance has led to a number of
%  publications~\cite{pub-tracking, pub-cosmics, pub-hlt,
%    pub-tracker-cosmics, pub-tracker-perf, pub-magnet-test,
%    pub-phys-tdr, pub-hips, pub-apvs, pub-feds}, with an aggregated
%  citation count in excess of 2000, and conference
%  proceedings~\cite{conf-chep2007-1, conf-chep2007-2, conf-chep2007-3,
%    conf-chep2007-4, conf-vertex2007, conf-twepp2006-1,
%    conf-twepp2006-2, conf-twepp2005, conf-twepp2004-1,
%    conf-twepp2004-2, conf-twepp2004-3, conf-twepp2004-4,
%    conf-twepp2004-5, conf-sirad2004, conf-twepp2002-1,
%    conf-twepp2002-2}. In recent years, I have been the lead
%  researcher and coordinator of a team that searches for evidence of
%  supersymmetry (SUSY) and Dark Matter. The team is identified by CMS
%  as a High Priority Analysis. The HEP group at Bristol plays a
%  continuous and crucial role within the team, which has produced
%  several high-impact publications that have yielded an aggregated
%  citation count in excess of 800~\cite{pub-alphat6, pub-alphat5,
%    pub-alphat4, pub-alphat3, pub-alphat2, pub-alphat1}.}
%
%\cvitem{}{ In terms of responsibility within CMS, I have held
%  positions such as the convenorship of a CMS Tracker working group
%  ($\sim$15 persons, $\sim$5 FTE), as a member of the CMS Tracker
%  editorial board, and as an on-call detector systems expert. I have
%  coordinated the activities of a research team with autonomy, and
%  acted as contact person to the wider collabration, since 2011. In
%  October 2015, I accepted a new role within the CMS management
%  structure to act as a convenor of the SUSY inclusive working group,
%  which comprises several teams and in excess of 100 researchers. In
%  2016, several publications~\cite{prelim-17, prelim-16, prelim-15,
%    prelim-14, prelim-13} and preliminary results~\cite{prelim-17,
%    prelim-16, prelim-15, prelim-14, prelim-13} have been released by
%  the group based on the analysis of data collected during Run~2.}
%
%\cvitem{}{I have considerable experience in teaching and assessing
%  undergraduate students during laboratory and computing courses. I
%  regularly lecture to postgraduate students. I supervise closely the
%  work of both postdoctoral researchers and postgraduate students of
%  Imperial and other institutes in the UK (e.g. Bristol) and abroad. I
%  regularly participate in outreach activities, e.g. I interact with
%  undergraduate and school students participating in masterclasses or
%  summer schools, or act as a mentor for short periods of work
%  experience.}
%
%\cvitem{}{I strive to lead by example and thrive on motivating
%  researchers and students to deliver high-quality, high-impact
%  science within a competitive environment. My contributions have been
%  recognised by my peers within the CMS Collaboration, via important
%  scientific management roles and through invitations to communicate
%  scientific results on behalf of the Collaboration, including several
%  recent high-profile presentations at major international
%  conferences~\cite{conf-lathuile2016, conf-moriond2015,
%    conf-lathuile2012, conf-ichep2010}. I organise working meetings
%  within CMS on a weekly basis. I regularly participate in and present
%  at topical workshops, and I have organised and chaired sessions. I
%  actively participate in peer review, as a member of analysis review
%  committees within CMS and as a journal referee.}
%
%\cvitem{}{My experience and reputation in both the detector and
%  physics communities means that I am well placed to steer and
%  contribute to the future research plans of the HEP group, which has
%  a strong presence in the detector upgrade projects of CMS, and
%  ensure the group maximally exploits the physics potential of its
%  commitments to CMS.}

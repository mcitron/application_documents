% Copyright 2006-2012 Xavier Danaux (xdanaux@gmail.com).
%
% This work may be distributed and/or modified under the
% conditions of the LaTeX Project Public License version 1.3c,
% available at http://www.latex-project.org/lppl/.

\documentclass[11pt,a4paper,sans]{moderncv}

\moderncvstyle{classic} %oldstyle
\moderncvcolor{green}
\nopagenumbers{}

% misc
\usepackage{amsmath} 
\usepackage{ragged2e}

% adjust the page margins
\usepackage[scale=0.8]{geometry}
\setlength{\hintscolumnwidth}{2.5cm}

% to show numerical labels in the bibliography
\makeatletter\renewcommand*{\bibliographyitemlabel}{\@biblabel{\arabic{enumiv}}}\makeatother
%\renewcommand{\refname}{Scientific Communication}

%\usepackage{url} % urls in bibtex references
\usepackage{cite} % allow ranges of citations 
%\usepackage{hyperref} % embedded citation links
\usepackage{multibib} % bibliography with mutiple entries
\newcites{pub,review,prelim,proc,conf,thesis,talks}{{},{},{},{},{},{},{}} % Empty bibliography headers

% personal data
\firstname{}%Dr Robert J.}
\familyname{}%Bainbridge}
% all optional below
\title{}%Research Associate, Imperial College London} 
\address{Blackett Laboratory}{Imperial College London}
\phone{+41~(0)22~7671728}
\mobile{+44~(0)7713~342985}
\email{robert.elwood@cern.ch}
\homepage{cern.ch/robert.elwood}
\extrainfo{\today}
%\photo[64pt][0.4pt]{picture}
%\quote{Some quote (optional)} 

%%%%%%%%%%%%%%%%%%%%%%%%%%%%%%%%%%%%%%%%%%%%%%%%%%%%%%%%%%%%%%%%%%%%%%%%%%%%%%%%

\begin{document}

\makecvtitle

\vspace{-4.4cm}
%\begin{flushright}
\textbf{Lecturer in Experimental Particle Physics} \\
\textbf{Ref: ACAD101986} 
%\end{flushright}
\vspace{2.3cm}

Dear James Annott, Joel Goldstein, Dave Newbold, and Annela Seddon, 

\vspace{0.2cm}

\hspace{0.0cm}I am a Research Associate of Imperial College London,
working in the field of experimental high energy physics (HEP). I
would greatly appreciate your consideration for the position of
Lecturer in Experimental Particle Physics at the Bristol School of
Physics. My curriculum vitae and supporting documentation identify my
contributions to the areas of teaching, leadership, and research. The
supporting documentation comprises details on my teaching engagements,
my past research activities and future plans, and a summary of my
scientific communications.

\vspace{0.1cm}

\hspace{0.0cm}I have been a collaboration member of the CMS experiment
since 2000, which operates at the Large Hadron Collider (LHC), CERN,
Geneva. The laboratory and its experiments provide world-class
research in the area of fundamental particles and forces. The CMS
experiment is a flagship activity at CERN, which is supported by a
4000-strong international collaboration. The ``big science''
environment within which I work gives context to my achievements and
roles within this structure.

\vspace{0.1cm}

\hspace{0.0cm}My research activities and interests can be summarised
as follows. I have a decade of experience in large-scale silicon-based
tracking detectors, their control and readout systems, and supporting
infrastructure, such as the associated software projects. I have
coordinated and contributed to key areas of the CMS Tracker project,
from the characterisation and performance testing of individual
electronic components, through to the integration, commissioning, and
operational phases of the project. I have also led activities related
to the analysis of CMS data for several years. I have accumulated
significant expertise in the extraction of physics observables from
the collision data, and in searches for evidence of new fundamental
physics processes at the LHC. As an example of research impact, the
publications resulting from my work on searches for supersymmetry, of
which I am the leading author, have accumulated a citation count in
excess of 800. I have been posted on two separate Long Term
Attachments at CERN, prior to and during Run~I and Run~II of the LHC,
to coordinate UK activities in these areas. I currently convene a
physics working group comprising several analysis teams.

\vspace{0.1cm}

\hspace{0.0cm}I have considerable experience in the teaching and
assessment of students. I have acted as a demonstrator to
undergraduate students during laboratory sessions and computing
courses. I lecture to postgraduate students each academic year. I
supervise closely the work of both postdoctoral researchers and
postgraduate students of Imperial and other institutes in the UK
(e.g. Bristol) and abroad. I regularly participate in outreach
activities, such as masterclasses for school students, the CERN summer
school programme for undergraduates, and as a mentor for school
students wishing to gain work experience in a research environment.

\vspace{0.1cm}

\hspace{0.0cm}Finally, I would like to highlight my close connection
to the Bristol HEP group. I have worked in collaboration with Bristol
faculty members and postdoctoral researchers for several years. I have
also supervised the work of Bristol postgraduate students while based
at CERN, on behalf of their supervisors. I have always found the
interactions with my Bristol-based experimental collaborators to be
extremely productive.

\vspace{0.1cm}

\hspace{0.0cm}I believe that my experience in the areas of detector
systems and physics analysis can strengthen and complement the
research interests of the Bristol HEP group.  I would welcome the
opportunity to join the excellent Bristol physics department as a
lecturer.

\vspace{0.2cm}
Kind regards, \\
Dr. Robert Bainbridge

\end{document}

%\section{Covering letter}

%\vspace{0.2cm}
\begin{flushright}
\textbf{Lecturer in Experimental Particle Physics (Ref: ACAD101986).} \\
\end{flushright}
%\vspace{0.2cm}

Dear James Annott, Joel Goldstein, Dave Newbold, and Annela Seddon, \\

I am a Research Associate of Imperial College London, working in the
field of experimental high energy physics (HEP). I would greatly
appreciate your consideration for the Lecturer in Experimental
Particle Physics position that is currently open at the Bristol
physics department. My curriculum vitae and supporting documentation
identify my contributions to the areas of teaching, leadership, and
research. The supporting documentation comprises details on my
teaching engagements, my past research activities and
future plans, and a summary of my scientific communications. \\

I have been a collaboration member of the CMS experiment since 2000,
which operates at the Large Hadron Collider (LHC), CERN, Geneva. The
laboratory and its experiments provide world-class research in the
area of fundamental particles and forces. The CMS experiment is a
flagship activity at CERN, which is supported by a 4000-strong
international collaboration. The ``big science'' environment within
which I work gives context to my achievements and roles within
this structure. \\

My research activities and interests can be summarised as follows. I
have a decade of experience in large-scale silicon-based tracking
detectors, their control and readout systems, and supporting
infrastructure, such as the associated software projects. I have
coordinated and contributed to key areas of the CMS Tracker project,
during the early characterisation and performance testing of
electronic components, and throughout the integration, commissioning,
and operational phases. I have also led activities related to the
analysis of CMS data for several years. I have accumulated significant
expertise in the extraction of physics observables from the collision
data, and in searches for evidence of new fundamental physics
processes at the LHC. The publications resulting from my most recent
research activities concerning searches for new-physics processes, of
which I am a leading author, have alone accumulated a citation count
in excess of 800. I have been posted on two separate Long Term
Attachments at CERN, prior to and during Run~I and Run~II of the LHC,
to coordinate UK activities in these areas. I currently convene the
``SUSY Inclusive'' physics working group, which is one of the largest
within CMS, comprising several analysis teams. \\

I have considerable experience in the teaching and assessment of
students. I have acted as a demonstrator to undergraduate students
during laboratory sessions and computing courses. I lecture to
postgraduate students each academic year. I supervise closely the work
of both postdoctoral researchers and postgraduate students of Imperial
and other institutes in the UK (e.g. Bristol) and abroad. I regularly
participate in outreach activities, such as masterclasses for school
students, the CERN summer school programme for undergraduates, and as
a mentor for school students wishing to gain work experience in a
research environment. \\ 

Finally, I would like to highlight my close connection to the Bristol
HEP group. I have worked in very close collaboration with Bristol
faculty members and postdoctoral researchers for several years. I have
also supervised the work of Bristol postgraduate students while based
at CERN, on behalf of their supervisors. I have always found the
interactions with my Bristol-based experimental collaborators to be
very engaging and productive. \\ 

I believe that my experience in the areas of detector systems and
physics analysis can strengthen and complement the research interests
of the Bristol HEP group.  I would welcome the opportunity to join the
excellent Bristol physics department as a lecturer. \\

Kind regards, \\
Dr. Robert Bainbridge


%\cvitem{}{I strive to lead by example and thrive on motivating
%  researchers and students to deliver high-quality, high-impact
%  science within a competitive environment. 




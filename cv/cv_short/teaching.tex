\section{Teaching} %and outreach}

%\vspace{0.1cm}
%\subsection{Outreach activities with undergraduates and schools}
%
%\cvitem{}{I have on several occasions committed significant time to
%  act as a mentor to school and undergraduate students that wish to
%  gain experience in an academic research environment. For example, I
%  have mentored undergraduate students over several weeks that have
%  participated in the CERN Summer School programme, often from
%  universities of CERN non-member states, such as Brazil and Egypt. I
%  have also mentored 6$^{\text{th}}$ form school students for short
%  periods at CERN who wish to gain work experience. I have hosted
%  school students as part of the annual Particle Physics Masterclass
%  within the HEP group at Imperial. I also provide material that
%  describes the research activities for the website of the HEP group. 
%  I have also led tours of the CMS experimental areas and participate
%  in events that host a range of visitors to CERN. }

\vspace{0.1cm}
\subsection{Undergraduate students}

\cvitem{}{When based at Imperial, I teach experimental laboratory
  techniques and computing skills to undergraduate students. I provide
  feedback to the students as a demonstrator during these courses, and
  assess their coursework at the end of the term. I have also provided
  feedback to the Head of Experiment to improve the scripts that
  support the courses.}

\cvitem{2012}{Demonstration and assessment of a 2$^{\text{nd}}$-year
  C++ and python computing course.}

\cvitem{2013}{Demonstration and assessment of laboratory courses for
  3$^{\text{rd}}$ year undergraduate students.}

\vspace{0.1cm}
\subsection{Postgraduate students}
\vspace{0.1cm}

\cvitem{}{\textbf{Pastoral care}}

\cvitem{}{The postgraduate students are typically required to spend a
  significant fraction of their postdoctoral degree on a Long Term
  Attachment at CERN, with a duration as long as two years, to gain
  experience working in the environment of an international
  laboratory. This time serves as an important ingredient to their
  training. I aim to minimise the impact of the move abroad while the
  students undertake their studies, by providing pastoral care to all
  students under my supervision while they are based at CERN. I have
  an open door policy and I am always available to discuss any issues
  that they are facing.}

\cvitem{}{\textbf{Lectures and seminars}}

\cvitem{}{I provide short lecture series on topics in my area of
  expertise, such as supersymmetry. The subject is complex and
  constantly evolving due to the steady stream of new experimental
  results coming from collider- and non-collider-based 
  experiments. Hence the challenge is to present the subject matter in
  an accessible way, evolve the content on a yearly basis in order to
  be current, and provide context to this important area of
  physics. The lecture courses are listed below. I have also delivered
  seminars to new postgraduate students (as part of a series) with the
  aim of helping the students choose their research project. }

\cvitem{2013}{QCD and jet physics at the LHC.}

\cvitem{2014}{Experimental and phenomenological aspects of SUSY
  searches.}

\cvitem{2015}{Theoretical and experimental aspects of supersymmetry.}

\cvitem{}{\textbf{Training and mentoring}}

\cvitem{}{My aim is to develop the necessary technical, interpersonal,
  and communication skills in an individual that allow him/her to
  undertake projects in an autonomous manner. Another particularly
  important aspect of postgraduate training is to encourage
  independent, creative thinking and the development of novel ideas
  and solutions within the scope of the research project. I am
  responsible for organising formal weekly meetings that allow the
  students to report on their progress to the rest of the research
  team, including faculty members. These meetings allow me to monitor
  and organise the work of the individual and also the group as a
  whole. However, I also place emphasis on frequent informal
  discussions that allow me to provide feedback on very specific
  issues. }  
%or, equally, interact in the spirit of a "brainstorming" session to
%allow ideas to be explored fully. }

\cvitem{}{\textbf{Supervision and assessment}}

\cvitem{}{I have closely supervised the research of several
  postgraduate students from Imperial and other UK and international
  institutes over several years. My duties include the daily
  supervision of their work, informing the supervisors (faculty) of
  their progress and potential problems, the participation in the
  formal assessment of their progress (e.g. transfer vivas), and
  planning future work with the student. }

\cvitem{}{Several students, listed below, have graduated based on
  research peformed under my direct supervision. The studies of a
  further five students are currently in progress. These studies span
  a broad range of subject matter, from searches for new physics
  phenomena to detector performance. }

\vspace{0.2cm}

%\cvitem{}{\textbf{Graduated} (with completion dates)}

\cvitem{2008}{M.~Wingham, Imperial College London, UK, "Commissioning
  of the CMS tracker and preparing for early physics at the LHC".}

\cvitem{2009}{N.~Cripps, Imperial College London, UK, studies related
  to the CMS Tracker detector.}

\cvitem{2011}{P.~Kalavase, University Of California, USA, studies
  related to the CMS Tracker detector.}

\cvitem{2012}{T.~Whyntie, Imperial College London, UK, "Constraining
  the supersymmetric parameter space with early data from the Compact
  Muon Solenoid experiment".}

\cvitem{2012}{Z.~Hatherell, Imperial College London, UK, "Searching
  for SUSY in events with Jets and Missing Transverse Energy using
  $\alpha_\text{T}$ with the CMS Detector at the LHC".}

\cvitem{2012}{R.~Nandi, Imperial College London, UK, "A Search for
  Supersymmetry in Events with Photons and Jets from Proton-Proton
  Collisions at $\sqrt{s} = 7\,$TeV with the CMS Detector".}

\cvitem{2013}{S.~Rogerson, Imperial College London, UK, "A search for
  supersymmetry using the $\alpha_\text{T}$ variable with the CMS
  detector and the impact of experimental searches for supersymmetry
  on supersymmetric parameter space".}

\cvitem{2013}{B.~Mathias, Imperial College London, UK, "Search for
  supersymmetry in pp collisions with all-hadronic final states using
  the $\alpha_\text{T}$ variable with the CMS detector at the LHC".}

\cvitem{2014}{D.~Burton, Imperial College London, UK, "Searches for
  supersymmetric signatures in hadronic final states with the
  $\alpha_\text{T}$ variable".}

\cvitem{2015}{Y.~Eshaq, University of Rochester, USA, "Search for New
  Physics in All-hadronic Events with $\alpha_\text{T}$ in 8 TeV Data
  with CMS"}

\cvitem{2015}{C.~Lucas, University of Bristol, UK, "Search for
  supersymmetry in events with jets and missing energy at the LHC"}

\cvitem{2016}{M.~Baber, Imperial College London, UK, "Search for
  supersymmetry in the first $\sqrt{s} = 13$~TeV pp-collisions using
  the $\alpha_\text{T}$ variable with the CMS detector", submitted
  Oct. 2016.} 

\cvitem{}{{\em In progress, with expected completion dates:}}

\cvitem{2017}{A.~Elwood, Imperial College London, UK, 
  studies related to new-physics searches}

\cvitem{2017}{M.~Citron, Imperial College London, UK, 
  studies related to new-physics searches}

\cvitem{2017}{L.~Lo, University of Rochester, USA, 
  studies related to new-physics searches}

\cvitem{2018}{C.~Laner, Imperial College London, UK, 
  studies related to new-physics searches}


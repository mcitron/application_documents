\section{Coordination and project management}

\cvitem{2005--2011}{\textbf{Long Term Attachment at CERN}: I was
  posted at the laboratory to lead the activities of Imperial College
  concerning the commissioning, calibration, and operation of the CMS
  Tracker. The final two years focused on understanding the physics
  performance of the CMS detector and early searches for supersymmetry
  during LHC Run~1. }

\cvitem{2008--2009}{\textbf{Convenor of Data Quality Monitoring (DQM)
    working group}: I was a lead proponent and the first convenor of a
  new working group within the CMS Tracker project, tasked with
  delivering real-time monitoring tools for use during both the
  commissioning and operational phases of the experiment. I rapidly
  established a team of fifteen researchers (providing the equivalent
  of $\sim$5 FTE) that developed a range of applications to provide
  live information on the Tracker performance, used throughout LHC
  Run~1.}

\cvitem{2008--2010}{\textbf{CMS Tracker Editorial Board}: I was a
  member of the editorial board that was tasked with reviewing
  publications from the CMS Tracker community prior to journal
  submission.}

\cvitem{2010--2011}{\textbf{Shift Leader of operations for the CMS
    Tracker}: I provided on-call expert support for the shift
  operators during beam time and I was responsible for ensuring stable
  operations, minimising downtime, and monitoring performance during
  data-taking periods.}

\cvitem{2010--present}{\textbf{Peer review and collaboration}: I have
  participated in and presented at several workshops that bring
  together experimental and theoretical physicists to discuss research
  areas concerning Beyond Standard Model, Higgs, and Dark Matter
  physics. I have organised and chaired sessions of national UK CMS
  meetings. I have peer reviewed submissions to the journal Physics
  Letters B. Additionally, I have been a member of several Analysis
  Review Committees within CMS, which review the work of analysis
  teams and their publications prior to journal submission, in the
  areas of top quark, Exotic, Higgs, and supersymmetry physics.}
%, listed below.}
%~\cite{pub-hinv, pub-ttbar}.

%\cvitem{}{TOP-10-007: {\em Search for new physics in the $t\bar{t}$
%    invariant mass spectrum at $\sqrt{s} = 7\,$TeV}}
%
%\cvitem{}{EXO-11-055: {\em Search for Z'$\,\rightarrow\, t\bar{t}$
%    in high-mass semileptonic channel}}
%
%\cvitem{}{B2G-12-005: {\em Search for $t\bar{t}$ resonances in
%    boosted all-hadronic final state}}
%
%\cvitem{}{B2G-13-001: {\em Search for $t\bar{t}$ resonances in
%    semileptonic final states at $\sqrt{s} = 8\,$TeV}}
%
%\cvitem{}{HIG-13-013: {\em Search for invisible Higgs decays in the
%    VBF channel}}
%
%\cvitem{}{SUS-13-013: {\em Search for SUSY in the same-sign
%    dilepton final state}}
%
%\cvitem{}{SUS-16-005: {\em Search for stops in the hadronic final
%    state with the CMS top tagger}}
%
%\cvitem{}{SUS-16-006: {\em Search for top squark pairs in the
%    hadronic final state at 13 TeV}}
%
%\cvitem{}{SUS-16-009: {\em Searches for pair production of
%    third-generation squarks at 13 TeV}}

\cvitem{2011--present}{\textbf{Coordination of a CMS analysis team}: I
  have coordinated for several years the activities of a high-profile
  research team tasked with delivering a CMS High Priority Analysis,
  which currently comprises sixteen researchers, including two
  research fellows, three postdoctoral researchers, and six
  postgraduate students (and five faculty members) from several
  institutes across Europe and the US.}

\cvitem{2014--present}{\textbf{Long Term Attachment at CERN (second
    posting)}: I was recently posted to CERN for a second LTA to
  coordinate activities related to new physics searches during Run~2
  of the LHC. Additionally, I support the UK obligations to operation
  of the CMS detector during data-taking periods, as an experienced
  CMS Trigger shifter.  }

\cvitem{2015--present}{\textbf{Convenor of CMS SUSY Inclusive working
    group}: Since October 2015, I operate within the management
  structure of the CMS Collaboration as a convenor of one of the
  largest analysis working groups, which currently comprises seven
  analysis teams 
%  ($\alpha_\text{T}$, \texttt{MHT}, \texttt{MT2}, Razor, 1L
%  \texttt{MJ}, 1L $\Delta\phi$, displaced vertices, RPV SUSY) 
  and in excess of 100 authors that are searching for
  supersymmetry. My role is to review and steer the scientific output
  of the group. These activities have led to several
  publications~\cite{Khachatryan:2016xdt, Khachatryan:2016epu,
    Khachatryan:2016uwr, Khachatryan:2016xvy, Khachatryan:2016kdk} and
  preliminary results based on the first data set collected during
  Run~2.}

\cvitem{Ongoing}{\textbf{Grants}: I contribute to research proposals
  in order to secure funding to support the hardware and physics
  research activities within the Imperial HEP group. I also liaise
  closely with line managers to ensure that necessary financial
  support is foreseen to support the activities of postdoctoral
  researchers and postgraduate students. In order to demonstrate the
  value and impact of our research, I also have hosted visits from the
  Provost and CFO of Imperial College, as well as representatives from
  Members of Parliament, the UK Parliament Science and Technology
  Committee, and the Science and Technologies Facilities Council.}

\section{Research plan}

\cvitem{}{While the standard model (SM) of particle physics is an
  extremely successful theory, several crucial questions remain
  unanswered. Two of the most important questions that the field must
  tackle, that are of personal interest, concern the gauge hierarchy
  problem and the nature of dark matter, both of which can be resolved
  through a supersymmetric extension of the standard model. Further,
  while significant inroads can be made in our understanding of these
  two issues during Run~2 of the LHC, is it imperative to plan for the
  future, as is currently being done through the CMS upgrade projects
  aimed at the High Luminosity LHC. In the following, I describe my
  research interests. %plans. %and hopes. %for the future with Bristol. %for the next ten years.
%  Finally, while one cannot predict what the future may hold, it is
%  important to have the ability to react to new discoveries, at the
%  LHC or elsewhere in the field. 
}

\vspace{0.1cm}
\subsection{A natural solution to the gauge hierarchy problem and the
  nature of dark matter?}

\cvitem{}{The discovery of a new type of particle, a scalar boson, in
  2012 is the current crown jewel in the LHC physics program. All
  measurements performed so far on this state are consistent with the
  expected properties of a SM Higgs boson. However, it is known that
  the mass of the scalar Higgs is unstable to short-distance quantum
  corrections from loop processes, divergent according to a power-law
  dependence on scale, which is incompatible with its observed
  weak-scale mass of $\sim$125$\,$GeV. Without modification, the SM is
  unable to accommodate these quantum corrections without an enormous
  level of fine-tuning of model parameters. }

\cvitem{}{The supersymmetric extension of the standard model is able
  to alleviate this ``hierarchy problem'' through the conjecture that
  the superpartners (sparticles) to SM particles can compensate for
  the radiative corrections to the Higgs mass via loop processes
  involving sparticles. A ``natural'' theory requiring only a moderate
  level of fine tuning can be realised with a minimal sparticle
  spectrum at the TeV scale, including the superpartners to the top
  and bottom quarks, as well as the gluino (albeit with a weaker
  constraint on its mass). To date, no experimental evidence of new
  particle states has been found. With the discovery of a Higgs boson,
  the challenge of understanding whether weak-scale physics is natural
  or otherwise is one of the most important issues to be resolved
  within the field of particle physics. }

\cvitem{}{Another immediately pressing issue of interest to me is
  understanding the nature of dark matter, the existence of which is
  supported by a range of cosmological and astrophysical
  observations. 
%  , such as galactic rotation curves, gravitational lensing,
%  anisotropies in the cosmic microwave background. 
  With minimal assumptions, SUSY can provide a viable candidate for
  cold non-baryonic dark matter in the form of a stable, massive,
  weakly interacting particle (typically considered to be the
  neutralino or gravitino). }

\cvitem{}{The Bristol HEP group has long been associated with the
  analysis team within which I currently operate. The search has a
  long history and it is inclusive in its design, to give broad
  sensitivity across the model parameter space. I propose to
  (immediately) adapt the existing search to better target difficult
  regions of the SUSY parameter space. Specifically, three model
  classes are of interest to me: the heaviest sparticles, such as
  gluinos; the ``stealth stop'' region; and models that involve the
  direct pair production of light chargino and/or neutralino states
  with masses that are nearly degenerate. The first scenario benefits
  most from the new energier frontier of Run~2, while the latter two
  scenarios are poorly covered by the experimental searches yet are
  fully consistent with a ``natural'' solution to the gauge hierarchy
  problem. These scenarios, outlined below, could be interesting
  thesis topics for postgraduate students that are based on the data
  accumulated during Run~2.}

\cvitem{}{\textbf{Heavy gluinos}: I propose to revisit a method
  utilitised in the first two SUSY publications from the CMS UK
  community~\cite{pub-alphat1, pub-alphat2}. The approach employs the
  use of progressively tighter requirements on experimental acceptance
  (e.g. in terms of transverse momentum) for reconstructed physics
  objects (e.g. jets), such that the event kinematics remain
  ``invariant'' with respect to any variable that provides an estimate
  of the energy scale being probed. The assumption is that, above the
  characteristic scale of the dominant standard model backgrounds, say
  $\sim$500$\,$GeV, low-scale events from standard model processes,
  which can be measured with a high statistical precision, can be used
  to predict SM populations at higher scales, and potentially provide
  the necessary discriminating power to identify new-physics
  processes, such as the production of heavy gluinos. Following
  Refs.~\cite{pub-alphat1, pub-alphat2}, the approach was investigated
  in collaboration with J. Ellis and B. Gripaios, but it was not
  concluded at the time. However, the approach could be again of
  relevance given that we are operating at a new energy frontier, and
  probing scales well beyond those typical of SM processes. }

\cvitem{}{\textbf{``Stealth stops''}: Experimental coverage is
  relatively weak for regions of the SUSY parameter space when the
  masses of top squark and the LSP are both light and differ by a
  value equal (or close) to the mass of the top quark. For this
  scenario, it is difficult to exploit kinematical differences in
  order to discriminate the decays of pair-produced top squarks and
  the dominant background process, top-antitop production. I would aim
  to take advantage of the experience from UK SUSY activities,
  combined with the mature studies of top-antitop production from the
  Bristol HEP group, in order to exploit ways to improve the
  sensitivity to this difficult region of phase space. This class of
  model is currently a focus of attention within the experimental and
  theoretical communities, and will benefit from the large integrated
  luminosities expected from the LHC in the near future.  I recently
  initiated the creation of a dedicated task force within CMS that
  aims to understand in more detail the strengths and weaknesses of
  the CMS SUSY programme in this area. }

\cvitem{}{\textbf{Exploiting ISR}: The typical approach to searching
  for SUSY models involving near-degenerate mass spectra is to employ
  a ``monojet search'', which relies on initial state radiation (ISR)
  for experimental acceptance. While speculative, I am hoping to
  improve on the existing method, by investigating the potential to
  utilise ISR to discriminate between the production of new-physics
  and SM particles. The utility in this approach is based on the fact
  that the kinematics of the ISR system depend only on the mass of the
  particles involved in the production process, and not any of the
  particles produced in the decay. This differs from the usual
  approach of constructing variables that rely on kinematical
  quantities that are sensitive to all masses involved in the
  production and decay, the combinatorics of final state objects, and
  any weakly interacting particles that not are observed. This
  approach is potentially of interest to SUSY models involving the
  direct production of charginos and/or neutralinos, and possibly to
  constrain the branching fraction of Higgs boson to invisible final
  states.  }

%\vspace{0.1cm}
\clearpage
\subsection{CMS upgrades for the High Luminosity LHC, and exploiting the physics potential}

\cvitem{}{The search for new physics and, in the event of its
  discovery, the understanding of its nature will require a large
  amount of data to be recorded. The plans of the LHC machine group
  have evolved to provide high instantaneous and integated
  luminosities, well beyond the design specification for the CMS
  experiment. As a result, most of the key systems within CMS will be
  replaced or upgraded to maintain the physics performance of the
  detector. }

\cvitem{}{The key goal of the CMS ``Phase-II'' upgrade project for
  HL-LHC is to maintain the acceptance to electroweak physics
  processes. This requires the development of readout systems with
  higher granularity, to maintain manageable occupancies and provide
  improved resolutions, and with higher tolerances against the
  accumulated radiation doses than the current systems, to minimise
  any degradation in performance. These issues are particularly accute
  for the tracker systems. }

\cvitem{}{Further, a crucial development in the design of the CMS
  detector for HL-LHC is the ability to generate and utilise
  information from the tracking systems within the Level-1 trigger
  algorithms. Only by combining information from the tracking systems
  and the calorimeter and muon systems at the Level-1 stage of the
  trigger can rates be reduced to a level that is technically feasible
  on the timescales envisaged. }

\cvitem{}{The use of the tracking information within the trigger
  system provides a powerful handle to improve, for example, the
  identification of low-$p_{\text{T}}$ isolated charged leptons, the
  primary and other secondary vertices, and charged tracks resulting
  from additional ``pileup'' interactions, as well as improving the
  resolution of jet energies and missing transverse energy. These
  features are crucial for maintaining the experimental acceptance to
  electroweak physics processes, such as vector and scalar boson
  production, but also to new physics processes, such as the
  production of supersymmetric particles.}

\cvitem{}{Any potential discovery of new physics would rely heavily on
  the use of tracking information, both at the trigger level and as
  input to a global event reconstruction algorithm. This is especially
  relevant for SUSY, given the potential for a rich variety of event
  topologies and final states, for which a broad experimental
  acceptance is mandatory.}

\cvitem{}{\textbf{Detector integration and commissioning}: I am
  extremely keen to renew my involvement with detector systems, and
  the CMS Tracker upgrade project is an exciting one, with novel
  features, and it is key to the success of CMS at the HL-LHC. I have
  considerable experience across a broad range of areas, such as
  characterisation and performance testing of both front-and back-end
  electronics, integration of both small-scale prototype systems and
  large-scale systems, data acquisition, databasing, event
  reconstruction, etc. I would hope to provide leadership in the
  Tracker upgrade project during its construction and commissioning
  phases. }

\cvitem{}{\textbf{Physics exploitation via the track-trigger
    primitives}: I would also hope to maximally exploit the physics
  potential of the tracker detector once operational. In this regard,
  I would favour a focus on the novel use of tracking information in
  trigger. I would hope to develop trigger algorithms that utilise
  tracking information to have maximum impact in terms of physics
  reach. From the point of view of SUSY, an optimisation of the
  performance of triggers utilising the missing transverse energy
  would be high priority. Other opportunities involve the
  identification of soft leptons, to target compressed mass spectra,
  or the potential exploitation of secondary vertex identification at
  Level-1, to target third-generation squark signatures. Other regions
  of the parameter space could also be targeted, such as ``split
  SUSY'' models that produce R-hadrons with signatures involving
  displaced vertices. }

%\cvitem{}{Bristol's involvement in these areas is crucial if the group
%  wishes to maximally exploit the physics potential of the
%  HL-LHC. Bristol is key to this deliverable, with an established
%  track record in hardware activities within CMS.}

\vspace{0.1cm}
\subsection{Beyond the standard model and the LHC}

\cvitem{}{Neutrino physics provides a rich area of opportunity in the
  field of high energy physics. I am aware of Bristol's recent
  commitment to the new Deep Underground Neutrino Experiment
  (DUNE). The experiment has a wide range of goals, but a primary
  objective is to study neutrino oscillations with precision, such
  that a measure of the CP-violating phase parameter could potentially
  be made.
%
%  If non-zero, CP violation in the neutrino sector of the standard
%  model could help explain the matter-antimatter asymmetry that
%  defines our universe today.
%
%  An area that is of interest to me is the possibility to test the
%  anomolous result obtained from LSND, which is in tension with the
%  standard model expectation of only three neutrino flavors, when
%  considering the constraints from other solar and atmospheric
%  neutrino oscillation experiments. The LNSD result can be interpreted
%  as a sterile neutrino at the $\Delta m^2 \sim \text{eV}^2$
%  scale. DUNE is able to probe with precision this region of the
%  parameter space and confirm or refute this anomolous result
%  concerning $\nu_\mu \rightarrow \nu_\text{e}$ oscillations. 
%
%\cvitem{}{ 
  I have some experience in the field of neutrino oscillations, as I
  undertook a nine-month project to develop an analysis to search for
  evidence of $\nu_{\mu} \rightarrow \nu_{\tau}$ oscillations with the
  \texttt{NOMAD} experiment. This was done under the supervision of
  Prof. M. Baldo-Ceolin at the University of Padova as an
  \texttt{ERASMUS} student in 1999. Of greater importance is the
  experience I have gained with the CMS Tracker project. I have
  undertaken a range of roles during the various phases of the
  project, from the characterisation of early prototype components, to
  the delivery of a complete, operational detector, and all stages
  inbetween. Given the early stage of the DUNE project, I appreciate
  the potential for possible leadership opportunities within the UK
  and beyond. My experience would be useful to a new international
  project such as DUNE. I note that the timelines of DUNE and the
  HL-LHC are similar, and I would be open to discussing research
  priorities.}

%
%\cvitem{}{My experience and reputation in both the detector and
%  physics communities means that I am well placed to steer and
%  contribute to the future research plans of the HEP group, which has
%  a strong presence in the detector upgrade projects of CMS, and
%  ensure the group maximally exploits the physics potential of its
%  commitments to CMS.}





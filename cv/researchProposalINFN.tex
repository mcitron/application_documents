% Copyright 2006-2012 Xavier Danaux (xdanaux@gmail.com).
%
% This work may be distributed and/or modified under the
% conditions of the LaTeX Project Public License version 1.3c,
% available at http://www.latex-project.org/lppl/.

\documentclass[a4paper]{moderncv}

\moderncvstyle{classic} %oldstyle
\moderncvcolor{blue}
\nopagenumbers{}

% misc
\usepackage{amsmath} 
\usepackage{ragged2e}

% adjust the page margins
\usepackage[scale=0.8]{geometry}
\setlength{\hintscolumnwidth}{2.5cm}

% to show numerical labels in the bibliography
\makeatletter\renewcommand*{\bibliographyitemlabel}{\@biblabel{\arabic{enumiv}}}\makeatother
%\renewcommand{\refname}{Scientific Communication}

%\usepackage{url} % urls in bibtex references
\usepackage{cite} % allow ranges of citations 
%\usepackage{hyperref} % embedded citation links
\usepackage{multibib} % bibliography with mutiple entries
\newcites{pub,review,prelim,proc,conf,thesis,talks}{{},{},{},{},{},{},{}} % Empty bibliography headers

% personal data
\firstname{Adam}
\familyname{Elwood}
% all optional below
\title{}%Research Associate, Imperial College London} 
\address{Blackett Laboratory}{Imperial College London}
\mobile{+44~(0)7815~520631}
\email{adam.elwood09@imperial.ac.uk}
\extrainfo{\today}
%\photo[64pt][0.4pt]{picture}
%\quote{Some quote (optional)} 

%%%%%%%%%%%%%%%%%%%%%%%%%%%%%%%%%%%%%%%%%%%%%%%%%%%%%%%%%%%%%%%%%%%%%%%%%%%%%%%%

\begin{document}

\makecvtitle

\vspace{-4.4cm}
%\begin{flushright}
\textbf{``INFN POST-DOCTORAL FELLOWSHIPS IN EXPERIMENTAL PHYSICS''}\\
\textbf{YEAR 2017/2018} \\
\textbf{Statement of research interests} 
%\end{flushright}
\vspace{2.5cm}

With the discovery of a new particle resembling a
Higgs boson during Run~1 of the LHC, a major new priorioty for the
particle physics community is to study all of its properties. The self
couplings of the Higgs boson are precisely predicted in the Standard
Model (SM), measuring this property is therefore one of the most
important tests of the SM and crucial for verifying that the
Brout-Englert-Higgs mechanism is responsible for electroweak symmetry
breaking. Due to the destructive interference between the main
production mechanisms for pairs of Higgs bosons, this decay is
particularly sensitive to physics beyond the SM (BSM). Examples of BSM
physics effects that would contribute are anomalous SM quark
couplings, new BSM particles contributing to the virtual loop at
production and resonant enhancements of the decay. Additionally, some popular BSM
theories, such as supersymmetry (SUSY), predict the existence of
additional heavy Higgs bosons. Depending on the characteristics of the
model, a major decay mode of these heavy Higgs' can be a pair of SM Higgs
bosons. Searching for a high mass resonance in the di-Higgs final
state is therefore particularly crucial to probe these kinds of models.
\\\\
The CMS detector at the LHC is very well suited for studying the
properties of Higgs pair production. The combination of a tracker,
calorimeters and muon chambers built around a superconducting solenoid allow for
accurate determination of the decay products of Higgs bosons. Two of
the most sensitive di-Higgs final states are $\tau\tau bb$ and
$\tau\tau\gamma\gamma$. To maximise the acceptance to these final
states, it is necessary to utilise advanced reconstruction algorithms
such as techniques that allow for the discrimination of signal from
background by studying jet substructure. It is also important to
employ a robust trigger strategy to ensure maximum signal efficiency
within the constraints of significant rate from hadronic backgrounds.
\\\\
I have gained experience working on CMS throughout the course of my
PhD, masters thesis and undergraduate summer projects. My general area
of interest is in searching for indications of BSM physics. Throughout
my PhD studies I have gained significant experience developing jet
based Level-1 trigger algorithms with a focus on pileup subtraction,
and determining High Level Trigger strategies for a hadronic analysis.
Having contributed significantly to an all-hadronic SUSY search I have
particular knowledge and an interest in the reconstruction tools and
SM backgrounds that are relevant in jet dominated final states. Along
with running this analysis on the latest Run~2 LHC data in a
competitive time frame, I have also worked on developing strategies to
improve the background predictions for the analysis and maximise
signal sensitivity. Working on analyses that target Higgs pair
production offers a very good opportunity for me to expand my areas of
expertise, while allowing me the chance to use the experience I have
already gained to contribute significantly to the development to the
analyses as they stand.
\\\\
As the LHC starts to deliver a significant dataset of 13 TeV
proton-proton collisions, we are in a very relevant time for probing
new BSM phenomena with the di-Higgs final state. As the base level
physics object reconstruction in Run~2 starts to mature, there is
significant potential for developing new and more sophisticated
reconstruction techniques for the determination of jet substructure
and other complex hadronic variables. Exploitation of all possible
information for future reconstruction techniques will become even more relevant with
the challenge presented by the projected increase in pileup. The
optimisation of an effective trigger strategy is also something that
should occur as early as possible to ensure a maximum signal
sensitivity throughout the collection of data by CMS. As the LHC
program moves forward towards the High Luminosity LHC, di-Higgs
production remains relevant with the potential, in the absence of BSM
physics, to start probing the SM branching ratio.
\\\\ 
Given the award of this fellowship I would aim to start work
on the di-Higgs analyses from mid 2017, focusing on the $\tau\tau bb$
final state. The initial aims would be to further the development of
the trigger strategy and prepare to run the analysis on the latest
data sets. There is then significant potential to help
develop and apply new hadronic reconstruction techniques, particularly
when dealing with boosted and merged jets. The improvement of
background estimation techniques is additionally an area that can be
developed and one in which I have experience. All of these
developments will offer further improvements to the sensitivity of the
analyses. With the potential for $O(100) fb^{-1}$ data within a two
year time scale, the development and execution of the analysis is now
particularly relevant.  Additionally, with an eye toward the future of
the field, I would welcome the potential to get involved with hardware
developments to the CMS detector.  Maintaining or even improving the
detector performance in the future being of utmost importance for
di-Higgs analyses.

\end{document}

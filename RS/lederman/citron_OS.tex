\documentclass[12pt]{article}
\usepackage{amsmath}
\usepackage{cancel}
\usepackage{xspace}
\usepackage{amssymb}
\usepackage{amsthm}
\usepackage{amscd}
\usepackage{amsfonts}
\usepackage{graphicx}%
\usepackage{fancyhdr}

%\usepackage[a4paper, total={7in, 8in}]{geometry}
\usepackage[bottom=0.4in,left=1.0in,right=1.0in]{geometry}
\usepackage{fancyhdr}

\theoremstyle{plain} \numberwithin{equation}{section}
\newtheorem{theorem}{Theorem}[section]
\newtheorem{corollary}[theorem]{Corollary}
\newtheorem{conjecture}{Conjecture}
\newtheorem{lemma}[theorem]{Lemma}
\newtheorem{proposition}[theorem]{Proposition}
\theoremstyle{definition}
\newtheorem{definition}[theorem]{Definition}
\newtheorem{finalremark}[theorem]{Final Remark}
\newtheorem{remark}[theorem]{Remark}
\newtheorem{example}[theorem]{Example}
\newtheorem{question}{Question} \topmargin-2cm

% \textwidth7in
%
% \setlength{\topmargin}{0in} \addtolength{\topmargin}{-\headheight}
% \addtolength{\topmargin}{-\headsep}
%
% \setlength{\oddsidemargin}{0in}
%\newcommand{\alphat}{\ensuremath{\alpha_{\text{T}}}\xspace}
\DeclareRobustCommand{\alphat}{$\alpha_{\text{T}}~$}
\DeclareRobustCommand{\met}{$\mbox{$E_\text{T}^{\rm miss}$}\xspace$}


% \oddsidemargin  0.0in \evensidemargin 0.0in

\pagestyle{fancy}\lhead{Matthew Citron}\rhead{November 2016}
\chead{{\large{\bf Outreach statement}}} \lfoot{} \cfoot{\bf \thepage}

\newcounter{list}

\begin{document}
\section*{Outreach statement}
%\section*{\fontsize{30}{30}\selectfont Statement of intent}
\noindent 
I believe that outreach is an integral part of an academic career and greatly appreciate the opportunity 
to interact with with students with a scientific interest as well as the public as a whole. This is a vital 
undertaking to maintain support for fundamental research as well as to inspire the next generation of scientists.

During my Phd, I have enjoyed giving tours of CERN and CMS to members of the public, 
including members of the UK government. I believe that such opportunities to visit experiments
and interact with researchers is the best way to engage people and to share the importance and 
excitement of our work. I will continue performing such outreach activities, for example tours, 
public lectures and school visits. I believe the  ’Saturday morning physics’ programme is a very
good starting point and I plan to explore other opportunities offered by the Lederman science center. 

I am particularly interested in engaging school students who are interested in learning 
about scientific work. While at school I won a bursary to spend a month working in an 
astrophysics department. This experience played a crucial role in cementing my desire
to study and pursue a career in physics. In addition, during my PhD I have supervised pupils undertaking 
week long research projects on two occasions. Having both undertaken and supervised such projects, I am convinced that they are vital 
in inspiring students to pursue scientific careers. At Fermilab I would like to be involved 
with the Summer Internship programme and, if possible, supervise a high school student or teacher during the summer.

Fermilab as the main US particle physics laboratory has a very large scope of research topics from 
collider to intensity beams and astrophysics. With its leading contributions not only to CMS but as well 
as e.g. the g-2 experiment, NO$\nu$A/Dune, DES and Dark Matter searches I am looking forward to present our 
research to a keen public and also gain more insights myself.



\end{document}


%% end of file `template.tex'.


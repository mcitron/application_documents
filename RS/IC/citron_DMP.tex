\documentclass[11pt,a4paper]{article}
\usepackage{amsmath}
\usepackage{cancel}
\usepackage{xspace}
\usepackage{amssymb}
\usepackage{amsthm}
\usepackage{amscd}
\usepackage{amsfonts}
\usepackage{graphicx}%
\usepackage{fancyhdr}
\usepackage{atbegshi}
\usepackage[dvipsnames]{xcolor}

% \usepackage[a4paper, total={7in, 8in}]{geometry}
\usepackage[top=2cm,bottom=2cm,left=2cm,right=2cm]{geometry}
\setlength\headheight{13.6pt}
\usepackage{fontspec} 
\setmainfont{Arial}

\theoremstyle{plain} \numberwithin{equation}{section}
\newtheorem{theorem}{Theorem}[section]
\newtheorem{corollary}[theorem]{Corollary}
\newtheorem{conjecture}{Conjecture}
\newtheorem{lemma}[theorem]{Lemma}
\newtheorem{proposition}[theorem]{Proposition}
\theoremstyle{definition}
\newtheorem{definition}[theorem]{Definition}
\newtheorem{finalremark}[theorem]{Final Remark}
\newtheorem{remark}[theorem]{Remark}
\newtheorem{example}[theorem]{Example}
\newtheorem{question}{Question} \topmargin-2cm
% \renewcommand\refname{Five most important publications}

\pagestyle{fancy}\lhead{Matthew Citron}\rhead{September 2020}
\chead{{\large{\bf Data management plan}}} \lfoot{} \cfoot{\bf \thepage}

\DeclareRobustCommand{\alphat}{$\alpha_{\text{T}}~$}
\DeclareRobustCommand{\met}{$\mbox{$E_\text{T}^{\rm miss}$}\xspace$}
\DeclareRobustCommand{\ifb}{$\rm{fb}^{-1}~$}
\usepackage{cite}
\usepackage{hyperref}

\newcounter{list}

\begin{document}
\section*{Introduction}

The data management strategy outlined in this document follows the recommendations
outlined by the Data Preservation for HEP (DHEP) group. The data management and 
data processing for the CMS experiment is governed by the Computing Models of the WLCG and 
the LHC Experiments (\href{http://cds.cern.ch/record/1695401}{http://cds.cern.ch/record/1695401}). 
The Computing Models ensure that multiple copies of all raw data 
are stored in separate facilities, resilient metadata catalogues and
experimental conditions dataavases are maintained, 
and software versions are stored and indexed. As all data can be recreated from 
these raw data, these models ensure robust data preservation for current and future
generations of particle physicists. While the milliQan
experiment is not covered within these Computing Models, the same standards of data preservation
will be met. 

\section*{Data processing and storage}

The types of data that are produced by particle physics experiments can
be summarised into four ``levels":

\begin{description}
    \item [Level-4] {The most basic raw data produced by the experiments from which all other data can be derived.}
    \item [Level-3] {Reconstructed data that can be used by researchers for physics analysis.}
    \item [Level-2] {Data that can be used for outreach or provided to external researchers.}
    \item [Level-1] {Published analysis results}
\end{description}

The Level-4 and Level-3 data produced by the CMS experiment is entirely stored within the WLCG. 
The raw data is passed rapidly from the CMS experimental site to
the Tier-0 facility where it is duplicated to tape. The data is propagated by a 10Gbit/s or higher data transfer 
to Tier-1 data centres where a second copy of the data is stored on tape. CERN and 
remote data facilities assume responsibility for the indefinite storage and future accessibility of the raw data.
Reconstruction algorithms are run to produce Level-3 data that is stored at Tier-1 and Tier-2 facilities. 
This data is maintained on disk and duplicated depending on its popularity. Several hundred PB of
storage within the WLCG is required on both disk and tape.
The software used to reconstruct the data must also be preserved. This is maintained, along with
detailed documentation on dedicated Github and Gitlab respositories.
% Simulated data can in principle always be recreated provided 
% the software and the associated transforms have been preserved. 
% However out of prudence some MC data is also preserved along with associated real data.

The Level-4 data produced by the milliQan prototype is maintained
on disk at the University of California, Santa Barbara. This raw data is duplicated and stored
on external hard-drives and a dedicated database is maintained to ensure future accessibility. 
The total size of the dataset collected by the prototype is around a few hundred TB and
a similar dataset size is foreseen for the phase-1 milliQan detector. In order to improve the 
site redundancy an additional copy of the raw data will be produced and stored on disk at 
facilities in the US and Europe. The Level-3 data is stored on University of California supported
cloud-based storage that is secure and FERPA compliant. The software used to reconstruct the 
data is maintained on dedicated Github repositories. The metadata on the software version used
for the reconstruction of data for publication will be preserved on Github. 
The conditions of the detector during data-taking,
including the HV values applied to the PMTs and the magnetic field in the cavern will be stored
within the Level-4 data.

The analyses planned within this proposal will make use of the CERN Analysis Preservation (CAP) service
to facilitate their future reproduction. This service 
allows completed anlayses to be preserved, including resources such as files, code, workflows 
and metadata, on a centralised platform (\href{https://analysispreservation.cern.ch/welcome}{https://analysispreservation.cern.ch/welcome}).

\section*{Public access to data}

Results of the research will be
disseminated in standard refereed journals. Additional information
will be maintained on resources including the 
CERN document server (CDS) and HepData. The software required to access the data
on CDS (primarily ROOT) is publicly available and documented. The data 
format (YAML) for HepData is designed for long-term legibility and validity
of the data. For milliQan, Level-1 and Level-3 data
will also be shared online and archived on UC supported cloud-based storage.
Sufficient information will be provided to allow searches undertaken as
part of this proposal to be reinterpreted by researches outside of
the CMS and milliQan collaborations.

Select CMS Level-2 and Level-3 data, including simulated data, will be released on the CERN Open Data Portal 
some years after the data is collected. This data is suitable for both educational and research purposes
and are issued with Digital Object Identifiers such that they may be cited.

\end{document}


%% end of file `template.tex'.


\documentclass[11pt,twocolumn]{article}
\usepackage{amsmath}
\usepackage{cancel}
\usepackage{xspace}
\usepackage{amssymb}
\usepackage{amsthm}
\usepackage{amscd}
\usepackage{amsfonts}
\usepackage{graphicx}%
\usepackage{fancyhdr}
\usepackage[dvipsnames]{xcolor}

% \usepackage[a4paper, total={7in, 8in}]{geometry}
\usepackage[bottom=0.4in,left=0.8in,right=0.8in]{geometry}
\usepackage{fancyhdr}

\theoremstyle{plain} \numberwithin{equation}{section}
\newtheorem{theorem}{Theorem}[section]
\newtheorem{corollary}[theorem]{Corollary}
\newtheorem{conjecture}{Conjecture}
\newtheorem{lemma}[theorem]{Lemma}
\newtheorem{proposition}[theorem]{Proposition}
\theoremstyle{definition}
\newtheorem{definition}[theorem]{Definition}
\newtheorem{finalremark}[theorem]{Final Remark}
\newtheorem{remark}[theorem]{Remark}
\newtheorem{example}[theorem]{Example}
\newtheorem{question}{Question} \topmargin-2cm
% \renewcommand\refname{Five most important publications}

\pagestyle{fancy}\lhead{Matthew Citron}\rhead{July 2020}
\chead{{\large{\bf Searching for LLPs at the LHC}}} \lfoot{} \cfoot{\bf \thepage}

\DeclareRobustCommand{\alphat}{$\alpha_{\text{T}}~$}
\DeclareRobustCommand{\met}{$\mbox{$E_\text{T}^{\rm miss}$}\xspace$}
\DeclareRobustCommand{\ifb}{$\rm{fb}^{-1}~$}

\newcounter{list}

\begin{document}
\section*{Introduction}
%\section*{\fontsize{30}{30}\selectfont Statement of intent}
% \noindent 

The most important target of the LHC is the discovery of new physics.
Most searches at the LHC have focused on prompt decays of new particles,
however, long-lived particles (LLPs), with a non-negligible lifetime,
are a common feature in models of 
new physics. 
The discovery of a new LLP would be a powerful probe as measurement 
of the lifetime provides insight into the fundamental symmetries
and hierarchies of scale in the underlying model.
Searching for long-lived signatures is particularly 
challenging due to the necessity of dedicated reconstruction 
and triggering, and the prediction and rejection of non-standard backgrounds. 
I am going after this challenge in three ways: exploiting
the general purpose CMS detector in new and innovative ways,
collaborating with phenomeonlogists in the long-lived community to 
explore new ideas and techniques, and constructing a new dedicated 
detector for millicharged particles.

I have taken new approaches in searching for LLPs with
the CMS detector, carrying out an analysis
that pioneered the use of calorimetry timing 
to search for long-lived particles.
This analysis was the first at CMS to target hadronic decays 
beyond the acceptance of the tracking 
detector. The results were shown publicly at the Moriond Electroweak 
conference in 2018 and published 
in PLB in October 2019~\cite{2019134876}, 
the first published search from CMS or ATLAS using the 
full 135 \ifb Run 2 data set provided by the LHC at $\sqrt{s} = 13$ TeV. 

After completing the timing search I was selected as one of the two 
coordinators of the CMS physics subgroup responsible for overseeing
the long-lived search program. I have made encouraging 
new ideas a key thrust of my work, particularly in the 
crucial areas of dedicated triggering and reconstruction of long-lived signatures.
To this end, I have directly engaged with
trigger, reconstruction and detector experts as well as organising the 
first CMS long-lived workshop. I believe communication with theorists and experimentalists
across the field is critical to keeping a broad view of well-motivated
new physics and new ideas for discovery at detectors. I have engaged
in phenomenological studies throughout my research~\cite{deVries:2015hva,Citron:2012fg} 
and am actively involved in coordination efforts for the wider LLP community
As part of the LHC LLP community organising committee, I
organised the LHC LLP workshop in May 2020 and am currently
coordinating a study for the US snowmass process.

I am active in an experiment beyond CMS, with a leading 
role in the construction and data analysis of a prototype 
millicharged particle detector
at the LHC (the milliQan prototype). In particular,
coordinating a search for particles with charges much
smaller than the electron charge
that provides world-leading constraints. 

In the coming years, the jumps in energies and luminosity that marked the previous running of the 
LHC will come to an end. While inclusive analyses will continue to play a role in 
exploring ever larger phase space, new ideas and detectors will be critical to 
providing the best chance for discovery. In the following paragraphs I describe
how I will use my fellowship to search for long-lived signatures at the LHC.

\section*{Searching for LLPs with CMS}

One of the most challenging, but well-motivated, signatures
of new physics is the soft decay of LLPs to jets. Such soft signatures can 
arise for long-lived particles produced though
well-motivated Higgs or vector boson decays, as well as for ``compressed'' models 
that have small mass splittings, a common 
feature of new physics models with long-lived particles. 
The low energy scale of such signatures means general purpose triggers
are inefficient and rejecting backgrounds with the current CMS subdetectors 
is challenging. New dedicated triggers and CMS detector upgrades will 
therefore provide the best prospects
for discovery of such signatures, however, the Run 2 data set provides
the perfect opportunity to test new techniques and search strategies
and characterise non-standard backgrounds. To this end I have initiated the 
first search for displaced hadronic decays in the CMS muon system.

In the first years of my fellowship, I plan to concentrate on prototyping 
searches with the Run 2 data set. This will include adapting the delayed jets search
to target lower energy LLPs, using prompt objects
in the final state to provide trigger acceptance. I am already collaborating with 
researchers at Imperial College on 
searching for displaced decays of sterile neutrinos
using a deep neural network to tag displaced jets. During my fellowship, I plan to develop
the machine learning tagger further to encompass more exotic signature such as dark-showers, in
which a parent particle promptly hadronises to a high 
multiplicity of intermediate long-lived dark-QCD particles that undergo displaced decays. 
This can lead to an ``emerging jet'' signature that so far has only
been experimentally explored for highly energetic events. For Run 2, a specialised data set triggered
with a soft displaced single muon, the b-parked data set, provides an ideal sample to search for
such a signature.

As we prepare for the upcoming Run 3 of the LHC, which is expected to provide a $\sim 300$ \ifb
data set from 2022--2025 at $\sqrt{s} = 14$ TeV, it is clear that continuing to slowly accrue
data at a similar energy to Run 2 provides little prospect for the discovery of new physics.
It is therefore vital to develop new trigger strategies targeting exotic signatures
such as displaced decays of LLPs. I have directly engaged with trigger and hardware experts 
on multiple occasions to encourage the development of new strategies. 
I am currently working on hardware based 
triggers for the CMS HCAL, taking advantage of the depth segmentation and timing 
information provided by the recent upgrades to target a variety of signatures including displaced
decays and emerging jets. This builds on my previous experience 
implementing hardware trigger algorithms for jets and energy sums. 
In the first year of my fellowship I plan to
work with trigger experts at Imperial College to implement algorithms such as these
and will work to optimise their performance throughout Run 3. This will allow me to design 
searches with substantially improved sensitivity, building 
on prototype analyses with Run 2 data.

Looking forward to HL-LHC, scheduled to begin in 2027, there are ample opportunities
for detector upgrades to substantially improve sensitivity to LLPs with a 3000 \ifb
data set. One such new detector, the High Granularity Calorimeter, 
provides high resolution spatial, timing and energy deposits. Imperial College is actively involved in 
developing the clustering algorithms used to reconstruct particles showering in the detector. 
During my fellowship I plan to build on my experience with searches for 
displaced decays using calorimetry detectors, by collaborating with researches at Imperial College
to develop reconstruction strategies for displaced decays in new detectors such as the HGCAL.
As for Run 3, the trigger will be vital to extend acceptance to new physics and so I plan
to develop new strategies with the flexible architecture of FPGAs, largely being developed in 
the UK, to design hardware triggers for LLPs.

\section*{Searching for millicharged particles with milliQan}

The milliQan detector is a small dedicated detector 
designed to search for fractionally 
charged particles produced in the high energy proton-proton collisions of the LHC. 
Such particles can be well motivated by dark sector models where a kinetic mixing between
the dark and SM photon gives dark fermions a small electric charge.
The existing, general purpose detectors at the LHC are blind to particles with
$Q< 0.1 e$ as fractionally charged particles deposit only $(Q/e)^2$ of the energy 
deposited by a particle with charge $e$ of
the same mass. Therefore, a dedicated experiment is needed to 
provide sensitivity to the signatures of 
millicharged particles. The proposed experiment achieves 
this through the use of an array of long scintillator
bars properly aligned with the CMS interaction point, where the LHC beams are
brought together for collisions. Each scintillator bar is coupled to a
Photomultiplier Tube (PMT) capable of detecting a single scintillation photon, providing sensitivity to
charges as low as $0.001 e$. 

Over the past few years I have led the design, operation and analysis
of data from a small prototype of the milliQan detector. I have taken active involvement
in the construction of the scintillator and PMT components. With the data
collected during the 2018 running of the LHC, I coordinated the activities
of a group of approximately 15 postdocs, graduate students 
and undergraduates to calibrate the detector and undertake a
search for millicharge particles with world-leading sensitivity~\cite{ball2020search}. 

In the course of my fellowship I would like to continue my leadership role 
within the milliQan collaboration. We are currently seeking around £$1.6m$ in 
funding to build a large scale detector in time for the HL-LHC. 
However, with the lessons learnt from the 
prototype I have designed a smaller experiment, the phase-1 milliQan detector, 
that may be installed in time for the upcoming run of the LHC. 
This detector, with a cost of only £$\sim$40k, uses an additional layer of scintillator 
as well as electronic noise
filtering and amplification of pulses from the PMTs to overcome limitations of the prototype
and allow the reach for millicharged particles to be extended by 
up to an order of magnitude in charge. I plan
to construct the scintillator plus PMT
components at Imperial College before assembling the detector in the experimental cavern
in time for Run 3 of the LHC.
The construction of detector components is relatively simple and, 
as for the prototype, can be largely undertaken
by students, with my supervision. Given the experience with the
prototype, this will be carried out with high efficiency to collect physics quality
data. Similar strategies to the prototype will be used to calibrate the response and timing of the 
detector, simulate signal and background processes and analyse the data to 
search for millicharged particles.
 
I also plan to further my role with the full-scale milliQan detector, as well
as support similar proposals for scintillation based detectors at Fermilab and J-PARC, which
have complementary sensitivity for lower mass millicharged particles. I am coordinating an effort to
robustly estimate the reach of these detectors, given the experience from the prototype. 
If funding becomes available, I will take a leading
role in the design and construction of the detector as well as the data analysis.
% and I am coordinating the projections for a range of similar detectors

\section*{Summary}

As a Postdoctoral Scholar working on CMS and milliQan
I have seized the opportunity to take leading roles within both collaborations
as well as contributing to the larger long-lived community.
Over the next five years, I plan to 
play a leading role within the milliQan collaboration, 
prototype new searches for LLPs with Run 2 data, develop new triggers and
analyses for the upcoming LHC Run 3, and ensure detectors at the HL-LHC
provide optimal sensitivity to LLPs. I believe my wide experience of the experimental
and theoretical challenges in searching for LLPs will be vital assets in achieving
the discovery of LLPs at the LHC. Finally, Imperial College is an ideal host institute, providing
complementary technical expertise to allow the hardware deployment of new triggers and LLP
reconstruction strategies.


\section*{Proposed timeline}
\begin{description}
\item[Year 1 (2021)]{Searches for light and compressed LLPs with Run 2 dataset, develop new trigger seeds
targeting displaced decays, construct phase-1 milliQan detector in time for start of next run of the LHC}
\item[Year 2 (2022)]{Run 3 of the LHC begins, publish results of Run 2 searches with improved limits or discovery, use experience to improve triggers for long-lived signatures and design searches with Run 3 data, operate phase-1 milliQan detector during Run 3, begin construction of millicharge detectors at Fermilab and/or J-PARC (if funding available)}
\item[Year 3 (2023)]{Publish first search results from Run 3 taking advantage of new triggers, develop algorithms for LLP triggering and reconstruction for HL-LHC detectors, continue to operate and publish first search results with phase-1 milliQan detector}
\item[Year 4 (2024)]{Publish search results from Run 3 using improved triggers developed for 2023 data-taking, continue to develop LLP algorithms for HL-LHC detectors}
\item[Year 5 (2025)]{Publish full Run 3 search results characterising excess or with substantially improved limits, begin construction of full milliQan detector at LHC (if funding available)}
\end{description}

\bibliographystyle{unsrt}
\bibliography{citron_RS} % 'citron' is the name of a BibTeX file

\end{document}


%% end of file `template.tex'.


\documentclass[11pt,a4paper]{article}
\usepackage{amsmath}
\usepackage{cancel}
\usepackage{xspace}
\usepackage{amssymb}
\usepackage{amsthm}
\usepackage{amscd}
\usepackage{amsfonts}
\usepackage{graphicx}%
\usepackage{fancyhdr}
\usepackage[dvipsnames]{xcolor}

% \usepackage[a4paper, total={7in, 8in}]{geometry}
\usepackage[top=2cm,bottom=2cm,left=2cm,right=2cm]{geometry}
\setlength\headheight{13.6pt}
\usepackage{fontspec} 
\setmainfont{Arial}

\theoremstyle{plain} \numberwithin{equation}{section}
\newtheorem{theorem}{Theorem}[section]
\newtheorem{corollary}[theorem]{Corollary}
\newtheorem{conjecture}{Conjecture}
\newtheorem{lemma}[theorem]{Lemma}
\newtheorem{proposition}[theorem]{Proposition}
\theoremstyle{definition}
\newtheorem{definition}[theorem]{Definition}
\newtheorem{finalremark}[theorem]{Final Remark}
\newtheorem{remark}[theorem]{Remark}
\newtheorem{example}[theorem]{Example}
\newtheorem{question}{Question} \topmargin-2cm
% \renewcommand\refname{Five most important publications}

\pagestyle{fancy}\lhead{Matthew Citron}\rhead{August 2020}
\chead{{\large{\bf Searching for LLPs at the LHC}}} \lfoot{} \cfoot{\bf \thepage}

\DeclareRobustCommand{\alphat}{$\alpha_{\text{T}}~$}
\DeclareRobustCommand{\met}{$\mbox{$E_\text{T}^{\rm miss}$}\xspace$}
\DeclareRobustCommand{\ifb}{$\rm{fb}^{-1}~$}

\newcounter{list}

\begin{document}
\section*{Introduction}
%\section*{\fontsize{30}{30}\selectfont Statement of intent}
% \noindent 

The most important target of the LHC is the discovery of new physics.
While most searches at the LHC have focused on prompt decays,
long-lived particles (LLPs), with a non-negligible lifetime, are a common feature in models of 
new physics. The discovery of a new LLP would form a powerful probe as measurement 
of its lifetime can provide insight into the fundamental symmetries
and hierarchies of scale in the underlying model.
Searching for long-lived signatures is particularly 
challenging due to the necessity of dedicated reconstruction 
and triggering, and the prediction and rejection of non-standard backgrounds. 
I am going after this challenge in two ways: exploiting
the general purpose CMS detector in new and innovative ways,
and constructing a new dedicated detector for long-lived particles
with a fractional electric charge.
In undertaking these efforts, I will continue to strengthen my collaboration
with phenomenologists to explore new ideas and techniques.

I have taken new approaches in searching for LLPs with
the CMS detector, pioneering the use of calorimetry timing 
to search for LLPs.
This analysis was the first at CMS to target hadronic decays 
beyond the acceptance of the tracking 
detector. The results were shown publicly at the Moriond Electroweak 
conference in 2018 and published 
in PLB in October 2019~\cite{2019134876}, 
the first published search from CMS or ATLAS using the 
full 135 \ifb Run 2 data set provided by the LHC at $\sqrt{s} = 13$ TeV. 
Since completing this search I have continued to generate new ideas,
initiating the first CMS search for hadronic decays in the muon system
and developing new triggers. After completing the timing search I was selected as one of two 
coordinators of the CMS physics subgroup responsible for overseeing
the long-lived search program. I have made encouraging 
new ideas a key thrust of my work, directly engaging with
trigger, reconstruction and detector experts as well as organising the 
first CMS long-lived workshop. 

I am active in an another experiment besides CMS, the proposed milliQan
detector for millicharged particles. I led the 
construction and data analysis of a prototype detector
at the LHC and designed a search 
for particles with charges much smaller than the electron charge
that provides world-leading constraints. I have also used this data to 
characterise the backgrounds and 
develop significant improvements for the design of the full detector.

I believe communication with theorists and experimentalists
across the field is critical to keeping a broad view of well-motivated
new physics and new ideas for discovery at detectors. I have engaged
in phenomenological studies throughout my research~\cite{deVries:2015hva,Citron:2012fg} 
and am actively involved in coordination efforts for the wider LLP community.
I am coordinating a study for the US Snowmass process and, 
as a member of the LHC LLP community organising committee, I
organised the LHC LLP workshop in May 2020.

In the coming years, the jumps in energies and luminosity that marked the previous runs of the 
LHC will come to an end. While inclusive analyses will continue to play a role, 
new ideas and detectors will be critical to 
providing the best chance for discovery. In the following paragraphs, I describe
how I will use my fellowship to search for long-lived signatures at the LHC.

\section*{Searching for LLPs with CMS}

One of the most challenging, but well-motivated, signatures
of new physics is the hadronic decay of LLPs without large
energy deposits in the final state. Such signatures can 
arise for LLPs produced though
well-motivated Higgs or vector boson decays, as well as for ``compressed'' models 
that have small mass splittings, a common 
feature of new physics models with LLPs.
The low energy scale of such signatures means general purpose triggers
are inefficient and rejecting backgrounds with the current CMS sub-detectors 
is challenging. New dedicated triggers and CMS detector upgrades will 
therefore provide the best prospects
for discovery. However, the Run 2 data set provides
the perfect opportunity to develop new techniques 
and characterise non-standard backgrounds, allowing new searches
that can already provide significantly improved sensitivity to a range
of signatures. Throughout my fellowship, I plan to collaborate with
phenomenologists to identify strategies to target such LLP signatures in current
and future runs from the LHC.
    
In the first years of my fellowship, I plan to continue to
explore new techniques with the Run 2 data set. This will include 
adapting the calorimetry timing search
to target lower energy LLPs, using prompt objects
in the final state to provide trigger acceptance. I am already collaborating with 
researchers at Imperial College (IC) on 
searching for displaced decays of sterile neutrinos
using a deep neural network to tag displaced jets. During my fellowship, I plan to develop
the machine learning tagger further to encompass more exotic signatures such as dark-showers, in
which a parent particle promptly hadronises to a high 
multiplicity of intermediate long-lived dark-QCD particles that undergo displaced decays. 
This can lead to an ``emerging jet'' signature that so far has only
been experimentally explored for highly energetic events. For Run 2, a specialised data set triggered
with a soft displaced single muon, the b-parked data set, provides an ideal sample to search for
such a signature.

The upcoming Run 3 of the LHC is expected to provide a $\sim 300$ \ifb
data set from 2022--2025 at $\sqrt{s} = 14$ TeV. 
I plan to significantly improve the sensitivity
of searches for long-lived particles by developing dedicated triggers to 
dramatically increase signal acceptance.
I have directly engaged with trigger and hardware experts 
to encourage the development of new strategies
and am working on hardware-based triggers for the CMS HCAL, 
taking advantage of the depth segmentation and timing 
information provided by the recent upgrades to target a variety of signatures, 
including displaced decays and emerging jets. This builds on my previous experience 
implementing hardware trigger algorithms for jets and energy sums. 
In the first year of my fellowship I plan to
work with trigger experts at IC to implement such algorithms and will 
work to optimise their performance throughout Run 3. This will allow me to design 
searches with substantially improved sensitivity and characterise the new physics in
the case of a discovery. 

The HL-LHC, scheduled to begin in 2027, is expected to provide a $\sim 3000$ \ifb
data set. Coupled with planned detector upgrades, the sensitivity to long-lived particles
can be substantially extended. One such upgrade project, the High Granularity Calorimeter, 
provides high resolution spatial, timing and energy measurements. IC researchers are involved in 
developing the clustering algorithms used to reconstruct particles showering in the detector. 
During my fellowship I plan to build on my experience with searches for 
displaced decays using calorimetry detectors, by collaborating with researchers at IC
to develop reconstruction methods for displaced decays in new detectors such as the HGCAL.
As for Run 3, the trigger will be vital to extend acceptance to new physics and so I plan
to develop new LLP trigger strategies by exploiting the flexible architecture of FPGAs, 
largely being developed in the UK. 


\section*{Searching for millicharged LLPs with milliQan}

The CMS detector can provide impressive sensitivity to a wide
range of new physics, however, certain long-lived signatures
require a dedicated detector to allow discovery. Millicharged 
particles can be well-motivated by dark sector 
models where a kinetic mixing between the dark and SM photon 
gives dark fermions a small electric charge.
The existing general purpose detectors at the LHC are blind to particles with
$Q< 0.1 e$ as they deposit only $(Q/e)^2$ of the energy 
deposited by a particle with charge $e$ of
the same mass. However, the proposed milliQan detector can cover this blind spot 
by using an array of long scintillator
bars aligned with the CMS interaction point, where the LHC beams are
brought together for collisions. To allow sensitivity to charges as low 
as $0.001 e$, each scintillator bar must be coupled to a
Photomultiplier Tube (PMT) capable of detecting a single scintillation photon.

Over the past few years I have led the design, construction and operation 
of a small prototype of the milliQan detector. Using the data
collected during the 2018 running of the LHC, I coordinated the activities
of a group of approximately 15 postdocs, graduate students 
and undergraduates to calibrate the detector and undertake a
search for millicharge particles with world-leading sensitivity~\cite{ball2020search}. 

In the course of my fellowship I plan to continue my leadership 
within the milliQan collaboration. Using the lessons learnt from the 
prototype I have designed an upgraded experiment, the phase-1 milliQan detector, 
that can be installed in time for Run 3 of the LHC. 
This detector uses an additional layer of scintillator 
as well as electronic noise
filtering and amplification of pulses from the PMTs to overcome limitations of the prototype
and allow the reach for millicharged particles to be lowered by 
up to an order of magnitude in charge. I received the Harvey L. Karp Discovery Award
based on my proposal to build the phase-1 milliQan detector that 
will provide \textsterling40k in funding to 
fully construct this experiment. 
I plan to construct the scintillator and PMT
components at IC before assembling the detector in the experimental cavern
in time for Run 3 of the LHC.  Similar strategies to the prototype will be used to 
calibrate the response and timing of the 
detector, simulate signal and background processes and analyse the data to 
search for millicharged particles. % The construction of detector components is relatively simple and, 
% as for the prototype, can be largely undertaken
% by students, with my supervision. Given the experience with the
% prototype, this will be carried out with high efficiency to collect physics quality
% data.

The phase-1 milliQan detector will be an important intermediate stage for building a full-scale
milliQan detector at the HL-LHC. I plan to continue my role in the design, construction and
operation of this experiment. In the event of a discovery, the milliQan detector can be used to trigger
the CMS detector to allow the new physics to be fully characterised. 
I will use my leading roles in both experiments to develop this trigger.
Finally, I will contribute to similar proposals for scintillator-based detectors at Fermilab and J-PARC, 
which have complementary sensitivity for lower mass particles. I am coordinating an effort to
robustly estimate the reach of all such millicharged particle detectors.

\section*{Summary}

As a Postdoctoral Scholar working on CMS and milliQan
I have seized the opportunity to take leading roles within both collaborations
as well as contributing to the larger LLP research community.
My experience with searches, phenomenology and 
new detectors means I am uniquely positioned to exploit the data
from the LHC to achieve a discovery of new long-lived physics.
Over the next five years, I plan to use this fellowship to establish
myself as an independent researcher. I will play a leading role 
within the milliQan collaboration, develop new searches for LLPs with Run 2 data, 
design new triggers and analyses for the upcoming LHC Run 3, and ensure detectors at the HL-LHC
provide optimal sensitivity. Finally, IC is an ideal host institute, providing
complementary technical expertise to allow the hardware deployment of new triggers and
reconstruction strategies. I will develop IC's role at the forefront
of the search for new physics and as a leader in the CMS trigger and
upgrade projects.


\section*{Proposed timeline}
\begin{description}
\item[Year 1 (2021)]{Searches for LLPs with Run 2 dataset, develop new trigger seeds
targeting displaced decays, construct phase-1 milliQan detector, pursue phenomenological study of signatures such as emerging jets}
\item[Year 2 (2022)]{Run 3 of the LHC begins, publish results of Run 2 searches, continue to develop triggers and design searches for long-lived signatures with Run 3 data, operate phase-1 milliQan detector, begin construction of millicharged particle detectors at Fermilab and/or J-PARC (if funding available)}
\item[Year 3 (2023)]{Publish first search results from Run 3 taking advantage of new triggers, develop algorithms for LLP triggering and reconstruction for HL-LHC detectors, continue to operate and publish first search results with phase-1 milliQan detector}
\item[Year 4 (2024)]{Publish LLP search results from Run 3 using improved triggers developed for 2023 data-taking, continue to develop LLP algorithms for HL-LHC detectors}
\item[Year 5 (2025)]{Publish full Run 3 search results, begin construction of full milliQan detector at LHC (if funding available)}
\end{description}

\bibliographystyle{unsrt}
\bibliography{citron_RS} % 'citron' is the name of a BibTeX file

\end{document}


%% end of file `template.tex'.


\documentclass[11pt,a4paper]{article}
\usepackage{amsmath}
\usepackage{cancel}
\usepackage{xspace}
\usepackage{amssymb}
\usepackage{amsthm}
\usepackage{amscd}
\usepackage{amsfonts}
\usepackage{graphicx}%
\usepackage{fancyhdr}
\usepackage{atbegshi}
\usepackage[dvipsnames]{xcolor}

% \usepackage[a4paper, total={7in, 8in}]{geometry}
\usepackage[top=2cm,bottom=2cm,left=2cm,right=2cm]{geometry}
\setlength\headheight{13.6pt}
\usepackage{fontspec} 
\setmainfont{Arial}

\theoremstyle{plain} \numberwithin{equation}{section}
\newtheorem{theorem}{Theorem}[section]
\newtheorem{corollary}[theorem]{Corollary}
\newtheorem{conjecture}{Conjecture}
\newtheorem{lemma}[theorem]{Lemma}
\newtheorem{proposition}[theorem]{Proposition}
\theoremstyle{definition}
\newtheorem{definition}[theorem]{Definition}
\newtheorem{finalremark}[theorem]{Final Remark}
\newtheorem{remark}[theorem]{Remark}
\newtheorem{example}[theorem]{Example}
\newtheorem{question}{Question} \topmargin-2cm
% \renewcommand\refname{Five most important publications}

\pagestyle{fancy}\lhead{Matthew Citron}\rhead{September 2020}
\chead{{\large{\bf Searching for LLPs at the LHC}}} \lfoot{} \cfoot{\bf \thepage}

\DeclareRobustCommand{\alphat}{$\alpha_{\text{T}}~$}
\DeclareRobustCommand{\met}{$\mbox{$E_\text{T}^{\rm miss}$}\xspace$}
\DeclareRobustCommand{\ifb}{$\rm{fb}^{-1}~$}

\newcounter{list}

\begin{document}
\section*{Introduction}
%\section*{\fontsize{30}{30}\selectfont Statement of intent}
% \noindent 

\nocite{ball2020search,2019134876,CMS:2016dbr,CMS:2015dbr,simplifiedLikelihood,Buckley_2019,Zabi:2016ljo,Kreis:2015jjr,Bagnaschi:2016afc,Bagnaschi:2015eha,Buchmueller:2014yva,Buchmueller:2012hv,deVries:2015hva,Buchmueller:2015uqa,Citron:2012fg,Alimena_2020}
The most important target of the CERN LHC is the discovery of new physics to
resolve fundamental mysteries in particle physics.
While the majority of searches at the LHC have focused on particles that 
decay promptly, long-lived particles (LLPs), with a non-negligible lifetime,
are a common feature in models that provide solutions to 
questions such as the origin of dark matter, baryogensis, and 
neutrino masses~\cite{Alimena_2020}. A particle with a 
non-negligible lifetime can arise for a variety of reasons, including small 
couplings between the LLP and its decay products, a heavy mediator for the decay, 
and an exact or approximate symmetry of the underlying model. 
Such mechanisms are known to exist 
throughout the Standard Model (SM), in which particles have
a wide range of lifetimes. Just as the properties of these particles 
have provided crucial insights into the nature of the SM, the discovery of a new LLP 
would form a powerful probe of the fundamental symmetries and hierarchies of scale in
the new physics model. 

While highly well-motivated, searching for long-lived signatures at the LHC is 
incredibly challenging because of the necessity of dedicated reconstruction 
and triggering, and the prediction and rejection of non-standard backgrounds. 
If awarded this fellowship, I will develop an independent research programme
that targets this challenge in two ways: exploiting
the general purpose CMS detector with new and innovative techniques,
and constructing a new dedicated detector for LLPs
with a fractional electric charge. In undertaking these efforts,
I will continue to strengthen my collaboration
with phenomenologists to explore new ideas and techniques.

Long-lived new physics signatures is a highly active and
growing research topic. I believe communication with theorists and experimentalists
across the field is critical to keeping a broad view of well-motivated
new physics and new ideas for discovery at detectors. I have engaged
in phenomenological studies throughout my research~\cite{deVries:2015hva,Citron:2012fg} 
and am actively involved in coordination efforts for the wider LLP community.
I am coordinating a study for the US Snowmass process and, 
as a member of the LHC LLP community organising committee, I
organised the LHC LLP workshop in May 2020 (with over 230 registered participants).

In the coming years, the jumps in energies and luminosity that marked the previous runs of the 
LHC will come to an end. While inclusive analyses will continue to play a role, 
new ideas and detectors will be critical to 
providing the best chance for discovery. In the following paragraphs, I describe
how I will use my fellowship to establish an independent research program 
to search for long-lived signatures at the LHC and beyond.

\section*{Searching for LLPs with CMS}

I have taken new approaches in searching for LLPs with
the CMS detector, pioneering the use of calorimetry timing. 
This analysis was the first at CMS to target hadronic decays 
beyond the acceptance of the tracking detector. 
The results were shown at the Moriond Electroweak 
conference in 2019 and published in October 2019~\cite{2019134876}, 
the first published search from CMS or ATLAS using the 
full 135 \ifb Run 2 dataset provided by the LHC at $\sqrt{s} = 13$ TeV. 
I have continued to generate new ideas, initiating the first 
CMS search for hadronic decays in the muon system
and developing new triggers. 
After completing the timing search, I was selected as one of two 
coordinators of the CMS physics subgroup responsible for overseeing
the long-lived search program. Since starting
my term, I have overseen five new searches that have been made public to the wider community 
and am currently reviewing around twenty searches with Run 2 data.
I have made encouraging new ideas a key thrust of my work, directly engaging with
trigger, reconstruction and detector experts as well as organising the 
first CMS long-lived workshop in January 2020. The focus of the workshop was 
on solutions to common issues for LLP searches and 
to highlight new directions in future runs of the LHC.  
The workshop was highly successful with over a hundred participants and
has been a starting point for multiple ongoing efforts. 

One of the most challenging, but well-motivated, signatures
of new physics is the hadronic decay of LLPs without large
energy deposits in the final state. This can
arise for LLPs produced though
well-motivated Higgs or vector boson decays, as well as for ``compressed'' models 
that have small mass splittings, a common 
feature of new physics models with LLPs.
For such low energy LLP signatures, the difficulty of triggering and rejecting dominant 
backgrounds has greatly limited sensitivity. Over the next years, I plan to overcome
these challenges by developing new dedicated triggers and strategies that 
take advantage of CMS detector upgrades to provide the best prospects for discovery.
The Run 2 dataset provides the perfect opportunity to develop new techniques 
and characterise non-standard backgrounds for challenging signatures of
LLPs. Such developments will also already allow new searches that can provide 
significantly improved sensitivity to a range of models. 
Throughout my fellowship, I plan to collaborate with
phenomenologists to identify and improve strategies to target LLP signatures in current
and future runs of the LHC.
    
In the first years of my fellowship, I plan to continue to
explore new techniques with the Run 2 dataset. This will include 
adapting the calorimetry timing search
to target lower energy LLPs, using prompt objects
in the final state to provide trigger acceptance. I am already collaborating with 
researchers at Imperial College (IC) on a
search for displaced decays of sterile neutrinos that
uses a deep neural network to tag displaced jets. During my fellowship, I plan to develop
the machine learning tagger further to encompass more exotic signatures such as dark-showers, in
which a parent particle promptly hadronises to a high 
multiplicity of intermediate long-lived dark-QCD particles that undergo displaced decays. 
This can lead to an ``emerging jet'' signature that so far has only
been experimentally explored for highly energetic events. For Run 2, a specialised dataset triggered
with a soft displaced single muon, the b-parked dataset, provides an ideal sample to search for
such a signature.

The upcoming Run 3 of the LHC is expected to provide a $\sim 300$ \ifb
dataset from 2022--2025 at $\sqrt{s} = 14$ TeV. 
I plan to significantly improve the sensitivity
of searches for long-lived particles by developing dedicated triggers to 
dramatically increase signal acceptance.
I have directly engaged with trigger and hardware experts 
to encourage the development of new strategies
and am working on hardware-based triggers for the CMS HCAL, 
taking advantage of the depth segmentation and timing 
information provided by the recent upgrades to target a variety of signatures, 
including displaced decays and emerging jets. This builds on my previous experience 
designing and implementing a hardware trigger algorithm for jets~\cite{Zabi:2016ljo,Kreis:2015jjr}. This algorithm
ran successfully throughout Run 2 of the LHC. Researchers at IC
have played a leading role in the CMS hardware-based trigger since the start of LHC running.
In the first year of my fellowship I plan to work with experts at IC to
implement dedicated LLP trigger algorithms and continue to optimise their performance throughout Run 3. 
The acceptance of the software-based high level trigger (HLT) can be greatly improved by
closing the large gap between the plateau value of the L1 trigger requirement
and the HLT threshold for quantities such as jet and energy sums. This can be achieved through
making requirements on displaced and delayed objects in the event. These trigger developments
will allow me to design searches with substantially improved sensitivity 
and characterise the new physics in
the case of a discovery. 

The HL-LHC, scheduled to begin in 2027, is expected to provide a $\sim 3000$ \ifb
dataset. The operating environment will be highly challenging with an increase in simultaneous
interactions from an average of 40 to 140 per event. Despite this challenge, 
detector upgrades will provide an incredible opportunity to extend sensitivity
to long-lived particles. These include a new timing detector for minimally ionising particles (MIPs)
the MIP timing detector (MTD), upgrades to the 
Electromagnetic Calorimeter (ECAL) that will improve the timing resolution by an order of magnitude
for electromagnetic energy deposits and a new High Granularity Calorimeter
(HGCAL) for the CMS endcap, which will provide high resolution spatial,
timing and energy measurements. For the HGCAL, IC researchers are involved in 
developing the clustering algorithms used to reconstruct particles showering in the detector. 
During my fellowship I plan to build on my experience with searches for 
displaced decays using calorimetry detectors, by collaborating with researchers at IC
to develop reconstruction methods for displaced decays with new detectors such as the HGCAL.
As for Run 3, the trigger will be vital to extend acceptance to new physics. I am currently 
engaged in a study of timing to trigger LLPs at the HL-LHC HLT, to be included
in the Phase-2 Technical Design Report. During my fellowship I plan to continue to develop new LLP trigger strategies.
In particular, exploiting the flexible architecture of FPGAs,
largely being developed in the UK, to provide high efficiency for LLPs at L1.

\section*{Searching for millicharged LLPs with milliQan}

The CMS detector can provide impressive sensitivity to a wide
range of new physics, however, certain long-lived signatures
require a dedicated detector to allow discovery. In recent years
the possibility that dark matter is not a single particle, but instead
a diverse set of particles with as complex a 
structure in their sector as the standard model, has risen in prominence.
In the case that the dark sector contains a massless dark photon, the main
physical effect is that new dark sector particles
that couple to the dark photon will have a small electric charge, a
"millicharge". Such millicharged LLPs may be produced copiously at the LHC, however,
the existing general purpose detectors will be blind to any particles with
$Q< 0.1 e$ as they deposit only $(Q/e)^2$ of the energy 
deposited by a particle with charge $e$ of
the same mass. A dedicated experiment, the milliQan detector, is required 
to cover this blind spot and exploit this unique opportunity 
for the discovery of millicharged particles.
Such a discovery would provide critical insight into the nature
of dark matter, one of the principle goals of the LHC.

The milliQan detector is composed of a large array of long scintillator bars
aligned with the CMS interaction point, where the LHC beams are brought together
for collisions. A large sensitive volume of scintillator 
is required to allow sensitivity to the small energy deposition of a millicharged
particle. To provide sensitivity to charges as low 
as $0.001 e$, each scintillator bar must be coupled to a
Photomultiplier Tube (PMT) capable of detecting a single scintillation photon.
I have played a leading role in the design, construction and operation
of a prototype milliQan detector that has been installed to 
prove the feasibility of such an experiment. In constructing the detector, I
supervised graduate and undergraduate students building 
scintillator components and coordinated interventions at CERN to 
assemble and maintain the experiment. Using the data from commissioning runs, 
I uncovered critical issues in the data acquisition that could be 
addressed early on in the running of the detector.
As physics coordinator for the milliQan experiment I oversee
the efforts of a group of approximately 15 postdocs, 
graduate students and undergraduates analysing the data from the milliQan prototype and
simulating the response of the detector to background and signal processes. 
With the data collected, I was able to calibrate the detector and simulation and undertake 
a search for millicharged particles with world-leading sensitivity~\cite{ball2020search}. 

In the course of my fellowship I plan to further my leadership 
within the milliQan collaboration. While successful, the sensitivity 
of the prototype was limited by exposure time, 
low efficiency for small energy deposits,
and background processes that became well understood through analysis of its data.
I have designed an upgraded experiment, the phase-1 milliQan detector, 
that can be installed in time for Run 3 of the LHC. 
This detector overcomes the limitations of the prototype by using
an additional layer of scintillator bars to suppress backgrounds to
a negligible level, as well as electronic noise
filtering and amplification of pulses from the PMTs to provide sensitivity significantly
lower energy deposits. Together these improvements will allow 
the reach for millicharged particles to be lowered by 
up to an order of magnitude in charge. I received the Harvey L. Karp Discovery Award
based on my proposal to build the phase-1 milliQan detector that 
will provide \textsterling40k in funding to 
fully construct this experiment. 

I plan to build the scintillator and PMT
components at IC before assembling the detector in the experimental cavern
in time for Run 3 of the LHC. 
The construction of detector
components is relatively simple and, as for the prototype, can be largely undertaken
by undergraduate researchers, with my supervision. This is an ideal 
opportunity to engage students in fundamental research, providing training 
for the next generation of scientists. Similar strategies to the prototype will be used to 
calibrate the response and timing of the 
detector, simulate signal and background processes and analyse the data to 
search for millicharged particles. Given my experience with the
prototype, this will be carried out with high efficiency to collect physics quality
data. In the event of a discovery, the milliQan 
detector can be used to trigger the CMS detector to allow the new physics to be fully characterised. 
I will use my leading roles in both experiments to develop this trigger.

Given the rapid data acquisition during Run 3 of the LHC, 
an initial search publication can be achieved by early to mid 2023 
with further publications to extend sensitivity or characterise 
an excess with enlarged datasets 
in early 2024 and 2025. With the funding for the phase-1 
milliQan detector already secured, the independence 
afforded by this fellowship will therefore allow me to establish myself as a 
leading figure in the search for millicharged particles.
The milliQan detector provides ample opportunity for public engagement 
and the training of graduate and undergraduate researchers. I will participate in the 
IC HEP masterclass program
as well as contribute to the IC summer festival to encourage engagement 
with scientific research.

The phase-1 milliQan detector will be an important intermediate stage for building 
a full-scale milliQan detector at the HL-LHC. During the operation of the phase-1 detector,
I plan to investigate the effectiveness of additional scintillator components for background
reduction as well as alternative readout electronics that may allow the cost of a full
detector to be substantially reduced. Finally, neutrino beam sources provide a 
complementary production mode for 
millicharged particles with significantly higher luminosity for masses under a few GeV. 
I will contribute to proposed scintillator-based detectors at neutrino sources at
Fermilab and J-PARC with similar designs to milliQan, as well as a proposed 
forward millicharged particle detector at the LHC (ForMINI). I am currently coordinating an effort to
robustly estimate the reach of all such millicharged particle detectors.
At IC, I will be able to profit from the active research program that
has been established by the HEP group at all three facilities. 

\section*{Summary}

As a Postdoctoral Scholar working on CMS and milliQan
I have seized the opportunity to take leading roles within both collaborations
as well as contributing to the larger LLP research community.
My experience with searches, phenomenology and 
new detectors means I am uniquely positioned to exploit the data
from the LHC to achieve a discovery of new long-lived physics.
Over the next five years, I plan to use this fellowship to establish
myself as an independent researcher. I will play a leading role 
within the milliQan collaboration, develop new searches for LLPs with Run 2 data, 
design new triggers and analyses for the upcoming LHC Run 3, and ensure detectors at the HL-LHC
provide optimal sensitivity. Finally, IC is an ideal host institute, providing
complementary technical expertise to allow the hardware deployment of new triggers and
reconstruction strategies. I will develop IC's role at the forefront
of the search for new physics and as a leader in the CMS trigger and
upgrade projects.


\section*{Proposed timeline}
\begin{description}
\item[Year 1 (2021)]{Searches for LLPs with Run 2 dataset, develop new trigger seeds
targeting displaced decays, construct and calibrate phase-1 milliQan detector, pursue phenomenological study of signatures such as emerging jets}
\item[Year 2 (2022)]{Run 3 of the LHC begins, publish results of Run 2 searches, deploy new triggers, develop searches for long-lived signatures with Run 3 data, operate phase-1 milliQan detector, begin construction of millicharged particle detectors at Fermilab and/or J-PARC (if funding available)}
\item[Year 3 (2023)]{Publish first search results from Run 3 taking advantage of new triggers, develop algorithms for LLP triggering and reconstruction for HL-LHC detectors, continue to operate and publish first search results with phase-1 milliQan detector}
\item[Year 4 (2024)]{Publish LLP search results from Run 3 using improved triggers developed for 2023 data taking, continue to develop LLP algorithms for HL-LHC detectors}
\item[Year 5 (2025)]{Publish full Run 3 search results, begin construction of full milliQan detector at LHC (if funding available)}
\end{description}
\it{All references herein refer to the accompanying publications list}
\newpage
\AtBeginShipout{%
\AtBeginShipoutDiscard
}
\bibliographystyle{unsrt}
\bibliography{citron_CV} % 'citron' is the name of a BibTeX file

\end{document}


%% end of file `template.tex'.


\documentclass[11pt]{article}
\usepackage{amsmath}
\usepackage{cancel}
\usepackage{xspace}
\usepackage{amssymb}
\usepackage{amsthm}
\usepackage{amscd}
\usepackage{amsfonts}
\usepackage{graphicx}%
\usepackage{fancyhdr}

% \usepackage[a4paper, total={7in, 8in}]{geometry}
\usepackage[bottom=0.4in,left=0.8in,right=0.8in]{geometry}
\usepackage{fancyhdr}

\theoremstyle{plain} \numberwithin{equation}{section}
\newtheorem{theorem}{Theorem}[section]
\newtheorem{corollary}[theorem]{Corollary}
\newtheorem{conjecture}{Conjecture}
\newtheorem{lemma}[theorem]{Lemma}
\newtheorem{proposition}[theorem]{Proposition}
\theoremstyle{definition}
\newtheorem{definition}[theorem]{Definition}
\newtheorem{finalremark}[theorem]{Final Remark}
\newtheorem{remark}[theorem]{Remark}
\newtheorem{example}[theorem]{Example}
\newtheorem{question}{Question} \topmargin-2cm
\renewcommand\refname{Five most important publications}

% \textwidth7in
%
% \setlength{\topmargin}{0in} \addtolength{\topmargin}{-\headheight}
% \addtolength{\topmargin}{-\headsep}
%
% \setlength{\oddsidemargin}{0in}
%\newcommand{\alphat}{\ensuremath{\alpha_{\text{T}}}\xspace}
\DeclareRobustCommand{\alphat}{$\alpha_{\text{T}}~$}
\DeclareRobustCommand{\met}{$\mbox{$E_\text{T}^{\rm miss}$}\xspace$}


% \oddsidemargin  0.0in \evensidemargin 0.0in

\pagestyle{fancy}\lhead{Matthew Citron}\rhead{May 2020}
\chead{{\large{\bf Vision of education and research}}} \lfoot{} \cfoot{\bf \thepage}

\newcounter{list}

\begin{document}
\section*{Vision of education and research}
%\section*{\fontsize{30}{30}\selectfont Statement of intent}
\noindent 

My research has focused on using data from the 
LHC to search for signatures of Beyond 
Standard Model (BSM) physics. Such BSM physics may allow fundamental issues
in the understanding of universe, such as the origin of dark matter, 
to be solved. Given the lack of an early discovery by inclusive searches in the first
years of operation of the LHC, I feel that the best prospects for discovery comes from exploiting 
general purpose detectors in new and innovative ways and pursuing the construction
of new dedicated detectors to search for exotic signatures. To support
this aim, I have had a leading role in the construction and
data analysis of a prototype millicharged particle (mcp) detector
at the LHC (the milliQan prototype). In particular,
coordinating a search for mcps
that provides world-leading constraints. 
In both building the detector and in data analysis I have taken
the opportunity to supervise and guide the work of 
many graduate and undergraduate students. As a member of the CMS collaboration 
I have pursued new approaches, undertaking a 
search that pioneered the use of calorimetry timing 
to search for long-lived particles (LLPs) at CMS and using my position as one of 
the two conveners of the LLP group within CMS to encourage new ideas, particularly in
the crucial areas of dedicated triggering and reconstruction of long-lived signatures.
% My search activities built on the experience gained during my PhD 
I believe communication with theorists and experimentalists
across the field is critical to keeping a broad view of well-motivated
BSM physics and new ideas for discovery at detectors. I have engaged
in phenomenological studies throughout my research and am
actively involved in coordination efforts for the wider LLP community.

In the coming years the jumps in energies and luminosity that marked the previous running of the 
LHC will come to an end. While inclusive analyses will continue to play a role in 
exploring ever larger phase space, new ideas and detectors will be critical to 
providing the best chance for discovery. In the following paragraphs I describe how the experience 
gained from my work on CMS and milliQan has prepared me well for the challenge 
of discovering BSM physics at the LHC.

\section*{Research activities}

\subsection*{The milliQan experiment}

In recent years there has been increasing interest in the possibility 
that astrophysical evidence for dark matter may indicate the existence 
of not just a single new particle but an entire dark sector
of particles and interactions with a rich and complex structure. The milliQan experiment 
was proposed in 2014 as a dedicated scintillator-based detector to search 
for millicharged particles that can arise in dark sector models. 

I became a member of the milliQan collaboration in Sept 2017 and immediately joined the
team installing a 1\% scale prototype designed to study the feasibility and
develop understanding of the experiment. I took the opportunity to develop
both my technical expertise and leadership skills, coordinating both
detector upgrades and data analysis. During the first commissioning run, 
from September 2017 to January 2018, I developed the framework 
needed to convert raw PMT waveforms into 
a format suitable for analysis and took part in monitoring
the data from the detector to ensure its smooth operation. 
In undertaking this monitoring, I uncovered critical issues such as unexpected
HV instabilities and failures in the data acquisition logic, which 
could be fixed early on in the running of the prototype.
This has provided important
experience in the rapid analysis and comprehension needed to operate
a complex detector in a challenging environment. 

In March 2018 I led the upgrade of the prototype, which built on the experience 
gained during the first operational run, to expand the size of the active area
sensitive to signal deposits and add new components designed to provide
active vetos for the signal volume and measure backgrounds. I helped design the upgraded prototype,
construct the PMT and scintillator components and 
supervised both undergraduate and graduate students at UCSB and CERN undertaking this work.
The upgraded prototype took data for commissioning from March--June 2018 
and data suitable for physics analysis was then taken from June until 
proton-proton collisions stopped in late October 2018. 
The lessons learnt from the commissioning runs of the prototype allowed highly 
efficient data collection for physics analysis.

In January 2018 I was tasked with the analysis of data from the 
milliQan prototype. This has involved coordinating the activities of a group of approximately
15 postdocs, graduate students and undergraduates from various institutes in the US, Europe and Asia. 
The analysis objectives included the calibration of the data from the detector,
characterising and measuring backgrounds, and simulating the generation and propagation of 
signals and backgrounds as well as the response of the detector. I was then able to design
a search for millicharged particles using the prototype detector. Despite using sub-optimal PMTs 
and being only a small fraction of the size of the full milliQan detector,
this search achieved world-leading sensitivity to fractionally charged particles and forms
a paper submitted to PRD~\cite{ball2020search}.

The background measurements and mitigation strategies that I coordinated
as part of the search with the milliQan prototype also provide important
lessons for the full-scale milliQan detector. In particular, I was able to show 
an additional layer of scintillator bars is necessary to achieve the required
level of background control. I therefore proposed an alteration 
in the design of the detector from three to four layers. The prototype has also 
been reconfigured to four layers and data taken from January--May 2019 to 
directly measure the background rates. This will be used to provide updated
projections in an upcoming paper (expected to be submitted by Autumn 2020).

\subsection*{Searches with CMS}

Over the last few years my main research interests with the CMS experiment
have been in searching for signatures of long-lived particles (LLPs). 
Such particles can appear in theories with a wide range of motivations, 
including providing solutions to fundamental problems of cosmology, naturalness considerations,
dark matter, and non-zero neutrino masses. The discovery
of a new LLP would be especially exciting as the non-trivial lifetime
provides a unique insight into the fundamental symmetries and hierarchies of scale in the underlying model.
Searching for long-lived signatures with a general purpose detector such as CMS is particularly 
challenging due to the necessity of dedicated reconstruction and triggering, and
the prediction and rejection of non-standard backgrounds. 

In 2018 I started work on searching for LLPs with CMS, 
pioneering the use of calorimetry timing to search for LLPs decaying to jets. This analysis was the first carried 
out at CMS to target hadronic decays beyond the acceptance of the tracking detector and required a detailed
understanding of the timing and energy reconstruction performance of the electromagnetic calorimeter. The analysis achieved 
world-leading limits for $\mathcal{O}$ TeV mass LLPs decaying to jets with $c\tau_{0} > 1$ m. 
The results were shown publicly at the Moriond Electroweak conference in 2018 and published in PLB in Octover 2019~\cite{2019134876}, 
the first published search from CMS or ATLAS using the full 13 TeV data set provided by the LHC. 

While carrying out this search I began engaging with the wider LLP physics community. 
By attending and presenting at multiple workshops I was able to build connections with 
experimentalists and theorists working on LLPs and attain a wide view of different 
search strategies and well-motivated theory models. In September 2019 I was appointed as
one of two CMS Exotica long-lived subgroup conveners, responsible for leading the 
LLP efforts within CMS as well as internal review of all CMS long-lived results. Since starting
my term, I have reviewed four new searches which have been made public to the wider community.
I have also made it a priority to increase the connections between different analysts within the
CMS long-lived group as well as between analysts and hardware, reconstruction 
and simulation experts within the CMS collaboration in order to find common solutions to 
common problems for long-lived analyses and to encourage the
adoption of new search strategies. To this end, I have directly engaged with
trigger, reconstruction and detector experts, including presenting new ideas at
meetings and workshops. This has led to dedicated studies on new triggers and 
analysis techniques being actively pursued by multiple groups. 
In January 2020, with the co-convenor of the CMS EXO long-lived group, I 
organised the first CMS long-lived workshop with discussions 
of common issues for LLP searches including triggering, simulation, reconstruction, 
the use of machine learning for LLPs, and exploiting hardware upgrades. 
The main aim of the workshop was to strengthen dialogue between analysts and experts within CMS,
strengthen connections with external theorists, and to highlight new directions 
for long-lived particles in future runs of the LHC.  
The workshop was highly successful with over 100 participants and has been a starting 
point for multiple ongoing studies. My interest in long-lived physics expands beyond the scope of CMS
and milliQan and so in Febuary 2020 I joined the LHC LLP community organising committee 
and helped to organise the successful LHC LLP workshop in May 2020.

In my own research, I am expanding my activities in searching for long-lived particles with the CMS
detector through exploiting new techniques and final states. I initiated and have active involvement
in the first search for displaced hadronic decays in the CMS muon system, supervising a graduate student's
work on this analysis. In addition I am contributing to a search using machine learning to look for 
displaced decays of sterile neutrinos, investigating possible new triggers
for the upcoming run of the LHC using new handles from the CMS HCAL in collaboration with
HCAL experts, and am investigating the use of timing information 
in searching for long-lived particles at the HL-LHC.

My interest in BSM physics extends beyond long-lived signatures. 
During my PhD (from 2013 -- 2017) I worked on the \alphat analysis searching
for BSM physics in a final state containing jets and missing transverse momentum. 
The record energy reached for Run 2 of the LHC provided an excellent possibility of discovery
in the first months of operation. Taking advantage of this opportunity
required rapid and reliable analysis of this dataset. 
I held a pivotal role in ensuring the results from the \alphat analysis were among the 
first to be shown publicly with $2.3~\text{fb}^{-1}$ of data at CERN in November 2015
and with $12.9~\text{fb}^{-1}$~\cite{CMS:2016dbr} of data at ICHEP16, for which I gave the successful
approval talk. Despite being targeted at prompt signatures, this latter search also 
included an interpretation of long-lived supersymmetry 
models. My key involvement in these analyses allowed me to gain important experience 
in quickly and robustly understanding and then analysing data to search for BSM physics. 

\subsection*{Phenomenology}

As part of the MasterCode collaboration from 2012 -- 2018 I developed a framework for 
deriving constraints from direct searches for BSM physics on GUT scale models of SUSY.
This required the comprehension, implementation and validation of several 
searches from both the CMS and ATLAS experiments. 
Using this framework, I worked to show that through combining several inclusive analyses targeting 
different final states the sensitivity of the limit to the non-coloured sector of the SUSY spectra can be approximately removed. 
These `universal limits' can be used to greatly reduce the time taken to sample a GUT model parameter space. 
Through this work, I also gained experience with event generation (PYTHIA) as well as fast detector simulation (DELPHES)~\cite{deVries:2015hva}.

In addition to my work for the MasterCode collaboration I have worked on 
studies of the discovery potential for SUSY at future colliders and how 
metastable supersymmetric taus (staus), predicted in certain SUSY models to provide the observed
dark matter relic density, may be discovered by experiments at the LHC~\cite{Citron:2012fg}. 
This work on metastable staus was an early consideration of long-lived decays to taus, which
is rapidly gaining in interest within the community.

My phenomenology experience has been very useful in understanding
the kinds of models that are well motivated theoretically and evade current experimental limits and may be targeted in the 
future. I have also gained an appreciation
of the information which is needed to reliably reinterpret an analysis and so which should be released in CMS publications.
In 2017, I worked with a collaborator in CMS to make a proposal 
for additional material to be released by CMS analyses
to allow their likelihoods to be easily reinterpreted by those outside the collaboration. This was adopted by the CMS
supersymmetry and exotica groups and formed the basis of a paper that looked more generally at improving
the reinterpretation of likelihoods.

\section*{Research Plans}

My extensive experience in

\begin{itemize}
\item rapid data comprehension and analysis
\item designing new search techniques and optimising sensitivity for BSM physics signatures
\item building, designing and analysing data from a new dedicated experiment
\item phenomenology and coordination within the long-lived community
\end{itemize}

\noindent will allow me to take a leading and pivotal role in searches for BSM physics and new experiments.
As inclusive searches have failed to uncover any evidence for BSM physics, 
I believe the best opportunity for discovery will 
come from novel techniques and dedicated experiments to exploit the data produced
by continued LHC operation at 13 TeV.  I would like to continue searching for 
signatures of BSM physics using CMS data from collisions at the LHC
and pursue the construction of a full scale milliQan detector.

The supervision of students both at UCSB and other institutes has 
formed an integral part of my work on milliQan and my recent analysis activities.
I hope to continue to support the uniquely important training and development 
that such activities can provide. The ability to make vital contributions
to projects at the forefront of physics research is vital in encouraging
students to become the next generation of scientists.

Over the last three years as a Postdoctoral Scholar working on CMS and milliQan
I have seized the opportunity to take leading roles within both collaborations 
and within the wider long-lived particle community. I am eager to build on my experience to play 
a pivotal role in searches for BSM physics as the LHC moves into a new and exciting phase. 
% The position at VUB provides the perfect opportunity to
% strengthen my activities within the milliQan collaboration and make VUB a European centre
% for research and development of the full milliQan detector.
 
%\clearpage\end{CJK*}                              % if you are typesetting your resume in Chinese using CJK; the \clearpage is required for fancyhdr to work correctly with CJK, though it kills the page numbering by making \lastpage undefined

\bibliographystyle{unsrt}
\bibliography{citron_VIS} % 'citron' is the name of a BibTeX file

\end{document}


%% end of file `template.tex'.

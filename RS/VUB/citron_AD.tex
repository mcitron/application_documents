\documentclass[11pt]{article}
\usepackage{amsmath}
\usepackage{cancel}
\usepackage{xspace}
\usepackage{amssymb}
\usepackage{amsthm}
\usepackage{amscd}
\usepackage{amsfonts}
\usepackage{graphicx}%
\usepackage{fancyhdr}

% \usepackage[a4paper, total={7in, 8in}]{geometry}
\usepackage[bottom=0.4in,left=0.8in,right=0.8in]{geometry}
\usepackage{fancyhdr}

\theoremstyle{plain} \numberwithin{equation}{section}
\newtheorem{theorem}{Theorem}[section]
\newtheorem{corollary}[theorem]{Corollary}
\newtheorem{conjecture}{Conjecture}
\newtheorem{lemma}[theorem]{Lemma}
\newtheorem{proposition}[theorem]{Proposition}
\theoremstyle{definition}
\newtheorem{definition}[theorem]{Definition}
\newtheorem{finalremark}[theorem]{Final Remark}
\newtheorem{remark}[theorem]{Remark}
\newtheorem{example}[theorem]{Example}
\newtheorem{question}{Question} \topmargin-2cm

% \textwidth7in
%
% \setlength{\topmargin}{0in} \addtolength{\topmargin}{-\headheight}
% \addtolength{\topmargin}{-\headsep}
%
% \setlength{\oddsidemargin}{0in}
%\newcommand{\alphat}{\ensuremath{\alpha_{\text{T}}}\xspace}
\DeclareRobustCommand{\alphat}{$\alpha_{\text{T}}~$}
\DeclareRobustCommand{\met}{$\mbox{$E_\text{T}^{\rm miss}$}\xspace$}


% \oddsidemargin  0.0in \evensidemargin 0.0in

\pagestyle{fancy}\lhead{Matthew Citron}\rhead{May 2020}
\chead{{\large{\bf Academic Dossier}}} \lfoot{} \cfoot{\bf \thepage}

\newcounter{list}

\begin{document}
\section*{Academic Dossier}
%\section*{\fontsize{30}{30}\selectfont Statement of intent}
\noindent 
My research has focused on using data from the LHC to search for signatures of Beyond Standard Model (BSM) physics which may resolve 
fundamental problems in particle physics, such as the origin of dark matter. 
In the three years of my PhD I have taken a leading role in the \alphat 
analysis group searching for generic models of Supersymmetry (SUSY) and dark matter. 
Within the SUSY group in the CMS collaboration, we used the first data at 13 TeV from Run 2 of the 
Large Hadron Collider (LHC) to place strong constraints on a wide range of BSM physics models. 
As part of a small group, I had direct involvement in areas throughout the \alphat search. 
This has allowed me to gain wide-ranging experience in both rapid comprehension
and analysis of data from the LHC as well as understanding in how sensitivity to BSM physics signatures can be maximised
while maintaining a robust analysis. In parallel, I have gained experience in hardware though work on the 
Level One (L1) trigger and in phenomenology through my involvement with the MasterCode collaboration.

In the coming years the jumps in energies and luminosity that marked the previous running of the 
LHC will come to an end. While inclusive style analyses will continue to play a role in 
exploring ever larger phase space, targeted searches for particular models 
may provide the best chance for discovery. In the following paragraphs I describe how the experience 
gained during my PhD has prepared me well for the challenge of discovering BSM physics at the LHC.

\section*{Research activities since January 2014}

\subsection*{The milliQan experiment}

In recent years there has been increasing interest in the possibility 
that astrophysical evidence for dark matter may indicate the existence 
of not just a single new particle but an entire "dark sector"
of particles and interactions with a rich and complex structure. The milliQan experiment 
was proposed in 2014 as a dedicated scintillator-based detector to search 
for millicharged particles ($\chi$), which can arise in dark sector models. 
The detector will consist of several layers of long scintillating bars pointing towards 
the interaction point at CMS, paired with high-gain, low-noise photomultiplier tubes (PMTs) 
capable of measuring a single scintillation photon. Such a design is necessary as the 
energy deposited by a millicharged particle 
is reduced by a factor of $Q/e^2$ compared to a particle of charge
$e$ with the same mass.

I joined the milliQan collaboration in Sept 2017 and immediately joined the
team installing a 1\% scale prototype designed to study the feasibility and
develop understanding of the experiment. The demonstrator is located in 
a challenging requirement that until 2020 could only be accessed when the LHC 
beam was not operational. During the first commissioning run, 
from September 2017 to January 2018, 
I helped write the framework needed to convert raw PMT waveforms into 
a format suitable for analysis and helped to constantly monitor
the data from the detector to ensure its smooth operation. While undertaking this 
monitoring I found a sudden spike in the trigger rate of the detector and determined
that this was due to the ramping down of the CMS magnetic field. This lead to the installation
of dedicated magnetic field sensors and additional shielding of the PMTs. As the magnetic field
was shown to negatively impact the performance of certain PMT species this also had
important ramifications for the choice of PMT for the full detector. In addition, I
found critical issues in the DAQ logic that caused signal like deposits to fail
trigger selection during early running.

In March 2018 I lead the upgrade of the demonstrator, which built on the experience 
gained during the first operational run, to expand the size of the active area
sensitive to signal deposits and add new components designed to shield the signal
sensitive area and measure backgrounds. I helped design the upgraded demonstrator,
helped constructed the new PMT and scintillator components as well as 
supervising both undergraduate and graduate students at UCSB undertaking this work.
The upgraded demonstrator took data for commissioning from March -- June 2018, during which 
time several issues were found and fixed in multiple interventions, before data suitable for physics
analysis was taken from June until proton-proton collisions stopped in late October 2018. 
The prototype detector collected a data set of 37.5 $fb^{-1}$, corresponding to 86\% of the total luminosity 
deliaved by the LHC in this period.

In January 2018 I was tasked with the analysis of data from the 
milliQan demonstrator. This invovled coordinating the activities of a group of approximately
15 postdocs, graduate students and undergraduates from various institutes in the US, Europe and Asia. 
The analysis objectives included the calibration of the data from the detector,
charaterising and measuring backgrounds, and simulating the generation and propagation of 
signals and backgrounds as well as the response of the detector. With these inputs I constructed
a search for millicharged particles using the prototype detector. Despite using sub-optimal PMTs 
and being only a small fraction of the size of the full milliqan detector,
this search achieved world-leading sensitivity to fractionally charged particles and has been formed
into a paper submitted to PRD~\cite{XX}.

The background measurements and mitigation strategies that I undertook for the search
with the milliQan demonstrator lead to important lessons for the design 
of the full milliQan detector. In particular, I was able to show 
that, contary to the assumption in the original experimental proposal, random overlap of dark rate counts
is a subdominant background source compared to the contribution from shower particles generated
by cosmic muons in the cavern. To mitigate this I proposed an alteration in the design of the detector from
three to four layers. The demonstrator was been altered to a four layer configuration and data collected from 
January -- May 2019 to directly measure background rates. This will be used to provide updated
projections in an upcoming paper (expected to be submitted by Autumn 2020).

\subsection*{Search for long-lived particles}

\subsection*{Search for BSM physics}

SUSY is a leading candidate as a BSM theory to resolve problems in the Standard Model.
For SUSY to naturally predict a Higgs boson mass at $m_H \approx 125~\text{GeV}$, coloured 
SUSY particles at the TeV scale may be expected. SUSY may also offer a compelling dark matter
candidate, the lightest supersymmetric particle (LSP). The final state from the coloured SUSY 
particle decays typically contains hadronic activity (in the form of jets) as well as momentum 
imbalance (\met) from the LSP. When the mass splittings in the SUSY spectra are small, 
discovery may be particularly challenging as the energy in the final state is reduced.

The record energy reached for Run 2 of the LHC provided an excellent possibility of discovery
in the first months of operation. I worked on the \alphat analysis searching
for BSM phyiscs in a final state containing jets and \met. 
Taking advantage of this opportunity for discovery required rapid and reliable analysis of this dataset. 
I held a pivotal role in ensuring the results from the \alphat analysis were among the 
first to be shown publicly with $2.3~{fb}^{-1}$ of data at CERN in November 2015
and with $12.9~{fb}^{-1}$ of data at ICHEP16, for which I gave the successful
approval talk. My key involvement in the analysis allowed me to gain important experience 
in quickly and robustly understanding and then analysing data to search for BSM physics. 

The first task I undertook with the \alphat analysis built on my experience with the 
trigger, described below, to design a L1 trigger strategy to increase acceptance for compressed models,
which have lower energies in the final state, while maintaining a low trigger rate. 
For any analysis the trigger is critical as data lost at this stage cannot be recovered. 
Taking advantage of the new L1 jet algorithm I was able to significantly increase 
acceptance for compressed models, beneficial for a wide ranges of searches.

My main responsibility for the \alphat analysis has been the statistical interpretation
of the data collected by the search. This requires a holistic and deep understanding
of all sections of the analysis that are included in the final likelihood model. During my
PhD I have worked to completely rewrite the statistical framework and redesign the likelihood model. 
I have been instrumental in efforts to measure systematic uncertainties in data and through simulation and to
ensure their effects are robustly included in the likelihood 
model with the correct correlation scheme. This is particularly important for compressed models, which typically
contribute most significantly in systematic limited regions. This experience is extremely valuable 
for future searches for BSM physics as correctly modelling the backgrounds and their uncertainties in a systematics 
dominated environment could be crucial for discovery.

Another of my responsibilities has been the inclusion of \met~shapes into the \alphat
analysis. This was a significant change in strategy as the analysis moved from a simple `cut and
count' to a shape based analysis. While this change allows significant increase in sensitivity to a wide 
range of models, it also provides challenges to ensure the analysis is robust. I achieved this
by ensuring the modeling of the \met~shape was validated, and systematic uncertainties derived, 
using signal depleted control regions in data. In addition, systematic effects on the \met~shape from known sources,
as well as their correlation, were derived from simulation and considered within the likelihood model.

Alongside my work in the \alphat analysis I have actively contributed within the SUSY group.
As well as my work on the trigger strategy I was part of a working group that investigated
the reliability of limits for the SUSY models in the `top corridor'. In such models, the mass splitting
of the supersymmetric top (stop) and the LSP is close to the top mass. This work required an in 
depth study of the features of the stop decay and how the signal may be separated from 
background in this regime and is useful experience for future searches for such `stealth stops'.

\subsection*{Hardware}
My hardware experience has centred on the level 1 hardware trigger and its
upgrade for the start of Run 2. Initially, I contributed to the development of the emulator for the upgraded
trigger system which simulates the algorithms used in the hardware. This required a good understanding
of the operation of the algorithms in the firmware. Building on this, 
I worked on designing an algorithm to identify jets and subtract contributions from simultaneous collisions, 
pile-up (PU). This required close collaboration with firmware and reconstruction experts to ensure the algorithm was 
both viable and effective. By taking advantage of the increased granularity and flexibility of the 
upgraded system the new algorithm saw a significant increase in performance, beneficial
to all analyses on CMS involving jets. Such service work is not only an important duty for the collaboration but is also
highly useful for gaining understanding of the detector performance and
can provide important lessons for analysis. As the LHC moves to higher instantaneous 
luminosities and pile-up, effective triggering will become an ever greater challenge 
for which my experience will be highly useful. 

\subsection*{Simplified Likelihood}
Independently from my group, I worked with a collaborator in CMS to make a proposal 
for additional material to be released by CMS analyses
to allow their searches to be easily reinterpreted by those outside the collaboration. 
This work built on my experience with the statistical framework for $\alpha_T$ as well as being 
informed by my work with the MasterCode collaboration. The predictions and covariances between analysis bins may be used
to define a simplified likelihood to allow reinterpretation for any search. 
A recommendation to release this information for all analyses
will be made in the SUSY and exotica groups. In addition, I co-authored a document,
that will be made public, describing generically how this may be used to reinterpret a search.

\subsection*{Phenomenology}
As part of the MasterCode collaboration I developed a framework for 
deriving constraints from direct searches for BSM physics on GUT scale models of SUSY.
This required the comprehension, implementation and validation of several 
searches from both the CMS and ATLAS experiments. Using this framework, I worked to show that through combining several inclusive analyses targeting 
different final states the sensitivity of the limit to the non-coloured sector of the SUSY spectra can be approximately removed. 
These `universal limits' can be used to greatly reduce the time taken to sample a GUT model parameter space. 
Through this work, I also gained experience with event generation (PYTHIA) as well as fast detector simulation (DELPHES).

In addition to my work for the mastercode collaboration I have worked on 
studies of the discovery potential for SUSY at future colliders and how 
metastable supersymmetric taus (staus), predicted in certain SUSY models to provide the observed
dark matter relic density, may be discovered by experiments at the LHC. 
This work on metastable staus has been cited by both the CMS and MoEDAL collaborations.

My phenomenology experience has been very useful in understanding
the kinds of models that are well motivated theoretically and evade current experimental limits and may be targeted in the 
future, such as anomaly mediated supersymmetry breaking models (mAMSB). I have also gained an appreciation
of the information which is needed to reliably reinterpret an analysis and so which should be released in CMS publications.

\section*{Research Plans}

My extensive experience in

\begin{itemize}
\item rapid data comprehension and analysis
\item designing a search and optimising sensitivity for BSM physics signatures
\item likelihood model building and statistical analysis 
\item triggering algorithms and strategies
\item phenomenology
\end{itemize}

will allow me to take a leading and pivotal role in searches for BSM physics within a large collaboration.
I would like to continue searching for signatures of such BSM physics using CMS data from collisions at the LHC.
In addition, I would like to take a greater role in contributing directly to the detector to improve performance
as well as gaining further understanding of detector performance. As inclusive searches have failed 
to uncover any evidence for BSM physics, I believe the best opportunity for discovery will 
come from searches targeting specific models using the large datasets produced 
by continued LHC operation at 13 TeV. Such searches will require good understanding of the detector,
the ability to optimise the sensitivity of a search and a robust evaluation of the background model.

Over the last three years as a PhD student on CMS I have seized the opportunity to gain experience with real data
and work effectively as part of a large collaboration, being based at CERN much of this time. I have played 
a key role within the very successful \alphat search as well as contributing to the SUSY group as a whole.
I am eager to build on my experience in analysis and hardware to play a pivotal role in searches for BSM physics as the LHC
moves into this new and exciting phase. Santa Barbara has a strong and leading presence within the particle
physics community with major contributions in calorimetry and for searches for interesting and sensitive signature
of BSM physics models. I believe it would be the ideal place to continue contributing to the search for BSM physics.

%\clearpage\end{CJK*}                              % if you are typesetting your resume in Chinese using CJK; the \clearpage is required for fancyhdr to work correctly with CJK, though it kills the page numbering by making \lastpage undefined
\end{document}


%% end of file `template.tex'.

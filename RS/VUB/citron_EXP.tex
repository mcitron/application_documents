\documentclass[11pt]{article}
\usepackage{amsmath}
\usepackage{cancel}
\usepackage{xspace}
\usepackage{amssymb}
\usepackage{amsthm}
\usepackage{amscd}
\usepackage{amsfonts}
\usepackage{graphicx}%
\usepackage{fancyhdr}

% \usepackage[a4paper, total={7in, 8in}]{geometry}
\usepackage[bottom=0.4in,left=0.8in,right=0.8in]{geometry}
\usepackage{fancyhdr}

\theoremstyle{plain} \numberwithin{equation}{section}
\newtheorem{theorem}{Theorem}[section]
\newtheorem{corollary}[theorem]{Corollary}
\newtheorem{conjecture}{Conjecture}
\newtheorem{lemma}[theorem]{Lemma}
\newtheorem{proposition}[theorem]{Proposition}
\theoremstyle{definition}
\newtheorem{definition}[theorem]{Definition}
\newtheorem{finalremark}[theorem]{Final Remark}
\newtheorem{remark}[theorem]{Remark}
\newtheorem{example}[theorem]{Example}
\newtheorem{question}{Question} \topmargin-2cm

% \textwidth7in
%
% \setlength{\topmargin}{0in} \addtolength{\topmargin}{-\headheight}
% \addtolength{\topmargin}{-\headsep}
%
% \setlength{\oddsidemargin}{0in}
%\newcommand{\alphat}{\ensuremath{\alpha_{\text{T}}}\xspace}
\DeclareRobustCommand{\alphat}{$\alpha_{\text{T}}~$}
\DeclareRobustCommand{\met}{$\mbox{$E_\text{T}^{\rm miss}$}\xspace$}


% \oddsidemargin  0.0in \evensidemargin 0.0in

\pagestyle{fancy}\lhead{Matthew Citron}\rhead{May 2020}
\chead{{\large{\bf Reasons for candidature}}} \lfoot{} \cfoot{\bf \thepage}

\newcounter{list}

\begin{document}
\section*{Reasons for candidature}
%\section*{\fontsize{30}{30}\selectfont Statement of intent}
\noindent 

The milliQan experiment is a dedicated scintillator-based detector
proposed to search for millicharged particles produced in collisions
at the LHC. I believe that I would be well suited to complementing the activities of the 
elementary particle physics group at Vrije Universiteit Brussel (VUB). 
The position at VUB provides the perfect opportunity to
strengthen my activities within the milliQan collaboration and make VUB a European centre
for research and development of the full milliQan detector.

\subsection*{The milliQan detector}

The proposed milliQan detector is comprised of a scintillator array placed
approximately 30m from the CMS interaction point (IP), 
with 17m of rock providing shielding from collisional backgrounds. 
Fractionally charged particles deposit only $(Q/e)^2$ of the energy 
deposited of a particle with charge $e$ with 
the same mass and therefore a large path length of scintillator is required for
detection. The milliQan detector is proposed as two 
$1~\mathrm{m}\times1~\mathrm{m}\times3~\mathrm{m}$ plastic scintillator arrays 
positioned next to each other and each aligned with the CMS IP.
These arrays are formed of four layers of 216 scintillator "bars" 
(864 bars per array) optically coupled to high-gain PMTs. The layers are 
separated by lead bricks to reduce correlated backgrounds 
between layers. Each array is surrounded by six 
scintillator ``panels" that serve as an active veto of muons produced 
by cosmic rays and their shower particles. Finally, at the top and bottom of the array
scintillator ``slabs" serve to identify muons produced at the CMS IP which
reach the detector. In order to control backgrounds, a hit in a bar in each of the four layers 
of the scintillator array within a small time 
window and in a path pointing at the CMS IP will be required. 
The highly modular design of the milliQan detector has allowed the feasibility 
to be confirmed through the operation of a 1\% prototype detector at the LHC.

The milliQan collaboration includes physicists from around ten institutes in the US, 
Europe and Asia. Dr. Steven Lowette, a faculty member of the elementary particle 
physics group at VUB, has been a member of the collaboration since
2019.  

\subsection*{Main activities at VUB}

The milliQan detector requires the assembly of a total of 1728 bars as well
as the slabs and panels. Each scintialltor volume 
must be optically coupled to a PMT+base, wrapped with reflective
material and light-tight black taping. 
Finally, after assembly each bar has to be checked for light leaks and calibrated 
before they can be sent to CERN for final assembly of the detector.
Partly, my activities at VUB would involve providing expertise to allow 
the construction and calibration of the detector components, 
however, the main purpose for my application is
to pursue several studies that are crucial to finalising the design and optimising the performance 
of the milliQan experiment. This would be carried out in collaboration with Dr Lowette and
will form a significant part of thesis work for undergraduates and graduates at VUB.

The active veto potential for the full detector must be studied in depth.
This is of vital importance to achieve sensitivity to low charge signals, which face large
backgrounds from correlated cosmic muon shower particles and radiation within the cavern. 
The panels surrounding the detector are too thin to efficiently identify deposits from such particles
and so the veto performance of bars on the outer edges of the detector must be extensively 
studied. It may also be possible to improve the light collection efficiency of the panels
using light guides or multiple PMT readouts such that lower energy deposits may be seen. 
In combination this will dramatically reduce the background for signals with $Q \sim 0.001e$.
The active veto can be studied using scintillator bars and panels both on the surface at VUB and
CERN, and underground at the detector position.

Higher charge signals ($Q \gtrsim 0.05$) face lower backgrounds from cavern processes,
however, distinguishing the signal from 
beam muons is difficult as both signatures will saturate the readout of the DAQ. 
The slabs that will be placed at the top and bottom of each milliQan array
are therefore critical to discriminate higher charge signals from background.
The design of these slabs has not been finalised and would form another interesting
and important project. A practical solution is to use four scintillator volumes
each covering a quarter of the surface area. However, issues including eliminating
edge spacing between the slabs, the required thickness and strategies for coupling
PMTs have not been finalised. It is foreseen that the design and construction of
these slabs will be driven by VUB.

It will be critical to calibrate the energy response and timing of the bars 
to identify signal deposits and effectively veto backgrounds. 
For the milliQan prototype this calibration was achieved using cosmic muons 
and their shower particles, however, the spread in position and energy of these processes
entailed large uncertainties. A radioactive source, such as Na22, would allow much greater control 
of these source properties and allow a higher precision calibration, crucial to efficiently 
rejecting backgrounds. This calibration will be undertaken by VUB students at CERN under supervision.

% \subsection*{Contingencies and other overlap with VUB}

It is possible that funding will not be available to construct the full milliQan detector in
time for the next run of the LHC, due to begin in May 2021. In this case a smaller 
detector may be constructed using available Hamamatsu R878 PMTs from a previous experiment. 
Such a detector can be approximately six times
larger than the previous prototype and collect at least five times the data, allowing for
significantly improved limits. The noise levels of the R878 PMTs 
are too high to allow triggering on single PE signals, however, from studies with the prototype
have shown it is possible to reject this noise using a simple low pass filter. The performance of
such a contingency experiment can be dramatically improved through the development of a
filter and amplifier applied to the PMT output. This can be designed and studied at VUB and
would be a clear opportunity for an undergraduate research project.

Alternative locations of the milliQan detector, either in the forward region or at a neutrino source 
at DUNE or J-PARC also provides an opportunity for research at VUB, regardless of medium-term funding availability. 
In this case the production cross section for lower masses would be greatly enhanced.

In addition to shared interests in milliQan, my work on long-lived and exotic signatures
at CMS overlaps with the research topics of Dr Steven Lowette and Dr Freya Blekman at VUB. 
While not being the main purpose of the position, I believe there would be ample opportunity
to build new collaborations for searches with the CMS detector. One possible example
would be the development of new dedicated triggers for fractionally 
charged particles to carry out a search using the muon system, with possible 
sensitivity to higher mass particles with $Q \gtrsim 0.1$, with obvious complementarity 
to the targets of the milliQan detector.

\subsection*{Funding}

I would plan to take part in funding proposals from VUB for the full milliQan detector
from the FWO funding agency. I have had experience contributing to a National Science
Foundation funding proposal for milliQan. I believe my extensive 
experience with both building detector components and coordinating 
a search for millicharged particles would strengthen the technical
expertise of future proposals.

\subsection*{Timeline}

The next run of the LHC is expected to begin in May 2021, however, only a small $20fb^{-1}$ dataset is 
expected to be collected. Therefore a clear target for the completion of the detector
is the beginning of 2022. Approximately six months is estimated for the construction with another six months
estimated for commissioning. This provides an ideal timeline for the visiting position at VUB as 
the studies outlined above can largely be undertaken before and during the construction of the detector.
The dedicated calibrations can take place during the commissioning phase. These studies will also be
required under the contingency scenario of a smaller detector with approximately the same timeline.

\end{document}


%% end of file `template.tex'.


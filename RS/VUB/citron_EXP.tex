\documentclass[11pt]{article}
\usepackage{amsmath}
\usepackage{cancel}
\usepackage{xspace}
\usepackage{amssymb}
\usepackage{amsthm}
\usepackage{amscd}
\usepackage{amsfonts}
\usepackage{graphicx}%
\usepackage{fancyhdr}

% \usepackage[a4paper, total={7in, 8in}]{geometry}
\usepackage[bottom=0.4in,left=0.8in,right=0.8in]{geometry}
\usepackage{fancyhdr}

\theoremstyle{plain} \numberwithin{equation}{section}
\newtheorem{theorem}{Theorem}[section]
\newtheorem{corollary}[theorem]{Corollary}
\newtheorem{conjecture}{Conjecture}
\newtheorem{lemma}[theorem]{Lemma}
\newtheorem{proposition}[theorem]{Proposition}
\theoremstyle{definition}
\newtheorem{definition}[theorem]{Definition}
\newtheorem{finalremark}[theorem]{Final Remark}
\newtheorem{remark}[theorem]{Remark}
\newtheorem{example}[theorem]{Example}
\newtheorem{question}{Question} \topmargin-2cm

% \textwidth7in
%
% \setlength{\topmargin}{0in} \addtolength{\topmargin}{-\headheight}
% \addtolength{\topmargin}{-\headsep}
%
% \setlength{\oddsidemargin}{0in}
%\newcommand{\alphat}{\ensuremath{\alpha_{\text{T}}}\xspace}
\DeclareRobustCommand{\alphat}{$\alpha_{\text{T}}~$}
\DeclareRobustCommand{\met}{$\mbox{$E_\text{T}^{\rm miss}$}\xspace$}


% \oddsidemargin  0.0in \evensidemargin 0.0in

\pagestyle{fancy}\lhead{Matthew Citron}\rhead{May 2020}
\chead{{\large{\bf Reasons for candidature}}} \lfoot{} \cfoot{\bf \thepage}

\newcounter{list}

\begin{document}
\section*{Reasons for candidature}
%\section*{\fontsize{30}{30}\selectfont Statement of intent}
\noindent 

The milliQan experiment is a dedicated scintillator-based detector
proposed to search for millicharged particles produced in collisions
at the LHC. The main thrust of my proposed activities at VUB 
would be to bring my experience in supervising the construction of the detector components
to support VUB in becoming a major contributor to the milliQan experiment.

\subsection*{The milliQan detector}

The proposed milliQan detector is comprised of a scintillator array placed in
the CMS drainage gallery approximately 30m from the CMS interaction point (IP), 
with 17m of rock providing shielding from collisional backgrounds. Fractionally charged particles 
deposit only $(Q/e)^2$ of the energy deposited of a particle with charge $e$ with 
the same mass and therefore a large path length of scintillator is required for
detection. The milliQan detector is proposed as two 
$1~\mathrm{m}\times1~\mathrm{m}\times3~\mathrm{m}$ plastic scintillator arrays 
positioned next to each other and each aligned with the CMS IP. The arrays are 
subdivided into 54 "modules" held in place by a mechanical cage on
a rotatable mechanical structure. Each module contains 4 longitudinal ``layers"
of $2\times2$ arrays of $5~\mathrm{cm}\times5~\mathrm{cm}\times60~\mathrm{cm}$ 
volume scintillator ``bars" optically coupled to high-gain PMTs. The 
proposed design is shown in Fig.~\ref{fig:detmodel}. The layers are 
separated by 5~cm lead bricks to reduce correlated backgrounds 
between layers. There are thus a total of 864 ($9\times6\times4\times4$) bars 
in each of the 2 arrays. Each array is surrounded by six 0.5 
inch thick scintillator ``panels" that serve as an active veto of muons produced 
by cosmic rays and their shower particles. Finally, at the top and bottom of the array
XX inch thick scintillator ``slabs" serve to identify muons produced at the CMS IP which
reach the detector. 

% This prototype has also provided important lessons for design of the full detector, motivating
% a four layer array to control backgrounds. As described in Sec.~\ref{sec:},
% the design can be easily altered to meet funding constraints or to target different
% millicharged particle mass ranges.

Each scintillator bar is expected to produce an average of $\mathcal{O}(1)$ photoelectron (PE) 
from each attached PMT for a millicharged particle with $Q=\mathcal{O}(10^{-3})~e$ that traverses 
our 60~cm plastic scintillator. In order to control backgrounds, a hit in
a bar in each of the four layers of the scintillator array within a small time 
window and in a path pointing at the CMS IP will be required. For charges around 
$Q\sim(10^{-3})~e$ the background is expected to be $\mathcal{O}(10)$ events per year while for higher
charges, $Q\gtrsim 10^{-2}~e$, additional requirements on the size of the response in the bars and hits in the slabs
allow a total background of less than one event. The highly modular design of the milliQan detector has allowed the feasibility 
to be confirmed through the operation of a 1\% prototype detector at the LHC.

The milliQan collaboration includes physicists from around ten institutes in the US, 
Europe and Asia. Dr Steven Lowette at VUB has been a member of the collaboration since
2019.  

\subsection*{Main activities at VUB}

The bulk of the construction will be the 1728 bars needed for the detector. 
Each assembly consists of optically coupling a PMT+base to the scintillator, 
wrapping with reflective material, and light-tight black taping. 
After assembly each bar has to be checked for light leaks and calibrated. 
Tested bars will be shipped to LHC Point 5 at CERN, where installation into the module 
cages will occur. For the milliQan prototype this construction
and calibration was undertaken at UCSB by undergraduates under my supervision.
As part of this position, I would come to VUB to assist in construction and
provide training to students to undertake this work, however, the
the main propose of the visiting position would be to pursue several
studies that are crucial to finalising the design and optimising the performance 
of the milliQan experiment. This could be carried out in collaboration with Dr Lowette and
could could form a significant part of thesis work for undergraduates and graduates at VUB.

One such study would be the active veto potential for the full detector which
is of vital important for the sensitivity reach for low charge signals as these face large
backgrounds from correlated cosmic muon shower particles and radiation within the cavern. 
The panels surrounding the detector are too thin to efficiently identify deposits from such particles
and so the veto performance of bars on the outer edges of the detector must be extensively 
studied. It may also be possible to improve the light collection efficiency of the panels
using light guides or multiple PMT readouts such that lower energy deposits may be seen. 
In combination this could dramatically reduce the background for signals with $Q \sim 0.001e$.
The active veto can be studied both on the surface at VUB and
CERN and underground at the detector position.

For higher charge signals ($Q \gtrsim 0.05$) it challenging to distinguish the 
signal from beam muons as the amplification required for the milliQan bars to
be able to detect single photons causes such signals to saturate the readout 
of the DAQ. The slabs that will be placed at the top and bottom of each milliQan array
are therefore critical to identifying the charge of signals which saturate the bars.
The design of these slabs has not been finalised and would form another interesting
project. A practical solution is to use four scintillator volumes
each covering a quarter of the surface area. However, issues including eliminating
edge spacing between the slabs, the required thickness and strategies for coupling
PMTs have not been finalised. It is foreseen that the design and construction of
these slabs could be driven by VUB.

Once the detector has been constructed it will be critical to calibrate the 
energy response and timing of the bars to identify signal and effectively veto backgrounds. 
With the prototype this was achieved using
cosmic muons and their shower particles, however, this entails large
uncertainties. A radioactive source, such as Na22, would allow much greater control 
on the position and energy deposits of the particles used to calibrate these crucial
quantities. This calibration could be undertaken by VUB students at CERN under supervision.

% \subsection*{Contingencies and other overlap with VUB}

It is possible that funding will not be available to construct the full milliqan detector in
time for the next run of the LHC, due to begin in XX. In this case a smaller detector may be constructed using
available PMTs taken from a previous experiment. Such a detector could be approximately six times
larger than the previous prototype and could collect at least five times the data, allowing for
signifcantly improved limits. The noise levels of the 878 PMTs 
are too high to allow triggering on single PE signals, however, from studies with the prototype
it has been possible to reject this noise using a simple low pass filter. The performance of
such a contingency experiment could be dramatically improved through the development of a
filter and amplifier applied to the PMT output. This could be studied and developed by VUB.

In addition to shared interests in milliQan, my work on long-lived and exotic signatures
overlaps with the research topics of Dr Lowette and Prof Blekman at VUB. While not
being the main purpose of the position, I believe there would be amply opportunity
to build new collaborations for searches with the CMS detector. One particular example 
could include the development of new dedicated triggers for fractionally 
charged particles to allow a search using the muon system with possible 
sensitivity to $Q> 0.1$, with obvious complementarity to work with the milliQan detector.

Alternative locations for milliQan as well as potential sensitivity to other signatures
are also of great interest and could be investigated through the position at VUB.

\subsection*{Funding}

I would plan to take part in funding proposals from VUB for the full milliQan detector. 
As I have extensive experience with both building detector components and coordinating 
a search for millicharged particles, I believe this would signifcantly strengthen the technical
expertise of proposals.

\subsection*{Timeline}

The next Run of the LHC is expected to begin in May 2021, however, only a small $20fb^{-1}$ dataset is 
expected to be collected. Therefore a clear target for the completion of the detector
is the begining of 2022. Approximately six months is estimated for the construction with another size months
estimated for commissioning. This provides an ideal timeline for the visiting position at VUB as 
the studies outined above can largely be undertaken before and during the construction of the detector.
The dedicated calibrations can take place during the commisioning phase. These studies will also be
required under the contingency scenario of a smaller detector (outlined above) with approximately the same timeline.

\end{document}


%% end of file `template.tex'.


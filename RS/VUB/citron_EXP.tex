\documentclass[11pt]{article}
\usepackage{amsmath}
\usepackage{cancel}
\usepackage{xspace}
\usepackage{amssymb}
\usepackage{amsthm}
\usepackage{amscd}
\usepackage{amsfonts}
\usepackage{graphicx}%
\usepackage{fancyhdr}

% \usepackage[a4paper, total={7in, 8in}]{geometry}
\usepackage[bottom=0.4in,left=0.8in,right=0.8in]{geometry}
\usepackage{fancyhdr}

\theoremstyle{plain} \numberwithin{equation}{section}
\newtheorem{theorem}{Theorem}[section]
\newtheorem{corollary}[theorem]{Corollary}
\newtheorem{conjecture}{Conjecture}
\newtheorem{lemma}[theorem]{Lemma}
\newtheorem{proposition}[theorem]{Proposition}
\theoremstyle{definition}
\newtheorem{definition}[theorem]{Definition}
\newtheorem{finalremark}[theorem]{Final Remark}
\newtheorem{remark}[theorem]{Remark}
\newtheorem{example}[theorem]{Example}
\newtheorem{question}{Question} \topmargin-2cm

% \textwidth7in
%
% \setlength{\topmargin}{0in} \addtolength{\topmargin}{-\headheight}
% \addtolength{\topmargin}{-\headsep}
%
% \setlength{\oddsidemargin}{0in}
%\newcommand{\alphat}{\ensuremath{\alpha_{\text{T}}}\xspace}
\DeclareRobustCommand{\alphat}{$\alpha_{\text{T}}~$}
\DeclareRobustCommand{\met}{$\mbox{$E_\text{T}^{\rm miss}$}\xspace$}


% \oddsidemargin  0.0in \evensidemargin 0.0in

\pagestyle{fancy}\lhead{Matthew Citron}\rhead{May 2020}
\chead{{\large{\bf Reasons for candidature}}} \lfoot{} \cfoot{\bf \thepage}

\newcounter{list}

\begin{document}
\section*{Reasons for candidature}
%\section*{\fontsize{30}{30}\selectfont Statement of intent}
\noindent 

I believe that I am well suited to complementing the activities of the 
elementary particle physics group at Vrije Universiteit Brussel (VUB). 
The position at VUB provides the perfect opportunity over the next three years 
to strengthen my leadership role within the milliQan collaboration as we move 
towards construction of the milliQan detector, increase my opportunities for 
teaching and supervision, and make VUB a European centre
for research and development of the milliQan detector.

\subsection*{The milliQan detector}

The milliQan detector is a small dedicated detector 
designed to search for fractionally 
charged particles produced in the high energy proton-proton collisions of the LHC. 
Such particles deposit only $(Q/e)^2$ of the energy 
deposited by a particle with charge $e$ of
the same mass so a large path length of scintillator is required for
their detection. The milliQan detector is proposed as two 
$1~\mathrm{m}\times1~\mathrm{m}\times3~\mathrm{m}$ plastic scintillator arrays 
in the CMS experimental cavern that are aligned with the proton-proton 
interaction point, where the LHC beams are brought together for collisions. 
The arrays are formed of four layers of 216 scintillator ``bars" 
(864 bars per array) optically coupled to high-gain photo-multiplier tubes (PMTs). To provide 
sensitivity to the small energy deposits from fractional charges as low as $0.001 e$, 
each PMT must be capable of detecting a single scintillation photon. Each array is surrounded by six 
scintillator ``panels" that serve as an active veto of muons produced 
by cosmic rays. Finally, at the top and bottom of the array
scintillator ``slabs" serve to identify muons produced in LHC collisions. 
The detector is designed such that backgrounds can be greatly mitigated
by requiring a hit in a bar in each of the four layers 
of the scintillator array within a small time 
window. The highly modular design of the milliQan detector 
has allowed the feasibility to be confirmed through the operation 
of a 1\% prototype detector at the LHC.

The milliQan collaboration includes 38 physicists from 10 institutes in the US, 
Europe and Asia. Dr. Steven Lowette, a faculty member of the elementary particle 
physics group at VUB, has been a member of the collaboration since
2019. The reasonably small size of the collaboration means that a single institute
can have a significant impact on the design and performance of the detector. 
With the visiting position at VUB, the research topics outlined below
will allow Dr Lowette and I to drive 
the next phase of the milliQan experiment. In addition,
there is significant scope for students to have vital contributions to 
achieving the physics goals of the experiment.  

\subsection*{Main activities at VUB}

The milliQan detector requires the assembly of a total of 1728 bars as well
as the slabs and panels. Partly, my activities at VUB will involve 
providing expertise to allow the construction and calibration of the detector components, 
however, the main purpose for my application is
to lead research projects that will allow VUB to make crucial contributions to
the design and performance of the milliQan experiment. 
This will be carried out in collaboration with Dr Lowette and
can form a significant part of the thesis work for undergraduate and graduate
students at VUB.

The potential for the full detector to actively veto backgrounds from cavern radiation
and cosmic muon shower particles is of vital importance to 
achieve sensitivity to signals with $Q \lesssim 0.05 e$. At VUB, I plan to 
evaluate the efficiency to identify backgrounds using 
bars on the outer edges of the detector. If a student is available, it will also
be possible to study possibilities for improving the light collection efficiency of the panels 
with light guides or multiple PMT readouts. These studies can be undertaken 
both on the surface at VUB and CERN, and underground at the detector position.

Higher charge signals ($Q \gtrsim 0.05 e$) face lower backgrounds from cavern processes,
however, distinguishing the signal from beam muons is difficult as both 
signatures will saturate the readout. The slabs that will be placed at the top and bottom 
of each milliQan array are therefore critical to discriminate higher charge signals from background.
With this position I will be able to finalise the design of these crucial detector components such that the 
design and construction of the slabs will be driven by VUB.

It will be critical to calibrate the energy response and timing of the bars. 
This calibration could be achieved using cosmic muons and their shower particles, 
however, the spread in position and energy of these processes
leads to large uncertainties. A radioactive source, such as Na22, would allow much greater control 
of these source properties and allow a higher precision calibration, crucial to efficiently 
rejecting backgrounds. With a student, I can design a source based calibration scheme 
at VUB that will be applied to the detector in the experimental cavern after construction.

% \subsection*{Contingencies and other overlap with VUB}
It is possible that funding will not be available to construct the full milliQan detector in
time for the next run of the LHC. In this case a smaller 
detector may be constructed using available Hamamatsu R878 PMTs from a previous experiment. 
Such a detector can be approximately six times
larger than the previous prototype and collect at least five times the data, allowing for
significantly improved limits. The noise levels of the R878 PMTs are too high to efficiently 
trigger on the low energy signatures of particles of $Q \lesssim 0.005 e$, however, studies with the prototype
have shown it is possible to reject this noise using a simple filter. The performance of
such a contingency experiment can be dramatically improved through the development of a
filter and amplifier that I will design and study at VUB. This is a clear opportunity 
for an undergraduate research project.

Alternative locations of the milliQan detector, either in the forward region or at a neutrino source 
at FNAL or J-PARC also provides an opportunity for research at VUB, 
regardless of medium-term funding availability. In this case, the production rate 
for lower masses is greatly enhanced.

In addition to shared interests in milliQan, my work on long-lived and exotic signatures
at CMS overlaps with the research topics of Dr Lowette and Dr Blekman at VUB. 
While not being the main purpose of the position, I believe there will be ample opportunity
to build new collaborations for searches with the CMS detector. One possible example
is the development of new dedicated triggers for fractionally 
charged particles to carry out a search using the muon system, having potential
sensitivity to higher mass particles with $Q \gtrsim 0.1 e$, with obvious complementarity 
to the targets of the milliQan detector.

\subsection*{Funding}

I plan to take part in funding proposals from VUB for the full milliQan detector
from the FWO funding agency. I have had experience contributing to a National Science
Foundation funding proposal for milliQan. I believe my extensive 
experience with both building detector components and coordinating 
a search for millicharged particles will strengthen the technical
expertise of future proposals.

\subsection*{Timeline}

The next run of the LHC is expected to begin in May 2021, however, only a small $20~\text{fb}^{-1}$ dataset is 
expected to be collected. Therefore a clear target for the completion of the detector
is the beginning of 2022. Based on our experience constructing the initial prototype, 
approximately six months is estimated for the construction and a further six months is
estimated for commissioning. This provides an ideal timeline for the visiting position at VUB as 
the studies outlined above can largely be undertaken before and during the construction of the detector.
The dedicated calibrations can take place during the commissioning phase. These studies will also be
required under the contingency scenario of a smaller detector with approximately the same timeline.
In either scenario the detector would be expected to collect $100~\text{fb}^{-1}$ in 2022 and a further $100~\text{fb}^{-1}$ in 2023.
Given the experience gained from the prototype, I believe a publication with the data set collected 
in 2022 could be achieved by early to mid 2023 and a second publication including data collected in both years in
early 2024. The milliQan detector can then be upgraded during next the long shutdown (2024--2027), with the upgraded
detector taking data at the High Luminosity LHC with a significantly increased instantaneous luminosity from 2027. 


\end{document}


%% end of file `template.tex'.


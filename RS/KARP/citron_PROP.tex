\documentclass[11pt]{article}
\usepackage{amsmath}
\usepackage{cancel}
\usepackage{xspace}
\usepackage{amssymb}
\usepackage{amsthm}
\usepackage{amscd}
\usepackage{amsfonts}
\usepackage{graphicx}%
\usepackage{fancyhdr}

% \usepackage[a4paper, total={7in, 8in}]{geometry}
\usepackage[bottom=0.4in,left=0.8in,right=0.8in]{geometry}
\usepackage{fancyhdr}

\theoremstyle{plain} \numberwithin{equation}{section}
\newtheorem{theorem}{Theorem}[section]
\newtheorem{corollary}[theorem]{Corollary}
\newtheorem{conjecture}{Conjecture}
\newtheorem{lemma}[theorem]{Lemma}
\newtheorem{proposition}[theorem]{Proposition}
\theoremstyle{definition}
\newtheorem{definition}[theorem]{Definition}
\newtheorem{finalremark}[theorem]{Final Remark}
\newtheorem{remark}[theorem]{Remark}
\newtheorem{example}[theorem]{Example}
\newtheorem{question}{Question} \topmargin-2cm

% \textwidth7in
%
% \setlength{\topmargin}{0in} \addtolength{\topmargin}{-\headheight}
% \addtolength{\topmargin}{-\headsep}
%
% \setlength{\oddsidemargin}{0in}
%\newcommand{\alphat}{\ensuremath{\alpha_{\text{T}}}\xspace}
\DeclareRobustCommand{\alphat}{$\alpha_{\text{T}}~$}
\DeclareRobustCommand{\met}{$\mbox{$E_\text{T}^{\rm miss}$}\xspace$}


% \oddsidemargin  0.0in \evensidemargin 0.0in

\pagestyle{fancy}\lhead{Matthew Citron}\rhead{July 2020}
\chead{{\large{\bf Research proposal}}} \lfoot{} \cfoot{\bf \thepage}

\newcounter{list}

\begin{document}
\section*{Research proposal}
%\section*{\fontsize{30}{30}\selectfont Statement of intent}
\noindent 

% The phase-1 milliQan experiment is a unique opportunity for a 
% new small-scale detector to provide a significant extension to the search 
% program at the CERN LHC, with a large discovery potential 
% for as yet unknown particles with a small electric charge, a ``millicharge''. 
% This award would fully fund the construction of a new detector,
% unlocking new territory in the search for millicharged particles and
% potentially uncovering the link to a hidden universe.
%
% \subsection*{Motivation}

One of the greatest mysteries in particle physics is the nature of dark matter.
While astrophysical evidence that such particles make up approximately 85\% of the matter
in the universe, there has been no direct observation of dark matter particles. In recent years, 
there has been increasing interest in the idea that dark matter points to the 
existance of not just a single new particle but an entire dark sector 
with a structure that could be as rich as complex as our own. We have not been able to
observe this universe as most dark sector particles cannot interact with
visible matter. However, it is possible that a quantum connection exists between our two universes. 
If a small ``kinetic" mixing is introduced between visible and dark photons this
leads to the existance of dark particles with a small electric charge.
Such particles are extremely difficult to detect as they interact very weakly with
matter in our universe, however, their discovery would uncover the link to 
a hidden universe.

At the end of 2021, the most powerful particle collider ever built, the CERN LHC, 
will resume proton-proton collisions. Large numbers of millicharge particles may be 
produced in these collisions, however, the multi-billion dollar general purpose
detectors, CMS and ATLAS, will not have sensitivity as millicharged particle interact
too weakly to be observed. A dedicated experiment, milliQan, is required to take advantage of 
this unique opportunity for the discovery of millicharged particles at the LHC.

Over the past few years I have led the design, operation and analysis
of data from a small prototype of the milliQan detector. Using this data I have coordinated
a search for millicharge particles with world leading-sensitivity, which has been submitted 
for publication. Given this experience and the approaching next run of the LHC, I believe this is an ideal 
time to construct an upgraded detector that overcomes the severe limitations of the prototype
in order to significantly extend sensitivity to millicharge particles. A full-scale 
detector will cost around \$2 million, however, the timeline for such a detector to be funded and built
is far too long to be completed in time.  

In this proposal, based on my experience from the prototype milliQan detector, 
I outline a detailed motivation and design for the phase-1 milliQan detector, 
with a budget of less than \$50k, that can achieve a dramatic extension 
in sensitivity to millicharge particles.
In addition, this detector will provide important experience for a future larger detector
that may be built during the next shutdown of the LHC (from late 2024), if funding becomes available.

%
\subsection*{Searching for millicharge particles at the LHC}
%
The parameter space spanned by the mass and charge of millicharged particles
has multiple constraints from both direct and indirect measurements.
However, the parameter space of millicharge particle mass from 1--100~GeV remains 
largely unexplored. This is an ideal mass
range for production at the LHC, however,
fractionally charged particles deposit only $(Q/e)^2$ of the energy 
deposited by a particle with charge $e$ of
the same mass. Therefore, a dedicted experiment is needed to provide sensitivity to the signatures of 
millicharged charged particles with $Q< 0.1 e$.
This can be achieved through the use of a 
large array of scintillator bars aligned with the CMS proton-proton 
interaction point, where the LHC beams are brought together for collisions. 
In late 2017 a prototype detector was installed to characterise the backgrounds and
prove the feasibility of such an experiment.

The milliQan prototype consists of eighteen $5~\rm{cm}\times5~\rm{cm}\times80~\rm{cm}$ 
scintillator bars arranged in three layers. The prototype ran succesfully from March 2018 to May 2019, providing crucial
experience in the operation of a dedicated scintillator based detector. 
The $37.5\rm{fb}^{-1}$ of data collected by the milliQan detector was used to carry out
a search for fractionally charged particles with world-leading sensitivity.
Despite this success, the sensitivity of the prototype was severly limited by
factors that were uncovered during its operation. 

This proposal is to install a scintillator-based detector
to search for millicharge particles with world-leading sensitivity.
The design of the experiment is motivated by 
the comprehensive understanding of the backgrounds and performance gained during the
operation of the prototype.

\subsection*{The phase-1 milliQan detector}

The proposed phase-1 milliQan detector is
composed of a $0.2~\mathrm{m}\times0.2~\mathrm{m}\times3.2~\mathrm{m}$ plastic scintillator array aligned with 
the CMS interaction point. The $3.2~\rm{m}$ active path length through the scintillator determines the sensitivity 
of the detector and is therefore designed to be as large as possible while fitting the constraints of the experimental cavern.
The scintillator arrays will be comprised of 48 $5~\rm{cm}\times5~\rm{cm}\times60~\rm{cm}$ 
and 16 $5~\rm{cm}\times5~\rm{cm}\times80~\rm{cm}$ bars arranged into 4 layers of $4\times4$ scintillator bars. 
The bars with length of $80\rm{cm}$ will be taken, without cost, from the prototype. Each scintillator bar must be optically
coupled to a PMT to readout the scintillator light. The PMTs used will be 
Hammatsu R878s, available at no cost from a previous experiment. The use of multiple layers allows backgrounds to be mitigated
by requiring a hit in each layer of the array within a small time window. The prototype detector, comprised
of three layers, faced a dominant background from correlated cosmic shower and environmental radiation sources.
By expanding to four layers, this background contribution can be greatly reduced to only a few events per year.
In addition, the expansion in size of each layer, from six to sixteen bars, will significantly improve
both the signal acceptance and active veto performance of the detector. 
In addition to the bars, two "slabs" of $20~\rm{cm}\times30~\rm{cm}\times2.5\rm{cm}$ scintillator and 
six thin "panels" of $25\rm{cm}\times160\rm{cm}\times1\rm{cm}$ 
will be inserted to tag charged particles such as muons from the CMS IP, study backgrounds from radiation, 
and provide an active veto. Analog pulses from the PMTs must be read out,
digitized, and stored for offline analysis. Which can be achieved using CAEN V1743 digitizers. 
These have 16 channels, each of which provide a high sample rate (up to 3.2 GS/s) and a long readout window (1024 samples), 
suitable for digitising the waveforms generated by the PMTs. The CAEN digitisers also allow a flexible triggering
scheme to be defined that relies on the coincidence of pulses within a small time window in PMTs in different layers. This is necessary
as the rate of single photon pulses across the detector (\~50kHz) is much larger than the maximum allowed data aquisition rate of O(100Hz). 
The phase-1 milliQan detector will require a total of 73 channels (64 bars, 2 slabs and 6 panels and one channel
for the LHC clock information from the CMS experiment) requiring five CAEN V1743 digitisers. 

To provide sensitivity to the small energy deposits from fractional charges as low as $0.001 e$,
each PMT must be capable of detecting a single scintillation photon.
In operating the prototype, it was found that Hammatsu R878 PMTs have large high frequency noise components
that make triggering and reconstructed the pulses from single scintillation photons impossible.
A simple amplifier incorporating a low-pass filter has been designed at UCSB to reject this noise and
allow such signals to be detected. In combination with the upgrade to 4 layers of scintillator bars
and enlarged active veto capability, this will allow the upgraded detector to have dramatically improved sensitivity.

The background for the four layer configuration has been conservatively estimated
through special runs of the prototype to be 0.4$\times10^{-3}$ events per hour. 
To collect the full 300 $fb^{-1}$ during the next run of the LHC 
requires a trigger live time of $4.2\times 10^3$ hours, which corresponds 
to an expected total background of only 1.6 events. With the ability to efficiently trigger
and reconstruct single photon events, this will allow the sensitivity to charges down to $Q=0.003 e$ for particles of
mass $< 1$ GeV and extend the best sensitivity in charge for millicharged particles of mass $< 40$ GeV
by more than an order of magnitude.

\subsection*{Construction and operation}

The bulk of the construction will be the assembly of the 48 new bars and 6 panels.
Each assembly consists of optically coupling a PMT held in a base
to the scintillator, wrapping it with reflective material, and light-tight black
taping. One assembled each scintillator volume must be checked for light leaks and calibrated.
This work is relatively simple and, as for the prototype, can be largely undertaken
by myself and undergraduates at UCSB. Once assembled, the bars and panels
must be sent to CERN to be installed into the detector. This work would be largely 
undertaken by postdoc and graduate student members of the milliQan collaboration 
(who are funded separately from this award). The support structure has been designed and could be 
constructed by an engineer at UCSB or CERN. Once constructed, the detector will be operated
remotely by members of the milliQan collaboration. Given the experience with the
prototype, this will be carried out with high efficiency to collect physics quality
data. Similar strategies to the prototype will be used to calibrate the response and timing of the 
detector, simulate signal and background processes and analyse the data to 
search for millicharged particles.

\subsection*{Budget}

The budget for the phase-1 milliQan detector to be covered by this award corresponds to total of \$45.6k and
breaks down as follows:

\begin{itemize}
    \item EJ-200 plastic scintillator bars (48) - \$10.3k
    \item CAEN V1743 digitizers (3) - \$21k
    \item Amplifier and filter circuit (72)  - \$0.2k
    \item Scintillator panels (6) - \$3.6k
    \item HV and readout cables - \$1.5k
    \item Travel/shipping to CERN - \$4k
    \item Mechanical structure/engineering costs - \$5k
\end{itemize}

The running costs will be covered by an agreement with CERN colleagues that has 
worked well in operation of the prototype. The following items will be reused, at no 
cost, from the prototype: HV and readout cables for 32 channels, two CAEN V1743 digitizer boards,
HV supply, steel support frame, and two scintillator slabs. The PMTs that will be coupled to each scintillator volume
are available, at no cost, from a previous experiment. My salary will be paid by
Dr. David Stuart.

\subsection*{Timeline}

The next run of the LHC is expected to begin in late 2021, providing a clear target 
for the completion of the detector. Based on our experience constructing the initial prototype, 
approximately six months is estimated for the construction (mainly in assembling the bars),
and a further six months is estimated for commissioning. The detector would be expected to 
collect $100~\text{fb}^{-1}$ in each of 2022, 2023 and 2024.
Given the experience gained from the prototype, I believe an initial publication with the data set collected 
in 2021-2022 could be achieved by early to mid 2023 with further publications with enlarged datasets 
in early 2024 and 2025. The milliQan detector can then be further 
upgraded during next the long shutdown (2024--2027), with the upgraded
detector taking data at the High Luminosity LHC with a significantly increased instantaneous luminosity from 2027. 


%% end of file `template.tex'.

\bibliographystyle{unsrt}
\bibliography{citron_PROP} % 'citron' is the name of a BibTeX file


\end{document}

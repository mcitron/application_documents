\documentclass[11pt]{article}
\usepackage{amsmath}
\usepackage{cancel}
\usepackage{xspace}
\usepackage{amssymb}
\usepackage{amsthm}
\usepackage{amscd}
\usepackage{amsfonts}
\usepackage{graphicx}%
\usepackage{fancyhdr}

% \usepackage[a4paper, total={7in, 8in}]{geometry}
\usepackage[bottom=0.4in,left=0.8in,right=0.8in]{geometry}
\usepackage{fancyhdr}

\theoremstyle{plain} \numberwithin{equation}{section}
\newtheorem{theorem}{Theorem}[section]
\newtheorem{corollary}[theorem]{Corollary}
\newtheorem{conjecture}{Conjecture}
\newtheorem{lemma}[theorem]{Lemma}
\newtheorem{proposition}[theorem]{Proposition}
\theoremstyle{definition}
\newtheorem{definition}[theorem]{Definition}
\newtheorem{finalremark}[theorem]{Final Remark}
\newtheorem{remark}[theorem]{Remark}
\newtheorem{example}[theorem]{Example}
\newtheorem{question}{Question} \topmargin-2cm

% \textwidth7in
%
% \setlength{\topmargin}{0in} \addtolength{\topmargin}{-\headheight}
% \addtolength{\topmargin}{-\headsep}
%
% \setlength{\oddsidemargin}{0in}
%\newcommand{\alphat}{\ensuremath{\alpha_{\text{T}}}\xspace}
\DeclareRobustCommand{\alphat}{$\alpha_{\text{T}}~$}
\DeclareRobustCommand{\met}{$\mbox{$E_\text{T}^{\rm miss}$}\xspace$}


% \oddsidemargin  0.0in \evensidemargin 0.0in

\pagestyle{fancy}\lhead{Matthew Citron}\rhead{July 2020}
\chead{{\large{\bf Research proposal}}} \lfoot{} \cfoot{\bf \thepage}

\newcounter{list}

\begin{document}
\section*{Research proposal}
%\section*{\fontsize{30}{30}\selectfont Statement of intent}
\noindent 

Intro/summary
\subsection*{Overview}

\subsection*{Motivation}

In recent years there has been increasing interest in the possibility 
that astrophysical evidence for dark matter may indicate the existence 
of not just a single new particle but an entire dark sector
of particles and interactions with a rich and complex structure. The milliQan experiment 
was proposed in 2014 as a dedicated scintillator-based detector to search 
for millicharged particles that can arise in dark sector models. 


\subsection*{The milliQan detector}

The milliQan detector is a dedicated detector 
designed to search for fractionally 
charged particles produced in the high energy proton-proton collisions of the LHC. 
Such particles deposit only $(Q/e)^2$ of the energy 
deposited by a particle with charge $e$ of
the same mass so a large path length of scintillator is required for
their detection. The milliQan detector is proposed as a large plastic
scintillator array in CMS experimental cavern. The array is comprised of multiple layers
of scintillator ``bars", each optically coupled to a high-gain photo-multiplier tube.
To provide sensitivity to the small energy deposits from fractional charges as low as $0.001 e$, 
each PMT must be capable of detecting a single scintillation photon. 
The detector is designed such that backgrounds can be greatly mitigated
by requiring a hit in a bar in each of the four layers 
of the scintillator array within a small time 
window. The highly modular design of the milliQan detector 
has allowed the feasibility to be confirmed through the operation 
of a small prototype detector at the LHC.

The milliQan prototype, shown in Figure~\ref{fig:demo}, consists of eighteen 
$5\times5\times80$ cm scintillator bars arranged in three layers 
of $2\times3$ scintillator plus PMT units. 
In addition to the bars, ``slabs" of $2.5 \times 20 \times 30$ cm scintillator 
and thin panels of $1 \times 18 \times 100$ cm scintillator are inserted 
to tag charged particles such as muons from the CMS IP, 
study backgrounds from radiation, and to simulate the active veto of a larger detector.
The prototype ran succesfully from March 2018 to May 2019, providing crucial
experience in the operation of a dedicated scintillator based detector. The data collected
also allowed the response of the dector to be calibrated and backgrounds to 
be characterised. Despite using sub-optimal PMTs 
and being only a small fraction of the size of the full milliQan detector,
this search achieved world-leading sensitivity to fractionally charged particles and forms
a paper submitted to PRD~\cite{ball2020search}. The purpose of this proposal is to build on
the experience gained from the operation of the prototype to design and operate a new detector
with enhanced sensitivty to millicharged particles.

The milliQan collaboration includes 38 physicists from 10 institutes in the US, 
Europe and Asia. I have been a member of the milliQan collaboration since Sept 2017 and immediately joined the
team installing the prototype designed to study the feasibility and
develop understanding of the experiment. I led the upgrade of the prototype, which built on the experience 
gained during the first operational run, to expand the size of the active area
sensitive to signal deposits and add new components designed to provide
active vetos for the signal volume and measure backgrounds. Since January 2018 
I have been responsible for coordinating the data analysis for the milliQan experiment.
The analysis objectives included the calibration of the data from the detector,
characterising and measuring backgrounds, and simulating the generation and propagation of 
signals and backgrounds as well as the response of the detector.

\subsection*{Upgraded prototype}

This proposal is install a phase 2 prototype of the milliQan experiment with dramatically improved sensitivity, 
particularly for smaller charges. TThe design of the experiment is motivated by 
the comprehensive understanding of the backgrounds and performance of the phase 1 prototype. 
The proposed upgrade is composed of a $0.2~\mathrm{m}\times0.2~\mathrm{m}\times3~\mathrm{m}$ plastic scintillator array aligned with 
the CMS interaction point. The scintillator arrays will be comprised of 64 $5~\rm{cm}\times5~\rm{cm}\times60cm$ 
bars arranged into 4 layers of $4\times4$ scintillator plus PMT units. The PMTs used will be Hammatsu R878s, 
available without cost from a previous experiment. 
The use of 4 layers (rather than the 3 used for the phase 1 prototype)
will dramatically reduce the background contribution from correlated cosmic shower and environmental radiation sources 
that limited the sensitivity of the prototype detector. In addition, the expansion in size of each layer,
from six to sixteen bars, will significantly improve both the signal acceptance and active veto performance of the detector. 
In addition to the bars, two "slabs" of $20~\rm{cm}\times30~\rm{cm}\times2.5cm$ scintillator and 
six thin "panels" of $50\rm{cm}\times150\rm{cm}\times{1}$ 
will be inserted to tag charged particles such as muons from the CMS IP, study backgrounds from radiation, 
and provide an active veto. Analog pulses from the PMTs must be read out,
digitized, and stored for offline analysis. Which can be achieved using CAEN V1743 digitizers. 
These have 16 channels, each of which provide a high sample rate (up to 3.2 GS/s) and a long readout window (1024 samples), 
suitable for digitising the waveforms generated by the PMTs. The CAEN digitisers also allow a flexible triggering
scheme to be defined that relies on the coincidence of pulses within a small time window in PMTs in different layers. This is necessary
as the rate of SPE pulses across the detector (\~50kHz) is much larger than the maximum allowed data aquisition rate of O(100Hz). 
The phase 2 prototype will require a total of 73 channels (64 bars, 2 slabs and 6 panels and one channel
for the LHC clock information from the CMS experiment) requiring five CAEN V1743 digitisers. 
In operating the phase 1 prototype, it was found that the R878 PMTs have large high frequency noise components
that cause too large a rate to efficiently trigger on single photoelectron pulses. A
simple amplifier incorporating a low-pass filter has been designed at UCSB to reject this noise and
allow such signals to be triggered. In combination with the update to 4 layers of scintillator bars
and enlarged active veto capability, this will allow the upgraded detector to have dramatically improved sensitivity.

The overall background for the four layer configuration is 
estimated to be around 1E-3 events per hour. The full 300 $fb^{-1}$ 
collected during the next run of the LHC corresponds to a trigger live time of $4.2\times 10^3$ hours, which
corresponds to a total background of around 3 events. With the ability to efficiently trigger
and reconstruct SPE events, this will allow sensitivity to charges down to $Q=0.003 e$ for particles of
mass < 1GeV and extend the best sensitivity for millicharged particles of mass $< 40$ GeV more than
an order of magnitude lower in charge.

\subsection*{Construction and operation}

The bulk of the construction will be the assembly of the 64 bars and 6 panels.
Each assembly consists of optically coupling a PMT held in a base
to the scintillator, wrapping it with reflective material, and light-tight black
taping. One assembled each scintillator volume must be checked for light leaks and calibrated.
This work is relatively simple and for the phase 1 prototype was largely 
done by undergraduates at UCSB. Once assembled, the bars and panels
must be sent to CERN to be installed into the detector. This work would be largely 
undertaken by postdoc and graduate student members of the milliQan collaboration 
(who are funded separately from this award). Given the experience with the phase 1 
prototype, the detector will be operated with high efficiency to collect physics quality
data. Similar strategies will be used to calibrate the response and timing of the 
detector, simulate signal and background processes and analyse the data to 
search for millicharged particles. These
are documented fully in Ref.~\cite{ball2020search}.

\subsection*{Budget}

The budget for the phase 2 prototype to be covered by this award corresponds to total of \$43.3k and
breaks down as follows:

\begin{itemize}
    \item EJ-200 plastic scintillator bars (72) - \$9.6k
    \item CAEN V1743 digitizers (3) - \$22.5k
    \item Amplifier and filter circuit  - \$0.2k
    \item HV and readout cables - \$1.5k
    \item Scintillator sheets - \$3.6k
    \item Travel/shipping to CERN - \$5k
    \item Mechanical structure - \$0.8k
\end{itemize}

The running costs will be covered by an agreement with CERN colleagues that has 
worked well in operation of the phase 1 prototype. The following items will be reused, at no 
cost, from the phase 1 prototype: HV and readout cables for 31 channels, two CAEN V1743 digitizer boards,
HV supply, support structure, and two scintillator slabs. The R878 PMTs that will be coupled to each scintillator volume
are available, again at no cost, from a previous experiment. My salary will be paid by
Dr. David Stuart.

\subsection*{Timeline}

The next run of the LHC is expected to begin in late 2021, providing a clear target 
for the completion of the detector. Based on our experience constructing the initial prototype, 
approximately six months is estimated for the construction and a further six months is
estimated for commissioning. The detector would be expected to 
collect $100~\text{fb}^{-1}$ in each of 2022, 2023 and 2024.
Given the experience gained from the prototype, I believe an initial publication with the data set collected 
in 2021-2022 could be achieved by early to mid 2023 with further publications with enlarged datasets 
in early 2024 and 2025. The milliQan detector can then be further 
upgraded during next the long shutdown (2024--2027), with the upgraded
detector taking data at the High Luminosity LHC with a significantly increased instantaneous luminosity from 2027. 




%% end of file `template.tex'.

\bibliographystyle{unsrt}
\bibliography{citron_PROP} % 'citron' is the name of a BibTeX file


\end{document}

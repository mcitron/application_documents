\documentclass[11pt]{article}
\usepackage{amsmath}
\usepackage{cancel}
\usepackage{xspace}
\usepackage{amssymb}
\usepackage{amsthm}
\usepackage{amscd}
\usepackage{amsfonts}
\usepackage{graphicx}%
\usepackage{fancyhdr}

% \usepackage[a4paper, total={7in, 8in}]{geometry}
\usepackage[bottom=0.4in,left=0.8in,right=0.8in]{geometry}
\usepackage{fancyhdr}

\theoremstyle{plain} \numberwithin{equation}{section}
\newtheorem{theorem}{Theorem}[section]
\newtheorem{corollary}[theorem]{Corollary}
\newtheorem{conjecture}{Conjecture}
\newtheorem{lemma}[theorem]{Lemma}
\newtheorem{proposition}[theorem]{Proposition}
\theoremstyle{definition}
\newtheorem{definition}[theorem]{Definition}
\newtheorem{finalremark}[theorem]{Final Remark}
\newtheorem{remark}[theorem]{Remark}
\newtheorem{example}[theorem]{Example}
\newtheorem{question}{Question} \topmargin-2cm

% \textwidth7in
%
% \setlength{\topmargin}{0in} \addtolength{\topmargin}{-\headheight}
% \addtolength{\topmargin}{-\headsep}
%
% \setlength{\oddsidemargin}{0in}
%\newcommand{\alphat}{\ensuremath{\alpha_{\text{T}}}\xspace}
\DeclareRobustCommand{\alphat}{$\alpha_{\text{T}}~$}
\DeclareRobustCommand{\met}{$\mbox{$E_\text{T}^{\rm miss}$}\xspace$}


% \oddsidemargin  0.0in \evensidemargin 0.0in

\pagestyle{fancy}\lhead{Matthew Citron}\rhead{July 2020}
\chead{{\large{\bf Research proposal for the Karp Discovery Award}}} \lfoot{} \cfoot{\bf \thepage}

\newcounter{list}

\begin{document}
\section*{Motivation and outline}
%\section*{\fontsize{30}{30}\selectfont Statement of intent}
\noindent 

% The phase-1 milliQan experiment is a unique opportunity for a 
% new small-scale detector to provide a significant extension to the search 
% program at the CERN LHC, with a large discovery potential 
% for as yet unknown particles with a small electric charge, a ``millicharge''. 
% This award would fully fund the construction of a new detector,
% unlocking new territory in the search for millicharged particles and
% potentially uncovering the link to a hidden universe.
%
% \subsection*{Motivation}

One of the greatest mysteries in particle physics is the nature of dark matter.
While astrophysical evidence suggests that such dark matter particles 
make up approximately 85\% of the matter
in the universe, they have never been observed directly. In recent years, 
there has been increasing interest in the idea that dark matter points to the 
existence of not just a single new particle but an entire dark sector 
with a structure that could be as rich and complex as our own. We have not been able to
observe this dark universe because most dark sector particles cannot interact with
visible matter. However, it is possible that a quantum connection exists between our two worlds. 
If a small ``kinetic" mixing is introduced between visible and dark photons this
leads to the existence of dark particles with a small electric charge, a ``millicharge''.
Such particles are extremely difficult to detect as they interact very weakly with
visible matter, however, their discovery would revolutionize our understanding
of the universe.

At the end of 2021, the most powerful particle collider ever built, the CERN LHC, 
will resume proton-proton collisions. Large numbers of millicharge particles may be 
produced in these collisions, however, the multi-billion dollar general purpose
detectors, CMS and ATLAS, will not have sensitivity as millicharged particles interact
too weakly to be observed by their detector designs. 
A dedicated experiment, milliQan, is required to take advantage of 
this unique opportunity for the discovery of millicharged particles at the LHC.

Over the past few years I have led the design, operation and analysis
of data from a small prototype of the milliQan detector. Using this data I have coordinated
a search for millicharge particles with world-leading sensitivity, which has been submitted 
for publication. Given this experience and the approaching next run of the LHC, I believe this is an ideal 
time to construct an upgraded detector that overcomes the limitations of the prototype and takes 
advantage of the upcoming LHC running to significantly extend sensitivity to millicharge particles. A full-scale 
detector will cost around \$2 million, however, the timeline for such a detector to be funded and built
is too long to be completed in time for next year's LHC running.  

In this proposal, based on my experience with the prototype detector,
I outline a detailed motivation and design 
for a new millicharge particle detector.
With a budget of less than \$50k, this detector can 
achieve a dramatic extension in sensitivity 
to millicharge particles, potentially uncovering the link
to a hidden universe.
In addition, this detector will provide 
important experience for a future larger detector
that may be built during the next shutdown of the LHC (from late 2024), 
provided funding becomes available.

%
\section*{Searching for millicharge particles at the LHC}
%
Previous experiments carried out at the Stanford Linear Accelerator Center (SLAC) 
searched for millicharged particles with masses up to about the electron mass, 0.5 MeV.
The LHC's much higher beam energy will allow the proposed detector to extend 
sensitivity to $\sim 100\,000$~MeV.
The existing, general purpose detectors at the LHC are blind to these particles because 
fractionally charged particles deposit only $(Q/e)^2$ of the energy 
deposited by a particle with charge $e$ of
the same mass. Therefore, a dedicated experiment is needed to provide sensitivity to the signatures of 
millicharged particles with $Q< 0.1 e$.
The proposed experiment achieves this through the use of an array of long scintillator
bars properly aligned with a small point where the LHC's proton beams are brought together for collisions. 

In late 2017, a small prototype detector was installed to characterise 
the backgrounds and prove the feasibility of such an experiment.
The prototype consisted of eighteen $5~\rm{cm}\times5~\rm{cm}\times80~\rm{cm}$ 
scintillator bars arranged in three layers. This detector ran successfully 
from March 2018 to May 2019, providing crucial
experience in the operation of a dedicated scintillator-based detector. 
The $37.5~\text{fb}^{-1}$ of data collected by the milliQan detector was used to carry out
a search for fractionally charged particles with world-leading sensitivity.
Despite this success, the sensitivity of the prototype was limited by exposure time, 
low efficiency for small energy deposits,
and background processes that became well understood through analysis of its data.

This proposal is to install a more extensive scintillator-based detector
to search for millicharge particles with world-leading sensitivity in the upcoming LHC running.
The design of the experiment, called the phase-1 milliQan detector, is motivated by 
the comprehensive understanding of the backgrounds and performance gained during the
operation of the prototype.

\subsection*{The phase-1 milliQan detector}

The proposed phase-1 milliQan detector is
composed of a $0.2~\mathrm{m}\times0.2~\mathrm{m}\times3.2~\mathrm{m}$ plastic scintillator array aligned with 
the CMS interaction point. The $3.2~\rm{m}$ active path length through the scintillator determines the sensitivity 
of the detector and is therefore designed to be as large as possible while fitting the constraints of the experimental cavern.
The scintillator arrays will be comprised of 48 $5~\rm{cm}\times5~\rm{cm}\times60~\rm{cm}$ 
and 16 $5~\rm{cm}\times5~\rm{cm}\times80~\rm{cm}$ bars arranged into four layers of $4\times4$ scintillator bars. 
The bars with length of $80\rm{cm}$ will be taken, without cost, from the prototype. Each scintillator bar must be optically
coupled to a photomultiplier tube (PMT) to readout the scintillator light. The PMTs used will be 
Hamamatsu R878s, available at no cost from a previous experiment. The use of multiple layers allows backgrounds to be mitigated
by requiring a hit in each layer of the array within a small ($\sim 15$~ns) time window. The prototype detector, comprised
of three layers, faced a dominant background from correlated cosmic ray muon showers and environmental radiation sources.
By expanding to four layers, this background contribution can be greatly reduced to only a few events per year.
In addition, the expansion in size of each layer, from six to sixteen bars, will significantly improve
both the signal acceptance and active veto performance of the detector. 
In addition to the bars, two ``slabs'' of $20~\rm{cm}\times30~\rm{cm}\times2.5~\rm{cm}$ scintillator and 
six thin ``panels'' of $25~\rm{cm}\times160~\rm{cm}\times1~\rm{cm}$ 
will be inserted to tag, study, and actively veto backgrounds from charged particles such as muons from the LHC, cosmic radiation, and ambient radioactivity.
Analog pulses from the PMTs must be read out,
digitized, and stored for offline analysis. This can be achieved using CAEN V1743 waveform digitizers. 
These have 16 channels, each of which provide a high sample rate (up to 3.2 GS/s) and a long readout window (1024 samples), 
suitable for digitizing the waveforms generated by the PMTs. The CAEN digitizers also allow a flexible triggering
scheme based on the coincidence of pulses within a small time window in the PMTs in different layers. 
Experience from the prototype shows that this triggering scheme works well, with a trigger rate of a few Hz even with the typical 
per-channel single photon pulse rate of up to $\sim 5$ kHz.
The phase-1 milliQan detector will require a total of 73 channels (64 bars, 2 slabs and 6 panels and one channel
for a 40 MHz beam synchronization signal from the LHC) requiring five CAEN V1743 digitizers. 

To provide sensitivity to the small energy deposits from charges as low as $0.001 e$,
each PMT must be capable of detecting a single scintillation photon.
In operating the prototype, it was found that Hamamatsu R878 PMTs have insufficient gain and large high frequency noise components
that make triggering and reconstructed the pulses from single scintillation photons impossible.
A simple amplifier incorporating a low-pass filter has been designed at UCSB to reject this noise and
allow such signals to be detected. In combination with the upgrade to four layers of scintillator bars
and enlarged active veto capability, this will allow the upgraded detector to have dramatically improved sensitivity.

The background for the four layer configuration has been conservatively estimated
through special runs of the prototype to be 0.2$\times10^{-3}$ events per hour. 
To collect the full 300 $\rm{fb}^{-1}$ during the next run of the LHC 
requires a trigger live time of $4.2\times 10^3$ hours, which corresponds 
to an expected total background of only 0.8 events. With the ability to efficiently trigger
and reconstruct single photon events, this will allow sensitivity to charges down to $Q=0.003 e$ for particles of
mass $< 1\,000$ MeV and extend the best sensitivity in charge for millicharged particles of mass $< 40\,000$ MeV
by more than an order of magnitude.

\subsection*{Construction and operation}

The bulk of the construction will be the assembly of the 48 new bars and 6 panels.
Each assembly consists of optically coupling a PMT held in a base
to the scintillator, wrapping it with reflective material, and light-tight black
taping. Once assembled, each scintillator volume must be checked for light leaks and calibrated.
This work is relatively simple and, as for the prototype, can be largely undertaken
by UCSB undergraduate researchers, with my supervision. Once assembled, the bars and panels
must be sent to CERN to be installed into the detector. This work would be 
undertaken by postdoc and graduate student members of the milliQan collaboration 
(who are funded separately from this award). The support structure has been designed and would be 
constructed by a technician at UCSB or CERN. Once constructed, the detector will be operated
remotely by members of the milliQan collaboration. Given the experience with the
prototype, this will be carried out with high efficiency to collect physics quality
data. Similar strategies to the prototype will be used to calibrate the response and timing of the 
detector, simulate signal and background processes and analyse the data to 
search for millicharged particles.

\section*{Budget}

The budget for the phase-1 milliQan detector to be covered by this award corresponds to total of \$45.6k and
breaks down as follows:

\begin{itemize}
    \item EJ-200 plastic scintillator bars (48) - \$10.3k
    \item CAEN V1743 digitizers (3) - \$21k
    \item Amplifier and filter circuit (72)  - \$0.2k
    \item Scintillator panels (6) - \$3.6k
    \item High voltage and readout cables - \$1.5k
    \item Travel/shipping to CERN - \$4k
    \item Mechanical structure/engineering costs - \$5k
\end{itemize}

The running costs will be covered by an agreement with CERN colleagues that has 
worked well in operation of the prototype. The following items will be reused, at no 
cost, from the prototype: HV and readout cables for 32 channels, two CAEN V1743 digitizer boards,
HV supply, steel support frame, and two scintillator slabs. The PMTs that will be coupled to each scintillator volume
are available, at no cost, from a previous experiment. My salary will continue to be paid by Dr. David Stuart 
using a DOE grant that includes support for exploration of new ideas such as this. 

\section*{Timeline}

The next run of the LHC is expected to begin in late 2021, providing a clear target 
for the completion of the detector. Based on our experience constructing the initial prototype, 
approximately six months is estimated for the construction (mainly in assembling the bars),
and a further six months is estimated for commissioning. The detector would be expected to 
collect $100~\text{fb}^{-1}$ in each of 2022, 2023 and 2024.
Given the experience gained from the prototype, I believe an initial publication with the data set collected 
in 2021--2022 could be achieved by early to mid 2023 with further publications with enlarged data sets 
in early 2024 and 2025. The milliQan detector can then be further 
upgraded during the next LHC shutdown period (2024--2027), with the upgraded
detector taking data at the High Luminosity LHC with a significantly increased instantaneous luminosity beginning in 2027. 


%% end of file `template.tex'.

% \bibliographystyle{unsrt}
% \bibliography{citron_PROP} % 'citron' is the name of a BibTeX file


\end{document}
